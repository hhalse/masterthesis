\section{Introduction to cyber-insurance}
Security breaches are increasingly prevalent in the Internet age causing huge financial losses
for companies and their users. When facing security breaches and risk, there are typically four ways to act \cite{bolot:cyber}:
1. Avoid the risk, 2. Retain the risk, 3. Self protect and mitigate the risk and 4. Transfer the risk.

%\begin{enumerate}
%\item Avoid the risk
%\item Retain the risk
%\item Self protect and mitigate the risk
%\item Transfer the risk
%\end{enumerate}
The ICT industry have so far tried to prevent risks with a mixture of options two and three. This has lead to many different techniques and software trying to detect threats and anomalies, to protect the users and infrastructure. Firewalls, intrusion- detection and prevention systems, are some of the solutions. These will reduce the risk, but do not eliminate the risk completely. Although they are all good and needed actions, it is impossible to achieve perfect cyber-security, due to many reasons: Threats are continuously evolving, there will always be accidents and security flaws, attackers have different intentions, network externalities and free-riding in security networks, the lemons-market in security products, misaligned incentives between users and product vendors, and many more. 
This is why we need cyber-insurance, as an fourth option, to handle the residual risk \cite{bolot:cyber2,ranjan:cyber}.

The market for cyber-insurance emerged in the late 80's, when security software companies began collaborating with insurance companies to offer insurance policies together with their security products. From a marketing perspective, adding insurance helped highlighting the supposedly high quality of the security software. Nevertheless, this new product was a comprehensive solution, which dealt with both risk reduction and residual risk \cite{bolot:new}. Continuing into the beginning of the new millennium, several companies started offering standalone cyber-insurance, which sat the frame for the current insurance product. However, as found in the papers \cite{ccost,evolvingcyber,CFCunder} both the US and the european market have not managed to reach its promising potential. Companies are weekly suffering from successful attacks, but most firms still have not acquired cyber-insurance. 
Researchers claim that the main reasons that cyber-insurance is struggling is the three unique problems, information asymmetry, correlated risk and interdependent agents \cite{bohme2010modeling}. 
\paragraph{Information asymmetry.}
Information asymmetry arises when one side of a transaction or decision has more or better information than the other party. There are two different cases of information asymmetry. The first one is called adverse selection, where one party simply has less information regarding the performance of the transaction. A good example for the cyber-industry, where an insurer has trouble confirming whether your computer/network is "healthy".
The other information asymmetry scenario is called moral hazard. It occurs after the signing of the contract, where one party deliberately takes some action that makes the possibility of loss higher, e.g. choosing not to lock your door. 
      
\paragraph{Correlated risk.}

Another big concern regarding cyber-insurance, is the correlated risk. Among other things, the problem occurs due to the need for standards. Standardization is an important part of the business of computers and computer networks. Generally it enables computers to communicate, install and use different software. A good example is operative systems for personal computers, today we only have a small set of operative systems available, and these systems are standardized, so they can use the same communication channels. The standards generate a lot of the value in the ICT industry, but they also make many threats possible. All systems that use the same standards, create a large number of similar exposure units which could be exploited at the same time. 
Thus create a significant difficulty for the cyber-insurance industry, because when a security breach occurs there is a high probability that it will occur to a large number of people.
 
 \paragraph{Interdependent security.}
Another problem in the ICT industry is interdependent security, meaning that you are not only dependent on your own investment in security, but also on everyone else's. 
Investment in security generates positive externalities, and as public goods, this encourages free riding. The problem is that the reward for a user investing in self-protection depends on the security in the rest of the network. i.e. The expected loss due to a security breach at one agent in the network, is not only dependent on this agent's level of investment in security, but also on the security investment done by adjacent agents, and their adjacent agents and so forth.
  
The cyber-insurance market seem to have a huge potential, but needs some new thinking to fully take advantage of it. We will take a new approach where we focus on finding network structures that will be beneficial for cyber-insurance, and see if it i possible for insurers to force these structures to evolve.

