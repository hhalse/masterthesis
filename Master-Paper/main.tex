% This is LLNCS.DOC the documentation file of
% the LaTeX2e class from Springer-Verlag
% for Lecture Notes in Computer Science, version 2.4
\documentclass{llncs}
\usepackage{llncsdoc}
%
\begin{document}

\title{Hamiltonian Mechanics}

\author{Ivar Ekeland\inst{1} \and Roger Temam\inst{2}}

\institute{Princeton University, Princeton NJ 08544, USA
\and
Universit\'{e} de Paris-Sud,
Laboratoire d'Analyse Num\'{e}rique, B\^{a}timent 425,\\
F-91405 Orsay Cedex, France}

\maketitle
%
\begin{abstract}
This paragraph shall summarize the contents of the paper
in short terms.
\end{abstract}
%
\section{Fixed-Period Problems: The Sublinear Case}
%
With this chapter, the preliminaries are over, and we begin the search
for periodic solutions \dots
%
\subsection{Autonomous Systems}
%
In this section we will consider the case when the Hamiltonian
$H(x)$ \dots
%
\subsubsection{The General Case: Nontriviality.}
%
We assume that $H$ is
$\left(A_{\infty}, B_{\infty}\right)$-subqua\-dra\-tic at
infinity, for some constant \dots
%
\paragraph{Notes and Comments.}
The first results on subharmonics were \dots
%
\begin{proposition}
Assume $H'(0)=0$ and $ H(0)=0$. Set \dots
\end{proposition}
\begin{proof}[of proposition]
Condition (8) means that, for every $\delta'>\delta$, there is
some $\varepsilon>0$ such that \dots \qed
\end{proof}
%
\begin{example}[{{\rmfamily External forcing}}]
Consider the system \dots
\end{example}
\begin{corollary}
Assume $H$ is $C^{2}$ and
$\left(a_{\infty}, b_{\infty}\right)$-subquadratic
at infinity. Let \dots
\end{corollary}
\begin{lemma}
Assume that $H$ is $C^{2}$ on $\bbbr^{2n}\backslash \{0\}$
and that $H''(x)$ is \dots
\end{lemma}
\begin{theorem}[Ghoussoub-Preiss]
Let $X$ be a Banach Space and $\Phi:X\to\bbbr$ \dots
\end{theorem}
\begin{definition}
We shall say that a $C^{1}$ function $\Phi:X\to\bbbr$ satisfies \dots
\end{definition}
%
\section{Fine Tuning of the Text}
%
The following should be used to improve the readability of the text:
\begin{flushleft}
\begin{tabular}{@{}p{.19\textwidth}p{.79\textwidth}}
\verb|\,|   & a thin space, e.g.\ between numbers or between units
              and num\-bers; a line division will not be made
              following this space\\
\verb|--|   & en dash; two strokes, without a space at either end\\
\verb*| -- |& en dash; two strokes, with  a space at either end\\
\verb|-|    & hyphen; one stroke, no space at either end\\
\verb|$-$|  & minus, in the text {\em only} \\[8mm]
{\em Input} & \verb|21\,$^{\circ}$C etc.,|\\
            &  \verb|Dr h.\,c.\,Rockefellar-Smith \dots|\\
            & \verb|20,000\,km and  Prof.\,Dr Mallory \dots|\\
            & \verb|1950--1985 \dots|\\
            & \verb|this -- written on a computer -- is now printed|\\
            & \verb|$-30$\,K \dots|\\[3mm]
{\em Output}& 21\,$^{\circ}$C etc., Dr h.\,c.\,Rockefellar-Smith \dots\\
            & 20,000\,km and  Prof.\,Dr Mallory \dots\\
            & 1950--1985 \dots\\
            & this -- written on a computer -- is now printed\\
            & $-30$\,K \dots
\end{tabular}
\end{flushleft}
%
\section {Special Typefaces}
%
Normal type (roman text) need not be coded. {\itshape Italic}
(\verb|{\em <text>}| better still \verb|\emph{<text>}|) or, if
necessary, {\bfseries boldface} should be used for emphasis.\\[6pt]
\begin{minipage}[t]{\textwidth}
\begin{flushleft}
\begin{tabular}{@{}p{.25\textwidth}@{\hskip6pt}p{.73\textwidth}@{}}
\verb|{\itshape Text}|   & {\itshape Italicized Text}\\[2pt]
\verb|{\em Text}|   & {\em Emphasized Text --
   if you would like to emphasize a {\em definition} within an
   italicized text (e.g.\ of a {\em theorem)} you should code the
   expression to be emphasized by} \verb|\em|.\\[2pt]
\verb|{\bfseries Text}|& {\bfseries Important Text}\\[2pt]
\verb|\vec{Symbol}| & Vectors may only appear in math mode. The default
   \LaTeX{} vector symbol has been adapted\footnotemark\
 to LLNCS conventions.\\[2pt]
 & \verb|$\vec{A \times B\cdot C}| yields $\vec{A\times B\cdot C}$\\
 & \verb|$\vec{A}^{T} \otimes \vec{B} \otimes|\\
 & \verb|\vec{\hat{D}}$|yields $\vec{A}^{T} \otimes \vec{B} \otimes
\vec{\hat{D}}$
\end{tabular}
\end{flushleft}
\end{minipage}

\footnotetext{If you absolutely must revive the original \LaTeX{}
design of the vector symbol (as an arrow accent), please specify the
option \texttt{[orivec]} in the \texttt{documentclass} line.}
\newpage
%
\section {Footnotes}
%
Footnotes within the text should be coded:
\begin{verbatim}
\footnote{Text}
\end{verbatim}
{\itshape Sample Input}
\begin{flushleft}
Text with a footnote\verb|\footnote{The |{\tt footnote is automatically
numbered.}\verb|}| and text continues \dots
\end{flushleft}
{\itshape Sample Output}
\begin{flushleft}
Text with a footnote\footnote{The footnote is automatically numbered.}
and text continues \dots
\end{flushleft}
%
\section {Lists}
%
Please code lists as described below:\\[2mm]
{\itshape Sample  Input}
\begin{verbatim}
\begin{enumerate}
  \item First item
  \item Second item
  \begin{enumerate}
    \item First nested item
    \item Second nested item
  \end{enumerate}
  \item Third item
\end{enumerate}
\end{verbatim}
{\itshape Sample Output}
 \begin{enumerate}
\item First item
\item Second item
  \begin{enumerate}
    \item First nested item
    \item Second nested item
  \end{enumerate}
\item Third item
\end{enumerate}
%
\section {Figures}
%
Figure environments should be inserted after (not in)
the  paragraph in which the figure is first mentioned.
They will be numbered automatically.

Preferably the images should be enclosed as PostScript files -- best as
EPS data using the epsfig package.

If you cannot include them into your output this way and use other
techniques for a separate production,
the figures (line drawings and those containing halftone inserts
as well as halftone figures) {\em should not be pasted into your
laserprinter output}. They should be enclosed separately in camera-ready
form (original artwork, glossy prints, photographs and/or slides). The
lettering should be suitable for reproduction, and after a
probably necessary reduction the height of capital letters should be at
least 1.8\,mm and not more than 2.5\,mm.
Check that lines and other details are uniformly black and
that the lettering on figures is clearly legible.

To leave the desired amount of space for the height of
your figures, please use the coding described below.
As can be seen in the output, we will automatically
provide 1\,cm space above and below the figure,
so that you should only leave the space equivalent to the size of the
figure itself. Please note that ``\verb|x|'' in the following
coding stands for the actual height of the figure:
\begin{verbatim}
\begin{figure}
\vspace{x cm}
\caption[ ]{...text of caption...}          (Do type [ ])
\end{figure}
\end{verbatim}
\begin{flushleft}
{\itshape Sample Input}
\end{flushleft}
\begin{verbatim}
\begin{figure}
\vspace{2.5cm}
\caption{This is the caption of the figure displaying a white
eagle and a white horse on a snow field}
\end{figure}
\end{verbatim}
\begin{flushleft}
{\itshape Sample Output}
\end{flushleft}
\begin{figure}
\vspace{2.5cm}
\caption{This is the caption of the figure displaying a white eagle and
a white horse on a snow field}
\end{figure}
%
\section{Tables}
%
Table captions should be treated
in the same way as figure legends, except that
the table captions appear {\itshape above} the tables. The tables
will be numbered automatically.
%
\subsection{Tables Coded with \protect\LaTeX{}}
%
Please use the following coding:\\[2mm]
{\itshape Sample Input}
\begin{verbatim}
\begin{table}
\caption{Critical $N$ values}
\begin{tabular}{llllll}
\hline\noalign{\smallskip}
${\mathrm M}_\odot$ & $\beta_{0}$ & $T_{\mathrm c6}$ & $\gamma$
  & $N_{\mathrm{crit}}^{\mathrm L}$
  & $N_{\mathrm{crit}}^{\mathrm{Te}}$\\
\noalign{\smallskip}
\hline
\noalign{\smallskip}
 30 & 0.82 & 38.4 & 35.7 & 154 & 320 \\
 60 & 0.67 & 42.1 & 34.7 & 138 & 340 \\
120 & 0.52 & 45.1 & 34.0 & 124 & 370 \\
\hline
\end{tabular}
\end{table}
\end{verbatim}

\medskip\noindent{\itshape Sample Output}
\begin{table}
\caption{Critical $N$ values}
\begin{center}
\renewcommand{\arraystretch}{1.4}
\setlength\tabcolsep{3pt}
\begin{tabular}{llllll}
\hline\noalign{\smallskip}
${\mathrm M}_\odot$ & $\beta_{0}$ & $T_{\mathrm c6}$ & $\gamma$
  & $N_{\mathrm{crit}}^{\mathrm L}$
  & $N_{\mathrm{crit}}^{\mathrm{Te}}$\\
\noalign{\smallskip}
\hline
\noalign{\smallskip}
 30 & 0.82 & 38.4 & 35.7 & 154 & 320 \\
 60 & 0.67 & 42.1 & 34.7 & 138 & 340 \\
120 & 0.52 & 45.1 & 34.0 & 124 & 370 \\
\hline
\end{tabular}
\end{center}
\end{table}

Before continuing your text you need an empty line. \dots

\vspace{3mm}
For further information you will find a complete description of
the tabular environment
on p.~62~ff. and p.~204 of the {\em \LaTeX{} User's Guide \& Reference
Manual\/} by Leslie Lamport.
%
\subsection{Tables Not Coded with \protect\LaTeX{}}
%
If you do not wish to code your table using \LaTeX{}
but prefer to have it reproduced separately,
proceed as for figures and use the following coding:\\[2mm]
{\itshape Sample Input}
\begin{verbatim}
\begin{table}
\caption{text of your caption}
\vspace{x cm}     % the actual height needed for your table
\end{table}
\end{verbatim}
%
\subsection{Signs and Characters}
%
\subsubsection*{Special Signs.}
%
You may need to use special signs.  The available ones are listed in the
{\em \LaTeX{} User's Guide \& Reference Manual\/} by Leslie Lamport,
pp.~41\,ff.
We have created further symbols for math mode (enclosed in \$):
\begin{center}
\begin{tabular}{l@{\hspace{1em}yields\hspace{1em}}
c@{\hspace{3em}}l@{\hspace{1em}yields\hspace{1em}}c}
\verb|\grole| & $\grole$ & \verb|\getsto| & $\getsto$\\
\verb|\lid|   & $\lid$   & \verb|\gid|    & $\gid$
\end{tabular}
\end{center}
%
\subsubsection*{Gothic (Fraktur).}
%
If gothic letters are {\itshape necessary}, please use those of the
relevant \AmSTeX{} alphabet which are available using the amstex
package of the American Mathematical Society.

In \LaTeX{} only the following gothic letters are available:
\verb|$\Re$| yields $\Re$ and \verb|$\Im$| yields $\Im$. These should
{\itshape not\/} be used when you need gothic letters for your contribution.
Use \AmSTeX{} gothic as explained above. For the real and the imaginary
parts of a complex number within math mode you should use instead:
\verb|$\mathrm{Re}$| (which yields Re) or \verb|$\mathrm{Im}$| (which
yields Im).
%
\subsubsection*{Script.}
%
For script capitals use the coding
\begin{center}
\begin{tabular}{l@{\hspace{1em}which yields\hspace{1em}}c}
\verb|$\mathcal{AB}$| & $\mathcal{AB}$
\end{tabular}
\end{center}
(see p.~42 of  the \LaTeX{} book).
%
\subsubsection*{Special Roman.}
%
If you need other symbols than those below, you could use
the blackboard bold characters of \AmSTeX{},  but there might arise
capacity problems
in loading additional \AmSTeX{} fonts. Therefore  we created
the blackboard bold characters listed below.
Some of them are not esthetically
satisfactory. This need not deter you from using them:
in the final printed form they will be
replaced by the well-designed MT (monotype) characters of
the phototypesetting machine.
\begin{flushleft}
\begin{tabular}{@{}ll@{ yields }
c@{\hspace{1.em}}ll@{ yields }c}
\verb|\bbbc| & (complex numbers)   & $\bbbc$
  & \verb|\bbbf| & (blackboard bold F) & $\bbbf$\\
\verb|\bbbh| & (blackboard bold H) & $\bbbh$
  & \verb|\bbbk| & (blackboard bold K) & $\bbbk$\\
\verb|\bbbm| & (blackboard bold M) & $\bbbm$
  & \verb|\bbbn| & (natural numbers N) & $\bbbn$\\
\verb|\bbbp| & (blackboard bold P) & $\bbbp$
  & \verb|\bbbq| & (rational numbers)  & $\bbbq$\\
\verb|\bbbr| & (real numbers)      & $\bbbr$
  & \verb|\bbbs| & (blackboard bold S) & $\bbbs$\\
\verb|\bbbt| & (blackboard bold T) & $\bbbt$
  & \verb|\bbbz| & (whole numbers)     & $\bbbz$\\
\verb|\bbbone| & (symbol one)      & $\bbbone$
\end{tabular}
\end{flushleft}
\begin{displaymath}
\begin{array}{c}
\bbbc^{\bbbc^{\bbbc}} \otimes
\bbbf_{\bbbf_{\bbbf}} \otimes
\bbbh_{\bbbh_{\bbbh}} \otimes
\bbbk_{\bbbk_{\bbbk}} \otimes
\bbbm^{\bbbm^{\bbbm}} \otimes
\bbbn_{\bbbn_{\bbbn}} \otimes
\bbbp^{\bbbp^{\bbbp}}\\[2mm]
\otimes
\bbbq_{\bbbq_{\bbbq}} \otimes
\bbbr^{\bbbr^{\bbbr}} \otimes
\bbbs^{\bbbs_{\bbbs}} \otimes
\bbbt^{\bbbt^{\bbbt}} \otimes
\bbbz \otimes
\bbbone^{\bbbone_{\bbbone}}
\end{array}
\end{displaymath}
%
\section{References}
\label{refer}
%
There are three reference systems available; only one, of course,
should be used for your contribution. With each system (by
number only, by letter-number or by author-year) a reference list
containing all citations in the
text, should be included at the end of your contribution placing the
\LaTeX{} environment \verb|thebibliography| there.
For an overall information on that environment
see the {\em \LaTeX{} User's Guide \& Reference
Manual\/} by Leslie Lamport, p.~71.

There is a special {\sc Bib}\TeX{} style for LLNCS that works along
with the class: \verb|splncs.bst|
-- call for it with a line \verb|\bibliographystyle{splncs}|.
If you plan to use another {\sc Bib}\TeX{} style you are customed to,
please specify the option \verb|[oribibl]| in the
\verb|documentclass| line, like:
\begin{verbatim}
\documentclass[oribibl]{llncs}
\end{verbatim}
This will retain the original \LaTeX{} code for the bibliographic
environment and the \verb|\cite| mechanism that many {\sc Bib}\TeX{}
applications rely on.
%
\subsection{References by Letter-Number or by Number Only}
%
References are cited in the text -- using the \verb|\cite|
command of \LaTeX{} -- by number or by letter-number in square
brackets, e.g.\ [1] or [E1, S2], [P1], according to your use of the
\verb|\bibitem| command in the \verb|thebibliography| environment. The
coding is as follows: if you choose your own label for the sources by
giving an optional argument to the \verb|\bibitem| command the citations
in the text are marked with the label you supplied. Otherwise a simple
numbering is done, which is preferred.
\begin{verbatim}
The results in this section are a refined version
of \cite{clar:eke}; the minimality result of Proposition~14
was the first of its kind.
\end{verbatim}
The above input produces the citation: ``\dots\ refined version of
[CE1]; the min\-i\-mality\dots''. Then the \verb|\bibitem| entry of
the \verb|thebibliography| environment should read:
\begin{verbatim}
\begin{thebibliography}{[MT1]}
.
.
\bibitem[CE1]{clar:eke}
Clarke, F., Ekeland, I.:
Nonlinear oscillations and boundary-value problems for
Hamiltonian systems.
Arch. Rat. Mech. Anal. {\bfseries 78} (1982) 315--333
.
.
\end{thebibliography}
\end{verbatim}
The complete bibliography looks like this:
%
\begin{thebibliography}{[MT1]}
%
\bibitem[CE1]{clar:eke}
Clarke, F., Ekeland, I.:
Nonlinear oscillations and
boundary-value problems for Hamiltonian systems.
Arch. Rat. Mech. Anal. {\bfseries 78} (1982) 315--333
%
\bibitem[CE2]{clar:eke:2}
Clarke, F., Ekeland, I.:
Solutions p\'{e}riodiques, du
p\'{e}riode donn\'{e}e, des \'{e}quations hamiltoniennes.
Note CRAS Paris {\bfseries 287} (1978) 1013--1015
%
\bibitem[MT1]{mich:tar}
Michalek, R., Tarantello, G.:
Subharmonic solutions with prescribed minimal
period for nonautonomous Hamiltonian systems.
J. Diff. Eq. {\bfseries 72} (1988) 28--55
%
\bibitem[Ta1]{tar}
Tarantello, G.:
Subharmonic solutions for Hamiltonian
systems via a $\bbbz_{p}$ pseudoindex theory.
Annali di Matematica Pura (to appear)
%
\bibitem[Ra1]{rab}
Rabinowitz, P.:
On subharmonic solutions of a Hamiltonian system.
Comm. Pure Appl. Math. {\bfseries 33} (1980) 609--633
\end{thebibliography}
%
\subsubsection*{Number-Only System.}
%
For this preferred system do not use the optional argument
in the \verb|\bibitem| command: then, only numbers will
appear for the citations in the text (enclosed in square brackets)
as well as for the marks in your
bibliography (here the number is only end-punctuated without
square brackets).

Subsequent citation numbers in the text are collapsed to ranges.
Non-numeric and undefined labels are handled correctly but no sorting is
done.

E.g., \verb|\cite{n1,n3,n2,n3,n4,n5,foo,n1,n2,n3,?,n4,n5}| -- where
\verb|n|$x$ is the key of the $x^{\mathrm{th}}$ \verb|\bibitem|
command in sequence, \verb|foo| is the key of a \verb|\bibitem| with an
optional argument, and \verb|?| is an undefined reference -- gives
1,3,2-5,foo,1-3,?,4,5 as the citation reference.

\begin{verbatim}
\begin{thebibliography}{1}
\bibitem {clar:eke}
Clarke, F., Ekeland, I.:
Nonlinear oscillations and boundary-value problems for
Hamiltonian systems.
Arch. Rat. Mech. Anal. {\bfseries 78} (1982) 315--333
\end{thebibliography}
\end{verbatim}
%
\subsection{Author-Year System}
%
References are cited in the text by name and year in parentheses
and should look as follows:
(Smith 1970, 1980), (Ekeland et al. 1985, Theorem 2), (Jones and Jaffe
1986; Farrow 1988, Chap.\,2). If the name is part of the sentence
only the year may appear in parentheses,
e.g.\ Ekeland et al. (1985, Sect.\,2.1)
The reference list should contain all citations occurring in the text,
ordered alphabetically by surname (with initials following). If there
are several works by the same author(s) the references should be listed
in the appropriate order indicated below:
\begin{alpherate}
\setlength{\hfuzz}{5pt}
\item
One author: list works chronologically;
\item
Author and same co-author(s): list works chronologically;
\item
Author and different co-authors: list works alphabetically
according to co-authors.
\end{alpherate}
If there are several works by the same author(s) and in the same year,
but which are cited separately, they should be distinguished by the use
of ``a'', ``b'' etc., e.g.\ (Smith 1982a), (Ekeland et al. 1982b).
%
\subsubsection*{How to Code Author-Year System.}
%
If you want to use this system you have to specify the option
\verb|[citeauthoryear]| in the \verb|documentclass|, like:
\begin{verbatim}
\documentclass[citeauthoryear]{llncs}
\end{verbatim}
Write your citations in the text explicitly except for the year, leaving
that up to \LaTeX{} with the \verb|\cite| command. Then give only the
appropriate year as the optional argument (i.e. the label in square
brackets) with the \verb|\bibitem| command(s).\\[2mm]
{\itshape Sample Input}
\begin{verbatim}
The results in this section are a refined version
of Clarke and Ekeland (\cite{clar:eke}); the minimality result of
Proposition~14 was the first of its kind.
\end{verbatim}
The above input produces the citation: ``\dots\ refined version of
Clarke and Ekeland (1982); the minimality\dots''. Then the
\verb|\bibitem| entry of \verb|clar:eke| in the \verb|thebibliography|
environment should read:
\begin{verbatim}
\begin{thebibliography}{}  % (do not forget {})
.
.
\bibitem[1982]{clar:eke}
Clarke, F., Ekeland, I.:
Nonlinear oscillations and boundary-value problems for
Hamiltonian systems.
Arch. Rat. Mech. Anal. {\bfseries 78} (1982) 315--333
.
.
\end{thebibliography}
\end{verbatim}
{\itshape Sample Output}
\bibauthoryear
%
\end{document}
