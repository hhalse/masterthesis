\section{Discussion}
In this paper we have introduced tree different models, model 1: Bonus, model 2: Bonus and discount, model 3: Symmetric connection game with discount.

Model 1 can be applied to certain real-world scenarios, such as software development firms/chains, or other networks where the final product is dependent on the collaboration of multiple participants.
This was done by including a bonus, which is first received when a node reaches the desired number of links (called max-degree). This made the separation process of insured and non-insured nodes more difficult for the insurer. Due to the possibility of achieving the bonus, a node will have more incentive to establish links, and is thus more accepting towards establishing links with risky nodes. The conditions for separating insured and non-insured nodes in this model are: $\beta+\gamma-r<I_{l}<\beta+\frac{\gamma}{m}$. For the separation of insured and non-insured nodes to be possible, the following has to hold: $1-\frac{1}{m}<\frac{r}{m}$. As we see, as $\gamma$ and/or $m$ increases, this gets more and more difficult to achieve. 

In Model 2 we tried to implement a common feature used by insurance companies, bulk discount, in order to see how this affected the network formation. The cost of insuring a link is now dependent on the node's degree. 
The discount resulted in even higher incentive for insured nodes to establish links with non-insured nodes. The reason is intuitive, since the cost of doing so decreases as the node degree increases. 
The condition to ensure separation becomes: $m(\beta+\gamma-r)<I_{l}<\beta+\frac{\gamma}{m}$, and as in the other models, this further complicates the separation process for the insurer. 

Since the incentive for establishing links has increased, the insurer has to set a higher price to compensate for this, therefore the potential price of anarchy is higher than in model 1 i.e. the more incentive for link establishment you have, the harder it gets to ensure separation of the nodes.  

In our last model we applied the discount to an already existing model, "the symmetric connection game". In this old game it has been shown that there are three different efficient and stable networks, clique, star and an empty network, that arise under certain cost conditions. If $I_{l}<\beta-\beta^{2}$, the efficient and stable network is a clique. If $\beta-\beta^{2}<I_{l}<\beta$ a star is both stable and efficient. If $I_{l}>\beta+\frac{N-2}{2}\beta^{2}$ an empty network is both stable and efficient. In general, a clique is the most efficient if the cost of establishing links is less than the benefit gained from indirect connections. A star is the most efficient if the cost is higher than the benefit from indirect connections, but less than the benefit of direct connections. 
Unfortunately, it is proved that as the number of nodes in the networks increases, the probability of the network ending up in star goes to zero. However, when we applied our insurance discount to this model, we found conjectures saying that, by setting the cost to the right level, one can with high probability ensure that either a clique, a star or a scale-free structure will evolve. This changes the connection game drastically, because now the insurer is able to force the network into three possible network formations, where the star has a fixation probability that exceeds the cliques. The insurer can use these findings to ensure that one of the beneficial structures, star or clique evolves. If the insurer is able to force a star to evolve, this can be used to drastically increase the overall security, and at the same time minimize the overall link cost. 

\section{Conclusion}
From our background study, it was revealed that the current market for cyber-insurance is far from healthy, and many have failed in attempts to establish a cyber-insurance market.
As described in the introduction, there are certain obstacles that are unique for cyber-insurance, and arguably these are the reasons why cyber-insurance has not emerged as expected. 
 However, we believe that there is a need for cyber-insurance, and that our new approach of analyzing the cyber-insurance market through graphs and network formation games could help establishing and improving the market.

We studied a variety of different network formation games, in order to find out if there were any superior network topologies that would fit as a cyber-insurance network, were ideally both the insurer and customers get a higher payoff from purchasing cyber-insurance. 
We found that star and clique networks had appropriate characteristics, not only do they have calculable fixation probability, but they could also generate better security and overall higher payoff for the nodes. With these networks in mind, we wanted to find a way of forcing networks to evolve into these structures.  We found that insurers could adjust the insurance premium in order to control the formation of networks. If the price is set to the right level, networks with calculable risk will evolve, and if the insurer is able to separate the nodes into two different networks, one consisting of trusted, insured nodes, the other of non-insured nodes, the trusted nodes can even further increase their payoff, compared to a non-trusted network. The insurer now possesses a tool for setting the insurance premium properly, possible resulting in better products for both the customer and the insurer.

\paragraph{Limitations and future work}
One limitation to our work, and a suggestion for future work, is to map our models and simulations to real-world networks in a more convincing way. Real-world networks are not random. Nodes may prefer to talk to nodes with high degree or low degree. In addition, the decision to use additive risk were taken due to the simplicity of the function and the fact that we do not know how a real-world risk distribution actually looks like.  By introducing a complex risk function, we would only have distorted the goal of our models. i.e. suggestions for improving our models is to introduce more realistic payoff functions.

Another interesting thing to research, is the game of choosing insurance or not. In future work this should be applied to our models, but this could also possibly be too complex, and only disrupt the models.