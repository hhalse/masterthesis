
\section{Related work 2}
Virus and worm propagation on the Internet can be modeled as epidemic spreads. 
When we look a 2-agent model we can observe correlation between one agents choice of investing in 
protection. If agent 1 has a connection to agent 2, the probability of agent 2 being contagion is 
strongly correlated to the choice of agent 1. In the case where agent 1 invests in protection,
 agent 2 will not be infected. However, if chooses not to invest in protection, the probability 
 of infection for agent 2 is p.
 After a number of equations the authors conclude that in presence of insurance, the optimal 
 strategy for all users is to invest in self-protecting services as long as this cost is low
  enough.

Further the authors looks at the situation where the cost of selv-protection is different for 
different agents (heterogeneous users) in a complete graph (n-n). The conclusion states that
 insurance increase the adoption for a fraction of the users, which creates the cascading effect
 that the rest of the users also gains benefit from investing in insurance. We end up in a state
  where everyone in the network are self-protected. 

In star shaped graphs (i.e. hubs), it is obvious that the network will decrease the probability
 contagion dramatically by investing in self-protection measures. The authors also assumes that it
 is likely that the other low connectivity nodes will follow the hub and adopt self-protection. 
