\chapter{Related work 2}
\label{chp:relatedwork2} 


Virus and worm propagation on the Internet can be modeled as epidemic spreads. 
When we look a 2-agent model we can observe correlation between one agents choice of investing in 
protection. If agent 1 has a connection to agent 2, the probability of agent 2 being contagion is 
strongly correlated to the choice of agent 1. In the case where agent 1 invests in protection,
 agent 2 will not be infected. However, if chooses not to invest in protection, the probability 
 of infection for agent 2 is p.
 After a number of equations the authors conclude that in presence of insurance, the optimal 
 strategy for all users is to invest in self-protecting services as long as this cost is low
  enough.

Further the authors looks at the situation where the cost of selv-protection is different for 
different agents (heterogeneous users) in a complete graph (n-n). The conclusion states that insurance increase the adoption for a fraction of the users, which creates the cascading effect that the rest of the users also gains benefit from investing in insurance. We end up in a state where everyone in the network are self-protected. 



\\
\\
Related work 2. TEST TEST TEST Here is an example of how to use acronyms such as \gls{ntnu}. The second time only \gls{ntnu} is shown and if there were several you would write \glspl{ntnu}. And here is an example\footnote{A footnote} of citation~\cite{Author:year:XYZ}. 

\Blindtext[3][1]

\begin{figure}
\centering
% dummy figure replacement 
\begin{tabular}{@{}c@{}}
\rule{.5\textwidth}{.5\textwidth} \\
\end{tabular}
\caption{\label{fig:example}A figure}
\end{figure}

\section{First section}\label{sec:first_section}

\subsection{First subsection with some \texorpdfstring{$\mathcal{M}ath$}{Math} symbol}\label{sec:first_ssection}

\blindtext
\begin{itemize}[topsep=-1em,parsep=0em,itemsep=0em] % see http://mirror.ctan.org/macros/latex/contrib/enumitem/enumitem.pdf for details about the parameters
 \item item1
 \item item2
 \item ...
\end{itemize}

\subsection{Mathematics}

\blindmathtrue
\blindtext

\begin{proposition}\label{def:a_proposition}
A proposition... (similar environments include: theorem, corrolary, conjecture, lemma)

\end{proposition}

\begin{proof}
\vspace*{-1em} % Adjust the space when parskip is set to 1em
And its proof.
\end{proof}

\begin{table}
\caption{\label{tab:example}A table}
\centering
\begin{tabular}[b]{| c | c | c | c | c |}
\hline
a & b & c & d & e \\ \hline
f & g & h & i & j \\ \hline
k & l & m & n & o \\ \hline
p & q & r & s & t \\ \hline
u & v & w & x & y \\ \hline
z & æ & ø & å &   \\ \hline
\end{tabular} 
\end{table}

\subsection{Source code example}

% \floatname{algorithm}{Source code} % if you want to rename 'Algorithm' to 'Source code'
\begin{algorithm}[h]
  \caption{The Hello World! program in Java.}
  \label{hello_world}
  % alternatively you may use algorithmic, or lstlisting from the listings package
  \begin{verbatim}
  
class HelloWorldApp {
  public static void main(String[] args) {
    //Display the string
    System.out.println("Hello World!");
  }
}
\end{verbatim}
\end{algorithm}

You can refer to figures using the predefined command like \fref{fig:example}, to pages like \pref{fig:example}, to tables like \tref{tab:example}, to chapters like \Cref{chp:example} and to sections like \Sref{sec:first_section} and you may define similar commands to refer to proposition, algorithms etc.
