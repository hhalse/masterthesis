\chapter{Introduction to Cyber Insurance}
\label{chp:introductionToCyberInsurance} 

\section{Motivation}
Security breaches are increasingly prevalent in the Internet age, causing huge financial losses for companies and their users. Cyber-insurance is a powerful economic concept that can help companies in the fight against such malicious behavior. Earlier research suggests that cyber-insurance has failed to reach its promising potential, although the concept of cyber-insurance has been around since the 1980s. 

The researchers claims that a functional model for cyber-insurance has to handle the problems regarding interdependent security, correlated risk and asymmetrical information. Many researchers have proposed models to solve these problems, but the market still strives to succeed. Another problem with cyber-insurance is to determine the overall risk in the network. If cyber-insurance networks were describable and analyzable by graphs, the calculation of overall risk would be much easier. We ask whether there exist network structures that are superior as cyber-insurance networks compared to other networks. If so, is it possible for insurers to determine the structure of these networks, or even better to create new or transform existing networks into a certain structure?

\section{Problem definition}
In this project, the goal is to analyze the current state of the cyber-insurance market. Study and characterize network structures suited to be used as a cyber-insurance network. A desired structure will possess some characteristics that would be beneficial for a cyber-insurance network. Additionally, we will build a model, which can relate to different real world scenarios, using network formation to force the creation of these structures. 

\section{Reader's guide}
Chapter 1 introduces the concept of cyber-insurance and presents a survey of the current cyber-insurance market. 
\\Chapter 2 discusses and summarizes related work. 
\\Chapter 3 shows how graphs can describe real-world networks, such as airline routes and stock markets. The chapter also presents and discusses the properties of graphs well suited for cyber-insurance networks. These graphs are the foundation for the models created in chapter 5. 
\\Chapter 4 presents the basic concepts of graphs and game theory. It also presents the simulation tool, Netlogo, used to simulate the different models of Chapter 5. 
\\Chapter 5 presents different endogenous\footnote{Exogenous: The network formation is given. Endogenous: The structure originates from within the network, i.e. the opposite of exogenous} network formation models, where the nodes are agents, with or without insurance, seeking to establish links with eachother.  
\\Chapter 6 discusses and summarizes the findings in the thesis, with focus on chapter 5. 


\section{Introduction}
Security breaches are increasingly prevalent in the Internet age causing huge financial losses
for companies and their users. When facing security breaches and risk, there are typically four ways to act \cite{bolot2008new}:
\begin{enumerate}
\item Avoid the risk
\item Retain the risk
\item Self protect and mitigate the risk
\item Transfer the risk
\end{enumerate}
The ICT industry have so far tried to prevent risks with a mixture of options two and three. This has lead to many different techniques and software trying to detect threats and anomalies, to protect the users and infrastructure. Firewalls, intrusion- detection and prevention systems, are some of the solutions. These will reduce the risk, but do not eliminate the risk completely. Although they are all good and needed actions, it is impossible to achieve perfect cyber-security, due to many reasons: Threats are continuously evolving, there will always be accidents and security flaws, attackers have different intentions, network externalities and free-riding in security networks, the lemons-market in security products, misaligned incentives between users and product vendors, and many more. 
This is why we need cyber-insurance, as an fourth option, to handle the residual risk \cite{lelarge2009economic,paldifferentiating}.

\subparagraph{Market potential for cyber-insurance.}
Just like regular insurance, cyber-insurance is an insurance product used to transfer financial risk, associated with computers and network-related incidents, over to a third party. This third party willingly accepts the risk, in exchange for a fee, called insurance premium.
The insurance is focused on computer-related issues, and could provide coverage against property loss and theft, data damage, cyber-extortion, loss of income due to denial of service attacks or computer failures, reduced reputation and customers churning due to leaked user information and so on. Traditional insurance policies rarely cover incidents like these, therefore a specialized insurance product is needed to handle these residual risks. I.e. there is a huge potential market for cyber-insurance \cite{washingtonpaper}.  


As mentioned, the concept of cyber-insurance has been around since the 1980s, and has failed to reach its promising potential. There might be several reasons for this slow development, however, it is believed that the main reason so far, is that no model deals with all the unique problems of cyber-insurance at the same time. In addition to the known difficulties of insurance, such as calculating risk, cyber-insurance differs from traditional insurance because it has to deal with the problem of asymmetric information, correlated risk and interdependent security \cite{networkgames}.  

\subparagraph{Traditional Insurance.} 
The basic structure of cyber-insurance relates to traditional insurance, where an insurance contract (policy) binds the insurance company to pay a specified amount to the insurance holder whenever an incident occurs. In return, the insurance holder has to pay a fixed monthly or annual fee (premium) to the insurance company. The contract includes a risk assessment of the company's vulnerability and clearly specifies the entitled amount of coverage for each of the different risks. These assessments are used to calculate the companies' premium \cite{robinson2012incentives}. Generally, this means that the security is negatively correlated with the premium costs. In cyber-insurance this means that the better the security, the lower the price on the insurance premium.
   
Generally, to ensure that their business is economically viable, the insurance company will require that insurable risks possess seven distinct characteristics \cite{mehr1980principles}: 
   \begin{enumerate}
   \item Large number of similar exposure units: Insurance is based on the principle of pooling resources, where insurance policies are offered to individual members of a large class. Meaning the more customers, the closer the predicted losses will get to the actual losses.
   \item Definite loss: A loss should take place at a known time, in a known place and from a known
    cause. Incidents such as a fire or car crash, are examples where these terms are easy to verify.
   \item Accidental loss: The event that triggers a claim should not be 
   something the insurer has discretion or control over.
   \item Large loss: The size of the loss must be meaningful from the perspective of the insured.
    Insurance premiums need to cover both the expected cost of the loss, and in addition, 
    cover all the expenses regarding issuing and administrating policies, adjusting losses and
     supplying the capital needed to be able to pay claims.
   \item Affordable premium: The premium must be proportional to the security offered, otherwise no one will offer/buy the insurance. In the situation where the likelihood of the insured event is high, and the cost is large, it is unlikely that the insurance company will offer the insurance, or if so the premium would be very high.
   \item Calculable loss: Both the probability and the cost of an insurable event,
    has to at least be possible to estimate. 
   \item Limited risk of catastrophically large losses: If losses happen all at the same time, the likelihood of
    the insurance company getting bankrupt is high. Therefore, losses are ideally independent and non-catastrophic. 
   \end{enumerate}
   
This model will also apply to the risks covered by cyber-insurance. Unfortunately there are additional obstacles regarding cyber-insurance. The three major problems with cyber-insurance are related to; information asymmetry, interdependent security and correlated risk. 
 
\subparagraph{Information asymmetry.}
Information asymmetry arises when one side of a transaction or decision has more or better information than the other party. There are two different cases of information asymmetry. The first one is called adverse selection, where one party simply has less information regarding the performance of the transaction. A good example is when buying health insurance, if a person with bad health purchases insurance, and the information about her health is not available to the insurer, we have a classical adverse selection scenario, where the insurer probably charges too little. We can observe a similar situation for the cyber-industry, where an insurer has no way of confirming whether your network is "healthy", i.e. not contaminated or infected. 
The other information asymmetry scenario is called moral hazard. It occurs after the signing of the contract, where one party deliberately takes some action that makes the possibility of loss higher, e.g. choosing not to lock your door, since you have insurance. Or in the computer setting, deliberately visiting hostile web-pages, or not using anti-virus software, firewalls or other self-protection software, although you are required to do so. \cite{solutiontoinfoasym}.
    
    
The task of measuring the level of security is very hard, and in order to lower the premiums people will have an incentive for hiding information about their security level, hence the problem with asymmetry is highly relevant. Another problem occurs on the customer side of the market. For a customer wanting to improve his/her defense mechanisms, the software security market often becomes a lemon's market\footnote{Lemon market, the problem of quality uncertainty, was first introduced in a paper \cite{lemonpaper} by the economist George Akerlof in 1970,and used the market for used cars as an example \cite{lemon}. The conclusion of the paper is that since the buyers lack information to distinguish a bad car(lemon) from a good one(cherry), the buyer will not pay the price the seller wants for a cherry, and the seller will not sell a cherry for the price of a lemon, and thus the lemons drive the cherries out of the market.}. 
It is difficult for the buyer to distinguish the performance of different software products, and thus the reasonable thing to do, is to buy the cheapest. Therefore, the good security products must cost the same as the bad. If the cost of producing good security software is too high, the problem can even result in abandoning the production of good software, because it would not be profitable.

      
\subparagraph{Correlated risk.}

Another big concern regarding cyber-insurance, is the correlated risk. Among other things, the problem occurs due to the need for standards. Standardization is an important part of the business of computers and computer networks. Generally it enables computers to communicate, install and use different software. A good example is operative systems for personal computers, today we only have a small set of operative systems available, and these systems are standardized, so they can use the same communication channels. The standards generate a lot of the value in the ICT industry, but they also make many threats possible. All systems that use the same standards, create a large number of similar exposure units, i.e. they share common vulnerabilities, which could be exploited at the same time. As we see, this violates the insurance characteristic of limited risk of catastrophically large losses.  
Thus create a significant difficulty for the cyber-insurance industry, because when a security breach occurs there is a high probability that it will occur to a large number of people, i.e. catastrophic and extreme events occur with a higher probability than in the regular insurance business. To compensate, the logical thing to do would be to raise the premium cost, this could however violate the characteristics of affordable premiums and large losses. 
If the security breach is large, it could even potentially cause so much damage, that the insurers will not be able to pay all the customers who suffered, and they could go bankrupt.\cite{bohme2010modeling}


\subparagraph{Interdependent security.}
Another problem in the ICT industry is interdependent security, meaning that you are not only dependent on your own investment in security, but also on everyone else's. 
Investment in security generates positive externalities, and as public goods, this encourages free riding. Why should I pay for security when I can just free ride on security invested by others? The problem is that the reward for a user investing in self-protection depends on the security in the rest of the network. i.e. The expected loss due to a security breach at one agent in the network, is not only dependent on this agent's level of investment in security, but also on the security investment done by adjacent agents, and their adjacent agents and so forth. A good example of this is the amount of spam sent every day, which depends on the number of compromised computers. Meaning if you have invested in security software of some kind, you still receive lots of spam because many other people have not invested \cite{towardsInsurable}.
 
\subparagraph{Calculating losses} As mentioned, a problem in several areas of insurance is the calculation of risk. In cyber-insurance, the unique obstacles contribute to making this particularly difficult. When facing a security breach there are two potential loss classes: \cite{bandyopadhyay2009managers,mehr1980principles} 
\begin{itemize}
\item Primary losses or first-degree losses: direct loss of information or data and operating loss. 
These arise from disuse, abuse or misuse of information.
 The cost of these arises from recovering, loss of revenue, 
 PR and information sharing costs, hiring of IT specialists etc.
 \item Secondary losses are indirectly triggered. These are the loss of reputation, goodwill, 
consumer confidence, competitive strength, credit rating and customer churning. 
\end{itemize}
The cost of the loss from both these classes can be difficult to determine, although the second one is probably the most difficult, since it is challenging to put a value on e.g. how many potential customers did they lose due to the reputation loss, how many customers churned, and what was their value etc.
It could also be difficult to determine when the loss happened, where and what caused it.
Another problem when calculating losses and determining insurance premiums, is the unavailablity of large amounts of historic data on cyber-crimes, which are needed in many insurance models to calculate the risk, losses and premiums. This problem arises i.a. because many firms do not reveal details about their experienced security breaches. \cite{herath2007cyber}

\subparagraph{Cyber-insurance instead of security.}
Another problem with cyber-insurance is actors seeing it as a solution to the problem of being secure. Instead of investing in security, they now have a way of buying their way out. 
However, as the paper \cite{bolot2008cyber} shows, this problem might be solved with the right pricing options, meaning that the insurance companies can create pricing models which make it economically beneficial to invest in security and cyber-insurance. Cyber-insurance can be used as an incentive for buying security. Such models will also make sense for the insurance company, since better security systems yields lower probability for incidents.

As we see there are many problems regarding cyber-insurance, but the insurance industry has been dealing with many difficult problems in other areas of life. Cyber-insurance faces many challenges, but we can't say that internet risks and damages can not be insured. We just need to find a way of helping the insurers to create a better product, i.e. the challenge is to find a way for the insurers to handle these special characteristics, in order to create a healthy cyber-insurance market. \cite{lelarge2009economic}
To help establishing a healthy cyber-insurance market, one needs to know its current status.

\section{The cyber-insurance market}
The market for cyber-insurance emerged in the late 80's, when security software companies began collaborating with insurance companies to offer insurance policies together with their security products. From a marketing perspective, adding insurance helped highlighting the supposedly high quality of the security software. Nevertheless, this new product was a comprehensive solution, which dealt with both risk reduction and residual risk \cite{bolot2008new}. Continuing into the beginning of the new millennium, several companies started offering standalone cyber-insurance, which sat the frame for the current insurance product. In Norway, startup-companies, such as Safensure AS were established with the goal to deliver cyber-insurance to the Norwegian and European market \cite{digi}. In addition, already well-established insurance companies, such as Gjensidige Nor, started offering insurance products intended for the web-site market. These insurances were created to insure lost income due to malicious hackers, denial of service and other well know cyber-attacks at that time. In 2001 Gjensidige Nor, in cooperation with the German company Tela Versicherung, offered businesses insurance against financial losses due to hacker attacks and sabotage in a range up to 5 million NOK, given that the companies could provide proof that specified security measures were taken \cite{dagensithackerforsikring}. 
 

Despite the fact that cyber-insurance has been around for a couple of decades, the market still struggles to gain a foothold. Safensure AS does not longer exist and Gjenside Nor does not advertise a cyber-insurance product anymore. There seems to be many challenges for both buyers and sellers. Buyers face tremendous confusion about cyber risks and their potential impacts on business. 
The paper \cite{Cyberworkshop} points out that people do not know or understand what kinds of risks the cyber-space involves, and how fatal the losses can be. Even when companies have decided to purchase a cyber-insurance, they are confused with what kind of insurance they should purchase, it is difficult to see what it covers, what is a reasonable price etc. Thus, the market for cyber-insurance tends to become a lemon.s market, where the buyer lacks knowledge, and struggles to see the differences between the different insurance contracts.


\subparagraph{The UK and US markets.}
We wanted to reveal the current status of the cyber-insurance market. We limited our survey to the UK and US markets, in addition to the Norwegian market. 
The first impression reveals that there are several different results and opinions regarding the health of the global cyber-insurance market. 
The paper \cite{ccost} studied a sample of 50 organizations in various industry sectors, located in the United States. They showed that on average every company suffered more than one successful attack every week, and argued that successful cyber-attacks could have serious financial consequences. They found that the median cost of cyber-crime in the U.S is \$5.9 million per year, ranging from \$1.5 million to \$36.5 million per company, which is a 56 $\%$ increase from 2010. 
 
 Another paper \cite{evolvingcyber} collected statistics about cyber-attacks in the UK, and claims that the costs are expected to be \pounds 27 billion a year, which makes cyber-crime one of UK's biggest emerging threats. In addition, the paper pointed out that the victims are not only large companies like Google and PlayStation, but also small businesses. Despite these numbers, only 35 $\%$ of the companies in the survey had purchased cyber-insurance. This is surprisingly low, since they found no shortage of providers. It was revealed that in the UK there are nine insurers who specializes in cyber-insurance, and in the US around 30-40 insurers.  
 
 
A UK firm, called CFC underwriting, who offers insurance to small and medium sized businesses, published an article \cite{CFCunder} claiming promising numbers for the US cyber-insurance market.
On US soil, 20-50$\%$ of the businesses purchased either standalone cyber-insurance or benefits from coverage provided in their existing insurance. However, despite recent years' focus on the increasing cyber-crime activity and the catastrophic consequences of having weak security, only 1$\%$ of European businesses are enrolled in an insurance program covering cyber-threats.
A more optimistic survey, \cite{compworld}, pointed out that more and more insurance companies offer cyber-insurance. Yet, of the 13000 companies, only 46 $\%$ reported that they were insured against the economic consequences of cyber-attacks.
 
 The media coverage on corporate threats such as Stuxnet and the attacks on Playstation, which lead to a compromise of 77 million user accounts including credit card numbers \cite{playstation}, shows that the cyber-threats are growing, and one would assume that we are in need of cyber-insurance. However, even though the number varies, the surveys show that a large share of companies have chosen not to protect themselves against the residual risk of cyber-attacks, by buying cyber-insurance.

\subparagraph{The Norwegian market.}
Our survey of the Norwegian insurance market revealed that specialized cyber-insurance companies, such as Safensure AS, do not exist anymore. Only one out of the five biggest insurance companies\footnote{Gjensidige, If Skadeforsikring, DNB, TRYG, Storebrand} offers something similar to a cyber-insurance. Gjensidige Nor offers what they call operation-loss-insurance, which covers expenses due to reconstruction of files and reinstalling software and denial of service attacks. In addition, it is also possible to insure against hacking and sabotage \citep{gjensidige}. From email correspondence with Gjensidige Nor it was clear that they needed lots of information regarding the company to be able to calculate the insurance premium. They required extensive information about the economic health of the company, and a model of what kind of software and hardware where used with estimated values on each component. Unfortunately, we were not able to obtain the cost of such an insurance.
  
\subparagraph{Market outlook.}
The survey from \cite{CFCunder} claimed that the US cyber-insurance market was much more mature than the European market. A possible reason is the breach notification laws. In the US, 46 states have mandatory breach notification laws, combined with significant penalties for companies failing to protect sensitive data. This means that the US government is creating incentives for firms to buy cyber-insurance.
 In Europe, only Germany and Austria have similar laws, forcing companies to notify affected customers of data leakage. A recent proposal of the EU wants to introduce the notification law in Europe, and also include penalties for serious data breaches, which could be set as high as 2 $\%$ of a company's global revenue \cite{CFCunder}. It is proposed that the law should take effect in 2014, although this is highly unlikely, considering the complexity of the effects of this law. Undoubtedly such a law would be a healthy injection to the cyber-insurance market. However, a market based on fear of the consequences of not being insured is not desirable. The ultimate goal for cyber-insurance is to correlate the purchase of cyber-insurance with companies growing desire to invest in more security, and hence lower the risk of being a victim of cyber-crimes. 
The article claims that the way to meet this goal, is to focus on the serious brand damage a company will experience and not just on the financial loss. 

In summary, the cyber-insurance market seems to have a huge potential, but needs some new thinking to fully take advantage of it. We will take another approach and focus on finding network structures that will help the insurers offer fair contracts, which is beneficial for both the customers and the suppliers. Hopefully, this can help in the process of establishing a healthy cyber-insurance market.

   
   
   
   
   
   
   
  

