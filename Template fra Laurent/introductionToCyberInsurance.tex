\chapter{Introduction to Cyber Insurance}
\label{chp:introductionToCyberInsurance} 

Cyber-insurance is an insurance product used to transfer financial risk
associated with computer and network related incidents over to a third party.
 Coverages provided by cyber-insurance policies may include property loss and
<<<<<<< HEAD
  theft, data damage, cyber-extortion, loss of income due to denial of

   service attacks or computer failures. \cite{washingtonPaper}

  
    Cyber-insurance works the same way as traditional insurance, where the insurance contract (policy) binds the insurance company to pay a specified amount to the insurance holder when certain incidents occurs. In return, the insurance holder has to pay a fixed sum (premium) to the insurance company. \cite{enisarapport}

  Traditional coverage policies rarely cover these incidents, therefore 
=======
  theft, data damage, cyber-extortion, loss of income due to denial of service attacks or computer failures. \cite{washingtonpaper}

Traditional coverage policies rarely cover these incidents, therefore 
>>>>>>> master
   cyber-insurance is seen as a huge potential market. However, the concept 
   of cyber-insurance has been around since the 1980s, but so far it has failed to reach its promising potential. 
  
 
Cyber-insurance works the same way as traditional insurance, where the insurance contract (policy) binds the insurance company to pay a specified amount to the insurance holder when certain incidents occurs. In return, the insurance holder has to pay a fixed sum (premium) to the insurance company. \cite{robinson2012incentives}
    As with other insurances, the cyber-insurance contract is signed between the insurance company and the insurer. The contract clearly specifies the type of coverage of the different risks, a risk assessment of the companies vulnerability and also an evaluation of how the companies security systems. These assessments are used to calculate the companies premium. \cite{robinson2012incentives} Generally, this will mean that the more secure a company is, the lower the premium costs.  

\section{The basics of insurability}

 Generally, insurable risks possesses seven common characteristics: \cite{mehr1980principles}
   \begin{enumerate}
   \item Large number of similar exposure units: Insurance companies is based on the principle of pooling resources, where insurance policies are offered to individual members of a large class, meaning the more insurers the predicted losses is closer to the actual losses. 
   \item Definite loss: A loss should take place at a known time, in a known place and from a known cause. Incidents such as a fire or car crash, are examples where these terms are easy to verify.
   \item Accidental loss: The event that triggers a claim should not be something the insurer has discretion or control over.
   \item Large loss: The size of the loss must be meaningful from the perspective of the insured. Insurance premiums need to cover both the expected cost of the loss, in addition, cover all the expenses regarding issuing and administrating policies, adjusting losses and supplying the capital needed to be able to pay claims.
   \item Affordable premium: The premium must be proportional to the security offered, otherwise no one will offer/buy the insurance. In the situation where the likelihood of the insured event is high, and the cost is large, it is unlikely that the insurance company will offer the insurance, or at least the premium would be too high for anyone to consider buying it. 
   \item Calculable loss: Both the probability and the cost of an insurable event, has to atleast be possible to estimate. 
   \item Limited risk of catastrophically large losses: If losses happen all at once the likelihood of the insurance company getting bankrupt is high. Therefore, losses are ideally independent and non-catastrophic. 
   
<<<<<<< HEAD
=======
   \end{enumerate}
    
    
 Although there are some problem areas, such as defining loss, where it often is difficult to demonstrate the location and cause of data breaches. Cyber-insurance appears to fit in to the general model of insurance. Standardization is important for network and computers, leading to many users using the same operation systems, and other software and hardware products, hence there is a large number of similar exposure units. Power outage, DoS-attacks etc. are usually a result of accidental loss.
 \\ Further, the premiums can be priced at a affordable level, in cyber-insurance the premium level will be highly dependent upon the companies security systems and policy. This also relates to the calculable loss, where better security systems yields lower probability for incidents. \cite{robinson2012incentives}
 Another problem cyber-insurance has to face is the fact that losses might be correlated, resulting in insurance companies have to pay large numbers of claims at once. Examples are policies including insurance against lost income due to denial of service of websites. If the backbone network is down for numerous reasons, every operator connected will loose the Internet connection, hence be entitled to receive compensation for the lost income.   
>>>>>>> master
 
   
 



\\
\\
... Random notater: One problem with cyber insurance is actors seeing it as a solution to the problem of being secure. Instead of investing in security, they now have a way of buying their way out. However, this problem might solve it self due to the fact that insurance companies only will indemnify the losses where victim can prove that a certain event has occurred. When it comes to cyber insurance, one often need computer forensics to generate the evidence needed. 






%\Blindtext[3][1]

%\begin{figure}
%\centering
%% dummy figure replacement 
%\begin{tabular}{@{}c@{}}
%\rule{.5\textwidth}{.5\textwidth} \\
%\end{tabular}
%\caption{\label{fig:example}A figure}
%\end{figure}

%\section{First section}\label{sec:first_section}

%\subsection{First subsection with some \texorpdfstring{$\mathcal{M}ath$}{Math} symbol}\label{sec:first_ssection}

%\blindtext
%\begin{itemize}[topsep=-1em,parsep=0em,itemsep=0em] % see http://mirror.ctan.org/macros/latex/contrib/enumitem/enumitem.pdf for details about the parameters
% \item item1
% \item item2
% \item ...
%\end{itemize}

%\subsection{Mathematics}

%\blindmathtrue
%\blindtext

%\begin{proposition}\label{def:a_proposition}
%A proposition... (similar environments include: theorem, corrolary, conjecture, lemma)

%\end{proposition}

%\begin{proof}
%\vspace*{-1em} % Adjust the space when parskip is set to 1em
%And its proof.
%\end{proof}

%\begin{table}
%\caption{\label{tab:example}A table}
%\centering
%\begin{tabular}[b]{| c | c | c | c | c |}
%\hline
%a & b & c & d & e \\ \hline
%f & g & h & i & j \\ \hline
%k & l & m & n & o \\ \hline
%p & q & r & s & t \\ \hline
%u & v & w & x & y \\ \hline
%z & æ & ø & å &   \\ \hline
%\end{tabular} 
%\end{table}

%\subsection{Source code example}

% \floatname{algorithm}{Source code} % if you want to rename 'Algorithm' to 'Source code'
%\begin{algorithm}[h]
%  \caption{The Hello World! program in Java.}
%  \label{hello_world}
  % alternatively you may use algorithmic, or lstlisting from the listings package
%  \begin{verbatim}
  
%class HelloWorldApp {
%  public static void main(String[] args) {
%    //Display the string
%    System.out.println("Hello World!");
%  }
%}
%\end{verbatim}
%\end{algorithm}

%You can refer to figures using the predefined command like \fref{fig:example}, to pages like \pref{fig:example}, to tables like \tref{tab:example}, to chapters like %\Cref{chp:example} and to sections like \Sref{sec:first_section} and you may define similar commands to refer to proposition, algorithms etc.
