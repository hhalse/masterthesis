\chapter{Introduction to Cyber Insurance}
\label{chp:introductionToCyberInsurance} 

Cyber-insurance is an insurance product used to transfer financial risk
associated with computer and network related incidents over to a third party.
 Coverages provided by cyber-insurance policies may include property loss and
  theft, data damage, cyber-extortion, loss of income due to denial of
   service attacks or computer failures. 
  
  Traditional coverage policies rarely cover these incidents, therefore 
   cyber-insurance is seen as a huge potential market. However, the concept 
   of cyber-insurance has been around since the 1980s, but so far it has failed to reach its promising potential. 
   
   Cyber-insurance works the same way as traditional insurance, where the insurance contract (policy) binds the insurance company to pay a specified amount to the insurance holder when specified in incidents occurs. In return, the insurance holder has to pay a fixed sum (premium) to the insurance company. \citep{enisarapport}
   
 



\\
\\
One problem with cyber insurance is actors see it as a solution to the problem of being secure. Instead of investing in security, they now have a way of buying 
their way out. However, this problem might solve it self due to the fact that insurance companies only will indemnify the losses where victim can prove that a certain event has occurred. When it comes to cyber insurance, one often need computer forensics to generate the evidence needed. 






%\Blindtext[3][1]

%\begin{figure}
%\centering
%% dummy figure replacement 
%\begin{tabular}{@{}c@{}}
%\rule{.5\textwidth}{.5\textwidth} \\
%\end{tabular}
%\caption{\label{fig:example}A figure}
%\end{figure}

%\section{First section}\label{sec:first_section}

%\subsection{First subsection with some \texorpdfstring{$\mathcal{M}ath$}{Math} symbol}\label{sec:first_ssection}

%\blindtext
%\begin{itemize}[topsep=-1em,parsep=0em,itemsep=0em] % see http://mirror.ctan.org/macros/latex/contrib/enumitem/enumitem.pdf for details about the parameters
% \item item1
% \item item2
% \item ...
%\end{itemize}

%\subsection{Mathematics}

%\blindmathtrue
%\blindtext

%\begin{proposition}\label{def:a_proposition}
%A proposition... (similar environments include: theorem, corrolary, conjecture, lemma)

%\end{proposition}

%\begin{proof}
%\vspace*{-1em} % Adjust the space when parskip is set to 1em
%And its proof.
%\end{proof}

%\begin{table}
%\caption{\label{tab:example}A table}
%\centering
%\begin{tabular}[b]{| c | c | c | c | c |}
%\hline
%a & b & c & d & e \\ \hline
%f & g & h & i & j \\ \hline
%k & l & m & n & o \\ \hline
%p & q & r & s & t \\ \hline
%u & v & w & x & y \\ \hline
%z & æ & ø & å &   \\ \hline
%\end{tabular} 
%\end{table}

%\subsection{Source code example}

% \floatname{algorithm}{Source code} % if you want to rename 'Algorithm' to 'Source code'
%\begin{algorithm}[h]
%  \caption{The Hello World! program in Java.}
%  \label{hello_world}
  % alternatively you may use algorithmic, or lstlisting from the listings package
%  \begin{verbatim}
  
%class HelloWorldApp {
%  public static void main(String[] args) {
%    //Display the string
%    System.out.println("Hello World!");
%  }
%}
%\end{verbatim}
%\end{algorithm}

%You can refer to figures using the predefined command like \fref{fig:example}, to pages like \pref{fig:example}, to tables like \tref{tab:example}, to chapters like %\Cref{chp:example} and to sections like \Sref{sec:first_section} and you may define similar commands to refer to proposition, algorithms etc.
