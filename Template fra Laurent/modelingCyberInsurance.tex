\chapter{Modeling Cyber-Insurance }
\label{chp:modelingCyberInsurance} 


\section{Network Formation}


In many scenarios agents seeks to create networks in order to directly benefit from each other. The established links might represent companies out sourcing part of their manufacturing, or cooperative agreements in the development of new software products. In addition to increase the trade-off, each of the established links represents risk of being a victim of cascading failures. The intuitive example is the spread of epidemic diseases, also  (node failures of a power grid and) financial contagion such as the one back in 2008 was a result of cascading failures. Strategic network formation using cyber-insurance can be used to prevent such situation in addition to increase the overall payoff of participants in a clustered network.


When deciding whether to establish connection to a neighbour agent, the payoff has to be a balance between the expected earnings and the risk of the other party failing to complete the transaction. This is the reason why we seek to only download content from trusted peers and outlaw MC-gangs are consistently skeptical to enter into new agreements despite promising increased earnings, since the risk of undercover police are too high. 


In the paper \cite{contagion}, they come up with some interesting results regarding network formation games. 
They set up a game where the nodes benefit from direct links, but these links also expose them for risk. 
Each node gains a payoff of  $a$ per link it establishes, but it can establish a maximum of $\delta$ links.
A failure occur at a node with probability $q$, and propagates on a link with probability $p$. If a nodes fail, it will receive a negative payoff of $b$, no matter how many links it has established.

The results from their model shows a situation where clustered graphs achieve a higher payoff when connected to trusted agents, compared to when connecting with random nodes. Unlike in anonymous graphs, where nodes connect to each other at random, nodes in these graphs share some information with their neighbours, which is used when deciding whether to form a link or not. 
To further explain these results, they show that there exists a critical point, called phase transition, which occurs when nodes have a node degree of $1/p$. 
At this point a node gets a payoff of $a/p$, to further increase the payoff the node needs to go into a region with significantly higher failure probability. 
Because once each node establish more than $1/p$ links, the edges which propagates risk, will with high probability form a large cluster. Which results in a rise in probability of node failure, and reduces the overall wellfare.
From this the paper say that when the minimum welfare exceeds 
$(1+f(\delta)*a/p)$
we have reached super critical payoff. Otherwise it is called sub-critical payoff. 
Further they show that the only possible way of ending up with supercritical payoff, is by forming clustered networks consisting of cliques with slightly more than $1/p$ nodes. 
If the nodes form an anonymous market, random linking, they can only get sub-critical payoff. 
In other words, if the nodes can choose who they connect with, and by doing so, creating trusted clustered markets, they can achieve a higher payoff, by exceeding the critical node degree point. But in random graphs, this is not possible. 


Inspired by this model, we created a model which shields light on how cyber-insurance can be used in network formation to prevent cascading failures and increase an agents payoff.  



\section{Very Simple Model \label{section:verysimplemodel}}

As a starting point the model is highly simplified in order to show the concept of how cyber-insurance can be used to create an insurable topology. Through out this chapter new features will be added to the model to make it more realistic and applicable. To begin, the model is formulated as follows.
A set of $n$ nodes are randomly chosen to be insured or not, as depicted in figure \ref{fig:firstmod1}. They all get their own fixed income, and by connecting to other nodes they can increase their payoff due to network externalities. Non-insured nodes will have a risk of failure i.e. an expected cost of failure. Therefore if an insured node chooses to connect to a non-insured node, he will also suffer from this expected cost of failure. 

In this simple model insured nodes are only willing to connect to other insured nodes, and non-insured can only connect to other non-insured nodes. Every node tries to connect with every other node, and since the decision is bidirectional, i.e. both nodes must agree on establishing a connection, to successfully do so. The resulting graphs will always be two fully connected cliques, one consisting of insured agents and the other of non-insured agents. 

This dichotomy represents a trusted environment for the insured nodes, because they are able to trust each other, because they know that no node in the clique can be connected to a non-insured node. In this way everyone is protected from the possible risks, such as financial catastrophes, viruses, diseases etc. 
The nodes in the trusted clique will gain a positive network externality from each of its connection, without having to worry about contagious risks from the other nodes in the clique.
A node in the non-insured clique will also receive the aggregated benefits from his connections in the clique, however each of the connection has a probability of failure. Hence this environment is not trusted, and connection always involves some risk. 

\begin{figure}[h]
\centering
\begin{subfigure}{.5\textwidth}
  \centering
  \includegraphics[width=0.4\linewidth]{../Figures/firstModelWithNoParameters1.png}
  \caption{\label{fig:firstmod1} 15 Agents randomly choosen to be either insured (green) or non-insured (red).}
\end{subfigure}
\quad
\begin{subfigure}{.46\textwidth}
  \centering
  \includegraphics[width=0.8\linewidth]{../Figures/firstModelWithNoParameters2.png}
  \caption{\label{fig:firstmod2} Two clustered networks. One consisting of insured agents the other consists of non-insured.}
\end{subfigure}
\caption{\label{fig:fincont} shows how agents connects to eachother according to model described in section \ref{section:verysimplemodel}.}
\end{figure}

There are many examples of nodes needing to establish connections, one example is a firm who needs to out-source certaint tasks to remain competitive. This outsourcing involves some risks, such as, will the company deliver at the reported time, to the reported costs, what happens if they fail to deliver, if they go bankrupt etc. If the firms that are going to establish links(contracts), know that the other firms are insured, it will be easier and more secure to establish links. In this way trusted cliques can evolve. The firms benefit from connecting to other insured firms, and the insurance company can offer fair prices to the insured companies, because they know that they are in a trusted clique.
Hence this model, although very simple, shows an insurable topology where insured agents benefit from being insured. 

The model is very simplified and suffer from many limitations, among others it is too simple to reflect the dynamics of a real world scenario, where each node will have different variables with different values. In addition, each node have a complete network information i.e. the problem with information asymmetry is not taken into account. But the model is a good starting point. 

\subsection{Model with parameters}
To make the simple model more realistic, we have to create a game with parameters. The characteristics of the game is as follows: an agent can be either insured or not insured, the insured ones have to pay an insurance cost $I_{0}$. Every agent starts with a fixed income, $\alpha$, to further increase their income they have to establish links to other agents. For each link they establish they receive a payoff of $\beta$. This represents the positive network externalities of the game. A insured node has to pay a cost of $I_{l}$ for every link he establishes. The game is bidirectional, i.e. both nodes need to agree of the establishment of a link, and if both are insured, they both have to pay the cost $I_{l}$. 
Every agent who is not insured will have a risk of failure(infected, bankruptcy, or some other type of failure), this is captured with the expected risk cost $r$. Every node, insured or not who connects to a non-insured node will also suffer from this cost. The link insurance cost and the risk are negative network externalities in the game. See Table \ref{tbl:simplegamepara} for an overview of the parameters. 
\begin{table}[h]
\centering
\begin{tabular}{lc}
 \hline
  $\alpha$ - agents fixed income\\
  $\beta$ - income from establishing a direct link \\
  $I_{o}$ - cost of having insurance. \\
  $I_{l}$ - increased insurance cost per link the node establishes\\
  $r$ - expected risk cost\\
  \hline
\end{tabular}
\caption{Table showing the parameters to be used in the first model \label{tbl:simplegamepara}}
\end{table}
\subparagraph{Two agent game}
To begin analysing the game, lets consider a two-person game. In this game the strategy space of both players consist of four different strategies. They can choose to purchase insurance or not, and to establish a link or not. I.e. the different strategies are: Be insured and establish link noted as: $IL$, 
be insured and do not establish link: $I\overline{L}$. Not insured and establish link: $\overline{I}L$, and not insured and do not establish link: $\overline{IL}$. It should be noted that since the decision to establish a connection is bidirectional, both have to choose a strategy where they want to establish a link, for the link to be establishment to be successful.
Figure \ref{fig:FirstGameTheoryModel} shows the different outcomes of this game.

\begin{sidewaysfigure} % <-- HERE
\centering
\includegraphics[width=1.0\textwidth]{../Figures/FirstGameWithParameters.png}
\caption{\label{fig:FirstGameTheoryModel} Normal form game, showing the different strategies and the payoffs  for the different outcomes. The payoff for agent A is written first, then the payoff for agent B is on the line beneath.
 An agent has a strategy space of size 4. }

\end{sidewaysfigure}


The simple game described earlier where the insured nodes form a trusted clique is what we want to achieve. 
When non-insured nodes connect to each other, they both end up with this payoff \ref{eq:notinsured payoff}. If they do not establish connection, they both receive the payoff \ref{eq:notinsured payoff2}.
\begin{equation}
U=\alpha - 2*r +\beta
\label{eq:notinsured payoff}
\end{equation}
\begin{equation}
U=\alpha - r
\label{eq:notinsured payoff2}
\end{equation}
From this equation we see that if $r>\beta$ then no connections will be made between non-insured nodes, because they would strictly prefer the payoff from not connecting.
If an insured agent connects to a non-insured one, he will end up with this payoff \ref{eq:insured-noninsured}. If he do not connect he will receive the payoff shown in \ref{eq:insured-noninsured2}.
\begin{equation}
U=\alpha - I_{0} - I_{l} - r + \beta
\label{eq:insured-noninsured}
\end{equation}
\begin{equation}
U=\alpha - I_{0}
\label{eq:insured-noninsured2}
\end{equation}
If $I_{l}+r>\beta$ the insured one will prefer not to connect, else he would prefer to establish a connection, and since the non-insured agent is always better off when connected to an insured agent, he will accept the establishment. This is easy to see when comparing the two payoffs. Payoff of connecting to an insured one: $U_{i}=\alpha - r + \beta$ is allways higher than not connecting: $U_{i}=\alpha - r$, as long as $\beta>0$.
\subparagraph{Multiple nodes}
We want our game to end up in a trusted clique of insured nodes, to achieve this we need to make sure that only insured nodes connect to each other, we want \ref{eq:insured-noninsured2} to be larger than \ref{eq:insured-noninsured}. I.e. This has to hold:
\begin{equation}
I_{l}+r>\beta
\label{eq:condition}
\end{equation}
When insured nodes connect to other insured nodes, they both end up with this payoff:
\begin{equation}
U=\alpha - I_{0} - I_{l} + \beta
\label{eq:insured-insured}
\end{equation}
To ensure that both agents will connect to each other this has to hold:
 \begin{equation}
I_{l}<\beta
\label{eq:condition2}
\end{equation}
For the game to end up as the one described earlier, where only insured nodes connect to other insured nodes, both \ref{eq:condition} and \ref{eq:condition2} has to hold, which gives us this limitation on the parameter $I_{l}$:
\begin{equation}
\beta-r<I_{l}<\beta
\label{eq:final condition}
\end{equation}
If the cost of insuring link is between these bounds, the insured nodes will only connect to other insured nodes. This holds for any number of players, because the relative change in payoff is linear and non-dependent on number of links already established. 
If the condition is fulfilled the game will end up in one or two cliques. Two cliques if the non-insured nodes connect to each other, and they do so if: $\beta>r$. 

In this model the network formation is done endogenously, and when following the limitation \ref{eq:final condition} we ensure that only insured nodes will connect, and the network ends up in a insurable network topology. 
We have neglected the information problem, because we made the assumption that all nodes can differentiate insured versus non insured nodes. This could be difficult to realize in many real world network, but in financial transactions and in software development networks, it is reasonable to assume that the parties can acquire this type of information regarding their transactional partners. And thus they solve the information asymmetry problem by them self, by requiring proof of insurance prior to establishing a connection. 
And as shown, if the insurance cost is within its needed limitations, the network will evolve endogenously, and end up with a insurable-clustered component. This is beneficial for both the insurer and the nodes.

\subparagraph{Example scenario}
By assigning values to the variables, we can show the outcome of the game in different scenarios. With the values from table \ref{tbl:simplegamevalue}, the insurance cost of establishing a link satisfies the condition in equation \ref{eq:final condition}.
If we play this game between two agents we can see the result in the normal form table \ref{fig:NFnumbers}. 
We see that the results are as we expected, insured nodes will only choose to connect with other insured nodes, they are also satisfied when not connected to each other, but this is not the social optimal outcome, and since they have complete network information, they will choose to connect to each other, because they will both achieve a higher payoff.
As we know, it does not matter if we only consider a two person game, because the change in payoffs of adding a link, is linear an independent of the agents degree, and if the insurance cost is right, it will never be beneficial for a insured node to connect to a non-insured. 

\begin{table}[h]
\centering
\begin{tabular}{lc}
 \hline
  $\alpha=10,
  \beta=10,
  I_{o}=5,
  I_{l}=3,
  r=8$\\
  \hline
\end{tabular}
\caption{Table showing the parameters and their assigned values \label{tbl:simplegamevalue}}
\end{table}

\begin{figure}[h]
\centering
\begin{tabular}{@{}c@{}}
\includegraphics[width=1.0\textwidth]{../Figures/gameTheoryModel1WithNumbers.png}
\end{tabular}
\caption[Caption for LOF]{Normal form game between two agents with the parameters given in \ref{tbl:simplegamevalue}, the best response of a player to a given strategy is marked with an underscore.There are two pure nash equilibriums in this game,$IL,IL$ and $I\underline{L},I\underline{L}$. \label{fig:NFnumbers}}
\end{figure}
When setting the insurance cost: $I_{l}<\beta - r$ or $I_{l}>\beta$ we are violating the condition who ensured that only insured nodes connected to each other.If we let the insurance cost be: $I_{l}<\beta - r$, like in the table \ref{tbl:simplegamevalue2}, we will get the results shown in Figure \ref{fig:NFnumbersViolating}. There are still the same nash equilibriums, but the interesting part is the best response of an insured agent who want to establish link, when the other agent is not insured but also want to establish link. This scenario has now changed, and we see that the insured agent would agree to the link establishment, this will result in an untrusted and thus an non-insurable topology. A game with multiple agents, insured and non-insured, will end up in one fully connected network.
On the other hand, if $I_{l}>\beta$, then it is easily seen that only the non-insured will connect to each other, and the insured nodes will choose not to connect to anyone because it is to expensive.
 Therefore to ensure that only insured agents connect to each other, equation \ref{eq:final condition} has to be satisfied. 
\begin{table}[h]
\centering
\begin{tabular}{lc}
 \hline
  $\alpha=10,
  \beta=10,
  I_{o}=5,
  I_{l}=1,
  r=8$\\
  \hline
\end{tabular}
\caption{Table showing the parameters and their assigned values, the insurance cost of establishing link is now violating the equation \ref{eq:final condition} \label{tbl:simplegamevalue2}}
\end{table}

\begin{figure}[h]
\centering
\begin{tabular}{@{}c@{}}
\includegraphics[width=1.0\textwidth]{../Figures/NotOptimalGameWithNumbers.png}
\end{tabular}
\caption[Caption for LOF]{Normal form game between two agents with the parameters given in table \ref{tbl:simplegamevalue2}, the best response of a player to a given strategy is marked with an underscore.There are two pure nash equilibriums in this game,$IL,IL$ and $I\underline{L},I\underline{L}$. \label{fig:NFnumbersViolating}}
\end{figure}
\subsection{Modelling network formation game}
To verify the result of this network formation game with multiple nodes, we performed different simulations.  The rules are as described earlier, when a node is considering establishing a link it chooses to do so if the payoff it will receive is larger than the payoff he already poses, and the decision is bilateral. 
In the simulator a node is insured with a probability, $p$. The network formation is performed by selecting two random nodes, not neighbouring each other, then both nodes checks if they would prefer to establish a connection or not. This selection is repeated until the network are fully connected or no more nodes are willing to establish new connections.
By selecting nodes at random and checking if both of them would like to connect to each other, we relax the assumption of full network information, because now nodes only get to know if another node is insured or not, by asking them.
\subparagraph{Simulation of game with optimal parameters}
\begin{figure}[h]
\centering
\begin{subfigure}{.5\textwidth}
  \centering
  \includegraphics[width=0.4\linewidth]{../Figures/FirstSimulationStart.png}
  \caption{\label{fig:firstsimulationstart} Ten nodes randomly, with probability 0.5, chosen to be either insured (green) or non-insured (red.}
\end{subfigure}
\quad
\begin{subfigure}{.46\textwidth}
  \centering
  \includegraphics[width=0.8\linewidth]{../Figures/FirstSimulationResult.png}
  \caption{\label{fig:firstsimulationresult} Two clustered fully connected networks. One consisting of insured agents the other consists of non-insured. The link establishment is done by following the rules described above}
\end{subfigure}
\caption{\label{fig:firstsimulation} The figure shows how ten nodes start out with no links, and then add links as long as they can increase their payoff, the result are two seperate cliques, one consisting of non-insured and the other of insured nodes.}
\end{figure}
In figure \ref{fig:firstsimulation} we see the result of a simulation with the parameters from table \ref{tbl:simplegamevalue}, with these parameters the game should end up in two cliques, one with insured nodes and another with non-insured. The result are shown in figure \ref{fig:firstsimulationresult}. Our calculations where confirmed, only insured nodes connect to each other.  
In this figure there are only included $n=10$ nodes, this is done to make the figure understandable and readable.
The same results where confirmed when performing the simulation with larger values of n, but the resulting picture was very complex and chaotic.
\subparagraph{Simulation of game with parameters violating equation \ref{eq:final condition}}
\begin{figure}[h]
\centering
\begin{subfigure}{.5\textwidth}
  \centering
\includegraphics[width=0.4\linewidth]{../Figures/FirstSimulationViolatingResult.png}

\caption{\label{fig:SimulationViolatingA} Ten nodes insured with probability 0.5, and the parameters from table \ref{tbl:simplegamevalue2}. The link insurance cost, $I_{l}$, is violating the condition in equation \ref{eq:final condition}, and the resulting network is one clique of both insured and non-insured nodes.}
\end{subfigure}
\quad
\begin{subfigure}{0.46\textwidth}
\centering
\includegraphics[width=0.4\textwidth]{../Figures/SimulationViolating2.png}

\caption{\label{fig:SimulationViolatingB} Ten nodes insured with probability 0.5, with parameters from table \ref{tbl:simplegamevalue2}, except that the link insurance cost was:, $I_{l}=11>\beta$. This resulted in a clique of only non-insured nodes. }
\end{subfigure}
\caption{\label{fig:SimulationViolating} The figure shows the two possible scenarios that violates the equation \ref{eq:final condition}, \ref{fig:SimulationViolatingA} shows the result when $I_{l}<\beta-r$ and \ref{fig:SimulationViolatingB} shows the result when $I_{l}>\beta$.}
\end{figure}

In this simulation, the cost of insuring a link where violating the equation \ref{eq:final condition}. The result can be seen in figure \ref{fig:SimulationViolating}.  In figure \ref{fig:SimulationViolatingA} we see the result when $I_{l}<\beta-r$, the result is one clique of both insured and non-insured nodes . In figure \ref{fig:SimulationViolatingB} the insurance cost is $I_{l}>\beta$, and as we see only non-insured nodes connect to each other, because the insurance cost per link cost more than the benefit gained from connecting to a new node, so the insured ones choose not to connect to any one. 

\subsection{Game with maximum node degree and bonus}
In this model we introduce a maximum node degree per node, and a payoff bonus when reaching this level. This is done to make a more applicable model. Lets consider a software firm who would want to develop a product, to complete the product they need help from some other firms, because these firms have more knowledge and are specialized on other things. When the product is finished they will get paid, but not before, and to finish the product they need to cooperate with other firms.
To model this scenario we added a bonus $\gamma$, which nodes receive when they reach their maximum node degree($m$). Except from this fact the game is as before, nodes connect to other nodes if they can reach a higher payoff by doing so.
\subsubsection{Four different scenarios}
To further analyse this model, lets take a closer look on the four different scenarios of the game, a insured node wants to connect to another insured node, insured tries to connect to a non-insured node, non-insured tries to connect to another non-insured node, and non-insured who tries to connect to a insured node.
\subparagraph{Insured to insured}
When a insured nodes tries to connect to another insured node, the node will do so if he can increase his payoff. 
let $U_{i}$ denote the payoff of a node with node-degree $i$. When adding a link the payoff the node receives is as follows:
\begin{equation}
    U_{i+1}= 
\begin{cases}
    \alpha + \beta - I_{0} - I_{l},& \text{if } i = 0\\
    U_{i}+\beta -I_{l},& \text{if }  i>0\\
    U_{i}+\beta -I_{l}+\gamma,& \text{if } i=m
    
\end{cases}
\label{eq:itoi}
\end{equation}
For insured nodes to connect to each other, $U_{i+1} > U_{i}$. This model is very similar to the earlier model, but we need to take the bonus it receives when reaching $m$ into consideration.
In this model this is done by adding, the bonus divided on the current degree, in the decision process of establishing link or not. 
\begin{eqnarray}
U_{i}+\beta-I_{l}+\frac{\gamma}{m-i}&>U_{i} \nonumber \\ 
\beta-I_{l}+\frac{\gamma}{m-i}&>0 \nonumber \\ 
\llap{$\rightarrow$\hspace{50pt}} \beta+\frac{\gamma}{m-i}&>I_{l} 
\label{eq:itoi2}
\end{eqnarray}
But the bonus is not added to the current payoff before the node reach the maximum degree, as shown in equation \ref{eq:itoi}.
In this way the model will change from the former models, because now the nodes have more incentive to connect to other nodes, and in some scenarios they will be willing to take a risk by connecting to a non-insured node that they would not have done before, due to the expected bonus. I.e. we insert some risk, because it is not certain that the nodes will obtain enough links, and if not, they will not receive their bonus, but they have already connected to the nodes.

\subparagraph{Insured connect to non-insured}
The payoff the insured node receives in this scenario is as follows:
\begin{equation}
    U_{i+1}= 
\begin{cases}
    \alpha + \beta - I_{0} - I_{l} -r,& \text{if } i = 0\\
    U_{i}+\beta -I_{l}-r,& \text{if }  i>0\\
    U_{i}+\beta -I_{l}-r+\gamma,& \text{if } i=m
\end{cases}
\label{eq:itonoti}
\end{equation}
To establish a connection from an insured node to a non-insured one, this has to hold:
\begin{eqnarray}
U_{i}+\beta-I_{l}-r+\frac{\gamma}{m-i}&>U_{i} \nonumber \\ 
\beta-I_{l}-r+\frac{\gamma}{m-i}&>0 \nonumber \\ 
\llap{$\rightarrow$\hspace{50pt}} \beta+\frac{\gamma}{m-i}-r&>I_{l} 
\end{eqnarray}
\subparagraph{non-insured to non-insured}
When a non-insured node connect to another not-insured node this is the payoff they receive:
\begin{equation}
    U_{i+1}= 
\begin{cases}
    \alpha + \beta -r,& \text{if } i = 0\\
    U_{i}+\beta -r,& \text{if }  i>0\\
    U_{i}+\beta -r +\gamma,& \text{if } i=m
\end{cases}
\end{equation}
And to establish the connection this equation has to hold:
\begin{eqnarray}
U_{i}+\beta-r+\frac{\gamma}{m-i}&>U_{i} \nonumber \\ 
\beta-r+\frac{\gamma}{m-i}&>0 \nonumber \\ 
\llap{$\rightarrow$\hspace{50pt}} \beta+\frac{\gamma}{m-i}&>r
\end{eqnarray}
\subparagraph{non-insured to insured}
\begin{equation}
    U_{i+1}= 
\begin{cases}
    \alpha + \beta,& \text{if } i = 0\\
    U_{i}+\beta,& \text{if }  i>0\\
    U_{i}+\beta +\gamma,& \text{if } i=m
\end{cases}
\end{equation}
As we this is a strictly increasing function, and thus a non-insured will always connect to an insured node if given the option, and as long as $\beta$ is positive, but this is given in the rules of the model.
\subparagraph{Limitations to ensure a clique of only insured}
We are interested in finding insurable topologies, and one insurable topology is a clique consisting of only insured nodes. By analysing the different scenarios above we can find limitations on the cost of insuring a link, that will force the network formation game to end up in an insurable clique.
We know that a insured node would want to connect to another ensured node if the equation \ref{eq:itoi2} is satisfied. 
In the equation we see that the expected bonus per established link is increasing, i.e. if a insured node of degree zero is willing to connect to another insured node, then every node with a degree bigger than zero also would like to connect to another insured node. Thus to ensure that insured nodes connect to each other this equation has to hold:
\[
\beta+\frac{\gamma}{m}>I_{l}
\label{eq:conditionitoi}
\]
We also want to ensure that insured nodes never establishes links with non-insured nodes, from \ref{eq:itonoti} we see that this has to hold:
\[
\beta+\frac{\gamma}{m-i}-r < I_{l}
\]
This can be simplified, we know the insured nodes with degree $m-1$ get the highest expected bonus when establishing a new link, i.e. if we can ensure that nodes with this degree do not establish links to non-insured nodes, we also know that every node with degree less than $m-1$ will not establish links to non-insured nodes. From this we get the equation:
\begin{eqnarray}
\beta+\frac{\gamma}{m-(m-1)}-r&<I_{l} \nonumber\\
\llap{$\rightarrow$\hspace{50pt}} \beta+\gamma-r&<I_{l}
\label{eq:condition-i-to-noti}
\end{eqnarray}
From the equations \ref{eq:conditionitoi} and \ref{eq:condition-i-to-noti} we get this final limitation on the link insurance cost:
\begin{equation}
\beta+\gamma-r<I_{l}<\beta+\frac{\gamma}{m}
\label{eq:final-insurance-clique-condition}
\end{equation}
A important fact about this limitation is that to make it possible for the game to end up in an insurable topology equation \ref{eq:noinsuredcliqueconditon} has to be satisfied. As we see from the equation, if the risk to bonus ratio gets to small it gets more and more unlikely to ensure an insurable topology. If we think about a real world scenario where you get a bonus for establishing a fixed number of equations, you would be more willing to take a risk if the possible reward of doing so is large, this is what this equation express. 
\begin{eqnarray}
\gamma-r &<\frac{\gamma}{m}\nonumber \\
1-\frac{r}{\gamma}&<\frac{1}{m} \nonumber \\
\llap{$\rightarrow$\hspace{50pt}}1-\frac{1}{m}&<\frac{r}{\gamma}
\label{eq:noinsuredcliqueconditon}
\end{eqnarray}

\subparagraph{Simulations}
First we simulate a network formation game when the link insurance cost satisfied the equation \ref{eq:final-insurance-clique-condition}. 
\begin{table}[h]
\centering
\begin{tabular}{lc}
 \hline
  $\alpha=10,
  \beta=10,
  I_{o}=5,
  I_{l}=9,
  r=8,
  \gamma=5,
  m=5
  $
  \\
  \hline
\end{tabular}
\caption{Parameters used in simulation \label{tbl:maxdegrevalues}}
\end{table}
\begin{figure}[h]
\centering
  \includegraphics[width=0.8\linewidth]{../Figures/BonusGameInsuredClique.png}
  \caption{\label{fig:bonusoptimal} Two clustered fully connected networks, created by simulating with the parameters from table \ref{tbl:maxdegrevalues} One consisting of insured agents the other consists of non-insured. }
\end{figure}
As we see in figure \ref{fig:bonusoptimal} the results where as expected, the cost of insuring a link satisfied the conditions found earlier and thus the result where two cliques, one consisting of only insured and the other of non-insured nodes.
If we change the link insurance cost, so it is just below the limit,$I_{l}=6$, the result is as shown in figure \ref{fig:bonusviolating}. Here we see that some of the insured node have taken a risk by connecting to one non-insured node to receive their bonus.  
\begin{figure}[h]
\centering
  \includegraphics[width=0.8\linewidth]{../Figures/BonusGameViolating.png}
  \caption{\label{fig:bonusviolating} Simulation when the cost of insuring a link is just below the limits. }
\end{figure}

\subsection{DETTE ET ANNET STED KANSKJE? BLIR LITT RART Å HOPPE INN I DET HER}
By adjusting the parameter one can assure that only insured agents connects to other insured agents, and the opposite,
that only uninsured agents connects to each other. Hence as we can see from the figure \ref{fig:fincont} clustered
networks of insured agents (red) are created, and  as the paper \cite{contagion} showed, these clustered trusted
networks, can achieve higher, super-critical, payoff by increasing their node degree past the critical point.

\begin{figure}[h]
\centering
\begin{subfigure}{.5\textwidth}
  \centering
  \includegraphics[width=0.8\linewidth]{../Figures/financialContagion1.png}
  \caption{\label{fig:fincont1} Initial graph with 10 agents.}
\end{subfigure}
\quad
\begin{subfigure}{.46\textwidth}
  \centering
  \includegraphics[width=0.8\linewidth]{../Figures/financialContagion2.png}
  \caption{\label{fig:fincont2} Insured agents (red) forms a network}
\end{subfigure}
\caption{\label{fig:fincont} shows how insured agents connects with each other to form a network to achieve super-critical payoffs.}
\end{figure}

, because the nodes can thus receive a super-critical payoff, and they are also insured against contagious risk.  

Figure \ref{fig:GTmodel1equations} presents the individual payoffs in a formation game between two agents in the described model. It is assumed that both agents has to have a desire to establish a connection in order to create a link between them. This is reasonable since a company would not prefer to enter into an agreement with negative expected payoff. As in this case would be the result when an insured agent is requested a connection with someone without insurance. 
 
 
\begin{figure}[h]
\centering
\begin{tabular}{@{}c@{}}
\includegraphics[width=1.0\textwidth]{../Figures/gameTheoryModel1WithEquations.png}
\end{tabular}
\caption[Caption for LOF]{\label{fig:GTmodel1equations} Normal form game between two agents individually choosing to purchase insurance and express desire to connect to the other  \footnotemark }
\end{figure}
\footnotetext{A link will only be created if both agents wishes to establish a connection.}

If we give value to the variables in figure \ref{fig:GTmodel1equations} one can observe the model's different equilibrium's. It is difficult to know exactly how the variables are set and this would vary considerably between different markets. In a real worlds scenario the variables would also be different for each agent. However in figure \ref{fig:GTmodel1} we decided to set a fixed value (which is assumed to be corresponding to the real values) for each variable in order to show a concept of how cyber-insurance can be used to create beneficial payoffs.
The following values where used: $\alpha$ = 10, $\beta$ = 10, $I_{o}$ = 5, $I_{l}$ = 2, $r$  = 20, $q$ = 0.5.


\begin{figure}[h]
\centering
\begin{tabular}{@{}c@{}}
\includegraphics[width=0.6\textwidth]{../Figures/gameTheoryModel1WithNumbers.png}
\end{tabular}
\caption{\label{fig:GTmodel1} Shows equilibrium's in the resulting payoff matrix.}
\end{figure}

From the payoff matrix \ref{fig:GTmodel1} we observe two different Nash equilibrium's: One when both agents are insured and wants to connect to the other agent, and one when both are insured but does not want to establish a connection. These are the possible outcomes between the two agents, however as we can see it the social optimal solution would be for two insured agents to connect with each other, i.e they would both receive a significantly higher payoff. This demonstrates that a cluster of insured nodes would achieve higher payoffs.  
