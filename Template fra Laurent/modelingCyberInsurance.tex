\chapter{Modeling Cyber-Insurance }
\label{chp:modelingCyberInsurance} 

In many scenarios nodes seeks to create networks in order to directly benefit from each other. The established links might represent companies out sourcing part of their manufacturing, or cooperative agreements in the development of new software products. In addition to increase the trade-off, each of the established links represents risk of being a victim of cascading failures. The intuitive example is the spread of epidemic diseases, also node failures of a power grid and financial contagion such as the one back in 2008 was a result of cascading failures. Strategic network formation using cyber-insurance can be used to prevent such situation in addition to increase the overall payoff of participants in a clustered network.


When deciding whether to establish connection to a neighbor agent, the payoff has to be higher in the balance between the expected earnings and the risk of the other party failing to complete the transaction. This is the reason why we seek to only download content from trusted peers and outlaw MC-gangs are consistently skeptical to enter into new agreements despite promising increased earnings, since the risk of undercover police are too high. 

\subparagraph{FLYTT DETTE TIL BACKGROUND og referer til det når vi viser at cliques er bra}
The paper \cite{contagion} come up with some interesting results regarding network formation games. 
They set up a game where the nodes benefit from direct links, but these links also expose them for risk. 
Each node gains a payoff of  $a$ per link it establishes, but it can establish a maximum of $\delta$ links.
A failure occur at a node with probability $q$, and propagates on a link with probability $p$. If a nodes fail, it will receive a negative payoff of $b$, no matter how many links it has established.

The results from their model shows a situation where clustered graphs achieve a higher payoff when connected to trusted agents, compared to when connecting with random nodes. Unlike in anonymous graphs, where nodes connect to each other at random, nodes in these graphs share some information with their neighbors, which is used when deciding whether to form a link or not. 
To further explain these results, they show that there exists a critical point, called \textit{phase transition}, which occurs when nodes have a node degree of $frac{1}{p}$. 
At this point a node gets a payoff of $frac{a}{p}$, and to further increase the payoff the node needs to go into a region with significantly higher failure probability. 
Because once each node establish more than $frac{1}{p}$ links, the contagious edges, will with high probability form a large cluster. Which results in a rise in probability of node failure, and reduces the overall welfare.
From this the paper say that when the minimum welfare exceeds 
$(1+f(\delta)*frac{a}{p})$
we have reached super critical payoff. Otherwise it is called sub-critical payoff. 
Further they show that the only possible way of ending up with supercritical payoff, is by forming clustered networks consisting of cliques with slightly more than $frac{1}{p}$ nodes. 
If the nodes form an anonymous market, random linking, they can only get sub-critical payoff. 
In other words, if the nodes can choose who they connect with, and by doing so, creating trusted clustered markets, they can achieve a higher payoff, by exceeding the critical node degree point. But in random graphs, this is not possible. 


Inspired by this model, we are step wise building a model which shields light on how cyber-insurance can be used in network formation games to prevent cascading failures and increase an agents payoff.  

\section{Model 1 - Initial Model \label{section:verysimplemodel}}

There are many examples of nodes needing to establish connections, one example is a company needing to out-source certain tasks to remain competitive. This outsourcing involves some risks, such as, will the company deliver at the reported time, to the reported costs, what happens if they fail to deliver, what if they go bankrupt etc. If the companies that are going to establish links(cooperative contracts), know that the other firms are insured, it will be more secure and reliable to enter into an cooperative agreement. In this way trusted cliques can evolve. The firms benefit from connecting to other insured firms, and the insurance company can offer fair prices to the insured companies, because the risk is calculable in a trusted clique.

As a starting point the model is highly simplified in order to show the concept of how cyber-insurance can be used to create an insurable topology. We assume that every node has complete network information, i.e. it knows how many nodes that exists, if they are insured or not. The link establishment process is bidirectional, meaning both nodes must agree to establish the connection.
Through out this chapter new features will be added to the model to make it more realistic and applicable. 

A set of $n$ nodes are randomly chosen to be insured or not, as depicted in Figure \ref{fig:firstmod1}. They all get their own fixed income, and by connecting to other nodes they can increase their payoff. Non-insured nodes will have a risk of failure i.e. an expected cost of failure. Therefore if an insured agents chooses to connect to a non-insured nodes they will also suffer from this expected cost of failure. To simplify the decision process, the model follows a rule that only allows insured to connect to other insured agents and non-insured agents can only connect with each other. 

The resulting graph will be two fully connected cliques, one consisting of insured agents and the other of non-insured agents, as shown in Figure  \ref{fig:firstmod2}. 


This dichotomy represents a trusted environment for the insured nodes, because they know that each node in the clique is insured against risk. These nodes will benefit from each connection without having to worry about contagious risks from the connected nodes. 
A node in the non-insured clique will also experience a change in payoff from the links it has established, however each of the links has a probability of failure. Hence this environment is not trusted, and a link establishment will always involve some risk. 

Hence this model, although very simple, shows an insurable topology where insured agents benefit from being insured.

\begin{figure}[h]
\centering
\begin{subfigure}{.5\textwidth}
  \centering
  \includegraphics[width=0.4\linewidth]{../Figures/firstModelWithNoParameters1.png}
  \caption{\label{fig:firstmod1} 15 Agents randomly choosen to be either insured (green) or non-insured (red).}
\end{subfigure}
\quad
\begin{subfigure}{.46\textwidth}
  \centering
  \includegraphics[width=0.8\linewidth]{../Figures/firstModelWithNoParameters2.png}
  \caption{\label{fig:firstmod2} Two clustered networks. One consisting of insured agents the other consists of non-insured.}
\end{subfigure}
\caption{\label{fig:firstmodfinal} Shows how agents connects to eachother according to model described in section \ref{section:verysimplemodel}.}
\end{figure}
 
\textbf{fjern dette?}
This model is very simplified and suffer from many limitations, among others it is too simple to reflect the dynamics of a real world scenario, where each node will have different variables with different values. Although it tries to deal with the problem of correlated risks and preventing free riders from entering the trusted clique (interdependent security problem), each node have a complete network information i.e. the problem with information asymmetry is not taken into account. 
\textbf{slutt paa fjern dette}

\section{Model 2 - Including Parameters}
To make the simple model more realistic, we have to introduce some parameters, that reflects real world scenarios.
It is fair to assume that the insured nodes must pay an insurance premium, and this premium should be dependent on the number of links the node establishes. When two insured nodes establish a link between each other, they both have to pay a premium, this is to make the game more fair, and more realistic. For example if the two nodes had different insurance companies, then both companies would charge them for insuring the link.
When a node, insured or not, establish a link to a non-insured node, this involves a risk, and this risk will be represented as a expected risk cost. However if the changes in payoff when establishing a link is only negative, then no node would want to establish links. Thus the nodes will also receive a positive change in payoff when establishing different links. 

\subsection{Characteristics of the model}
The type of the nodes are given in advance, i.e. they are chosen to be insured or non-insured. The process of establishing link is a bidirectional decision. 
The insured nodes have to pay an insurance cost $I_{0}$, which represents the cost of signing a contract with an insurance company. I.e this could be an actual fee or a cost reflecting the work a player has to do to get a contract. The insurance premium is $I_{l}$, the expected risk cost is represented by $r$. $\beta$ represents the benefit of establishing a link.  
Table \ref{tbl:simplegamepara} presents an overview of the parameters. 
\begin{table}[h]
\centering
\begin{tabular}{lc}
 \hline
  $\beta$ - income from establishing a direct link \\
  $I_{o}$ - cost of having insurance. \\
  $I_{l}$ - increased insurance cost per link the node establishes\\
  $r$ - expected risk cost\\
  \hline
\end{tabular}
\label{tbl:simplegamepara}
\end{table}
\subsection{Two nodes scenario}
To begin analyzing the model, lets start with a simple scenario involving only two nodes. In this game the strategy space of both players consist of four different strategies. A node can be insured or not, and choose whether to establish a link to the other node or not. I.e. the different strategies are: Be insured and establish link noted as: $IL$, 
be insured and not establish link: $I\overline{L}$. Not insured and establish link: $\overline{I}L$, and not insured and not establish link: $\overline{IL}$. It should be noted that since the decision to establish a connection is bidirectional, both have to choose a strategy where they want to establish a link, for the link establishment to be successful.
Figure \ref{fig:FirstGameTheoryModel} shows the different outcomes of this game.

\begin{sidewaysfigure} 
\centering
\includegraphics[width=1.0\textwidth]{../Figures/FirstGameWithParameters.png}
\caption{\label{fig:FirstGameTheoryModel} Normal form game, showing the different strategies and the payoffs  for the different outcomes. The payoff are written in this order, A then B's. An agent has a strategy space of size 4. Maa ENDRES, FIKS NAVN IKKE FIRM, MEN NODE}
\end{sidewaysfigure}

As long as both $I_{l} \text{ and } r$ is less than $\beta$, the only nash equilibrium in Figure \ref{fig:FirstGameTheoryModel} is when both nodes chooses $\overline{I}L$. If we first look at node A, we see that when node B chooses $IL$, the best response is $\overline{I}L$, because $ \beta >\beta- I_{l}$ . And since the game is symmetric, the same holds for node B. 
When one of the nodes chooses $\overline{I}L$, the best response will be $\overline{I}L$, because $\beta - r>\beta - I_{l} - r  $. 
And thus the only nash equilibrium is when both nodes play $\overline{I}L$.

This means that two nodes will end up in a classic prisoner's dilemma
\footnote{Prisoner's dilemma was originally framed by Merrill Flood and Melvin Dresher in 1950. The dilemma expresses a situation where two players each have two options whose output depends on the simultaneous choice made by the other. In the original dilemma concerns two prisoners which separately decides whether to confess to a crime \cite{oxfordPrisonersDilemma}. It is a paradox in decision analysis which shows why two individuals might not cooperate, even if it is in their best interest to do so.}
, where the best response is actually worse than the social optimal. In this case it is trivial to see that the social optimal scenario is for both nodes to choose $IL$, as long as $I_{l}<r$. However,  the nodes will choose not to buy insurance. Or else they could risk ending up in a case where they pay $I_{l}$ without receiving any other benefit. 

  
\section{Solving the prisonersdilemma}
One possibility for solving the problem that arised in the last section is to introduce a leader follower game. In this game the players does not act at the same time, but in order, and they can observe the other players action.
If we consider a game with only two players, player one are the first to select an action. He chooses to insure or not. Then after observing this action player two chooses if he would like to insure or not. Then they choose if they would establish link or not, in the same order.
The stackelberg model or the leader follower game, can be solved to find the subgame perfect Nash equilibrium. This is a strategy profile where every player plays his best response to the other players strategies. 


In this type of game the leader, will benefit from a first mover advantage, because he can now force the game in a direction he prefers. 
\begin{equation}
I_{l}<\beta \text{ and } I_{l}>\beta-r \text{ and } r<\beta
\label{eq:stackelbergcondition}
\end{equation}
By finding all subgame equilibria in Figure \ref{fig:stackelberg} except the last one, i.e. the subgame where player one chooses to Insure or not, we get this subgame equilibria: $(L,\overline{L^{I}_{1}},\overline{L^{II}_{1}},L^{III}_{1}), (I_{2},\overline{I^{I}_{2}},L_{2}L^{I}_{2},\overline{L^{II}_{2}},L^{III}_{2})$
We have now analyzed the different outcomes of player 1 choosing insure or not, thus he can now see what is the best thing for him to do. The two options he can choose between are: Insure and get payoff $\beta-I_{l}$ or not insure and get payoff $\beta-r$. I.e. if $I_{l}<r$ player one will chose to insure, and thus forcing the game to end up in a equilibrium where both players insure and establish link. If the cost of insuring is higher than the expected risk cost $r$, then obviously there is no reason to choose insurance.
From this we see that if the insurance price are set to the right amount, one player can force the outcome of the game to be the socially optimal outcome. 

Solve prisoners dilemma problem by forcing the other one to choose insurance


\begin{figure}[h]
\centering
  \includegraphics[width=0.9\linewidth]{../Figures/stackelberggame.png}
  \caption{\label{fig:stackelberg} Leader follower game, first player 1 chooses to insure or not, then player 2, and then they choose to establish link or not in the same order.}
\end{figure}


\subsection{Multiple nodes}
\subsubsection{Assumptions}
In this model we introduce multiple nodes, the type of the node is as before given in advance. The objective of this model is to find characteristic network formations that will evolve endogenously when the parameters are within certain conditions. Examples of characteristic networks of interest are cliques, scale-free and star networks.
We assume that every node has complete information of the network, i.e. every node knows the type of the other players. This is a very strong assumption, however in financial transactions and in software development networks, it is reasonable to assume that the parties can acquire this type of information regarding their transactional partners. For example by requiring proof of insurance prior to establishing a financial contract.
\subsubsection{Analyzis}
As mentioned our goal is to find how and when certain network formations evolve. We know that if a node can increase his payoff by establishing a link, he will do so. And thus we start by analyzing the four possible link establishment scenarios, insured to insured, insured to non-insured, non-insured to insured, and non-insured to non-insured. 
Let $U_{i}$ denote the payoff of a node with degree $i$, and let $U_{i+1}$ be the payoff a node will receive if it establishes a new link.
\subparagraph{Insured to insured}
When two insured nodes are considering establishing a link, they will do so, only if both receive a higher payoff.  In this scenario the the payoff function of adding a link is as shown in Eq.(\ref{eq:itoimodel2}).
\begin{equation}
    U_{i+1}= 
\begin{cases}
    \beta - I_{l},& \text{if } i = 0\\
    U_{i}+\beta -I_{l},& \text{if }  i>0
   
\end{cases}
\label{eq:itoimodel2}
\end{equation}
For insured node to connect to another insured node, the condition shown in Eq.(\ref{eq:itoi-condition}) has to hold.
\begin{equation}
I_{l}<\beta
\label{eq:itoi-condition}
\end{equation}
\subparagraph{Non-insured to insured}
The payoff a non-insured receives by connecting to a insured is as described in Eq. (\ref{eq:noti-to-i-model2}). As we see this will allways be a positive change in payoff, and thus an non-insured node will allways, if possible, want to connect to an insured node.
\begin{equation}
    U_{i+1}= 
\begin{cases}
    \beta,& \text{if } i = 0\\
    U_{i}+\beta,& \text{if }  i>0
   
\end{cases}
\label{eq:noti-to-i-model2}
\end{equation}
\subparagraph{Insured connect to non-insured}
The payoff a insured node receives in this scenario is as follows:
\begin{equation}
    U_{i+1}= 
\begin{cases}
    \beta - I_{l}-r,& \text{if } i = 0\\
    U_{i}+\beta -I_{l}-r,& \text{if }  i>0
   
\end{cases}
\label{eq:itonotimodel2}
\end{equation}
For this to happen Eq.(\ref{eq:itonoti-condition}) has to hold, a non-insured node will allways want to connect to a insured one, so this is the only condition that is needed for this to happen.
\begin{equation}
I_{l}+r<\beta
\label{eq:itonoti-condition}
\end{equation}
\subparagraph{Non-insured to Non-insured}
The payoff a non-insured nodes receives when connecting to another non-insured node is as follows:
\begin{equation}
    U_{i+1}= 
\begin{cases}
    \beta -r,& \text{if } i = 0\\
    U_{i}+\beta -r,& \text{if }  i>0
   
\end{cases}
\label{eq:notitonotimodel2}
\end{equation}
And for this link-establishment scenario to happen Eq.(\ref{eq:notitonoti-condition} has to hold.

\begin{equation}
\beta>r
\label{eq:notitonoti-condition}
\end{equation}
We want to find when different network structures evolve, one good starting point for this is a clique of only insured nodes. For this to happen, first of all insured nodes must connect to each other, i.e. Eq.(\ref{eq:itoi-condition}) has to hold. But we also need to ensure that insured nodes do not establish links with non-insured nodes. I.e. this has to hold:
\begin{equation}
I_{l}+r>\beta
\end{equation}
This gives us the limitation shown in Eq. (\ref{eq:final condition}) on insurance link cost.
\begin{equation}
\beta-r<I_{l}<\beta
\label{eq:final condition}
\end{equation}
If the condition is fulfilled the game will end up with a clique of only insured nodes, and another clique of non-insured nodes if: $\beta>r$. 

\subsubsection{Result and findings}

As shown, if the insurance cost is within its needed limitations, the network will evolve endogenously, and end up with an insurable-clustered component, which is beneficial for both the insurer and the nodes.
These findings is independent of number of players, because we only consider one link at a time, and the change in payoffs is linear an independent of the nodes degree.
\begin{table}[h]
\centering
\begin{tabular}{lc}
 \hline
  $\beta=10,
  I_{o}=5,
  I_{l}=3,
  r=8$\\
  \hline
\end{tabular}
\caption{Table showing the parameters and their assigned values \label{tbl:simplegamevalue}}
\end{table}

On the other hand, if $I_{l}>\beta$, then it is easily seen that only the non-insured will connect to each other, and the insured nodes will not choose to connect to anyone because it is to expensive.
 
\begin{table}[h]
\centering
\begin{tabular}{lc}
 \hline
  $\beta=10,
  I_{o}=5,
  I_{l}=1,
  r=8$\\
  \hline
\end{tabular}
\caption{Table showing the parameters and their assigned values, the insurance cost of establishing link is now violating Eq.(\ref{eq:final condition}) \label{tbl:simplegamevalue2}}
\end{table}
\section{Simulating Model 2}
To verify the result of this network formation game with multiple nodes, we performed different simulations. The network formation is performed by selecting two random nodes, not neighbouring each other, then both nodes checks whether they would prefer to establish a connection or not. The rules are as described earlier, when a node is considering establishing a link it chooses to do so if the payoff received is larger than the payoff he already poses, and the decision is bilateral. 
In the simulator a node is insured with a probability, $p$. This selection is repeated until the network are fully connected or no more nodes are willing to establish new connections.
By selecting nodes at random and checking if both of them would like to connect to each other, we relax the assumption of full network information, because now nodes only get to know if another node is insured or not, by asking them.

\subparagraph{Simulation of game with optimal parameters}
\begin{figure}[h]
\centering
\begin{subfigure}{.5\textwidth}
  \centering
  \includegraphics[width=0.4\linewidth]{../Figures/FirstSimulationStart.png}
  \caption{\label{fig:firstsimulationstart} Ten nodes randomly, with probability 0.5, chosen to be either insured (green) or non-insured (red.}
\end{subfigure}
\quad
\begin{subfigure}{.46\textwidth}
  \centering
  \includegraphics[width=0.8\linewidth]{../Figures/FirstSimulationResult.png}
  \caption{\label{fig:firstsimulationresult} Two clustered fully connected networks. One consisting of insured agents the other consists of non-insured. The link establishment is done by following the rules described above}
\end{subfigure}
\caption{\label{fig:firstsimulation} The figure shows how ten nodes start out with no links, and then add links as long as they can increase their payoff, the result are two seperate cliques, one consisting of non-insured and the other of insured nodes.}
\end{figure}
In Figure \ref{fig:firstsimulation} we see the result of a simulation with the parameters from table \ref{tbl:simplegamevalue}. Since these parameter values holds for the condition Eq.(\ref{eq:final condition}), the game should end up in two cliques, one with insured nodes and another with non-insured. The result are shown in Figure \ref{fig:firstsimulationresult}, and confirms our calculations where only insured nodes connect to each other.  
In this figure there are only included $n=10$ nodes, this is done to make the figure readable and easy to understand.
The same results where obtained when performing the simulation with larger values of n, however the resulting printouts was very complex and chaotic.


\subparagraph{Simulation of game with parameters violating Eq.(\ref{eq:final condition})}
\begin{figure}[h]
\centering
\begin{subfigure}{.5\textwidth}
  \centering
\includegraphics[width=0.4\linewidth]{../Figures/FirstSimulationViolatingResult.png}

\caption{\label{fig:SimulationViolatingA} Ten nodes insured with probability 0.5, and the parameters from table \ref{tbl:simplegamevalue2}. The link insurance cost, $I_{l}$, is violating the condition in Eq.(\ref{eq:final condition}), and the resulting network is one clique of both insured and non-insured nodes.}
\end{subfigure}
\quad
\begin{subfigure}{0.46\textwidth}
\centering
\includegraphics[width=0.4\textwidth]{../Figures/SimulationViolating2.png}

\caption{\label{fig:SimulationViolatingB} Ten nodes insured with probability 0.5, with parameters from table \ref{tbl:simplegamevalue2}, except that the link insurance cost was:, $I_{l}=11>\beta$. This resulted in a clique of only non-insured nodes. }
\end{subfigure}
\caption{\label{fig:SimulationViolating} The figure shows the two possible scenarios that violates the Eq.(\ref{eq:final condition}), \ref{fig:SimulationViolatingA} shows the result when $I_{l}<\beta-r$ and \ref{fig:SimulationViolatingB} shows the result when $I_{l}>\beta$.}
\end{figure}

In this simulation, the cost of insuring a link where violating the Eq.(\ref{eq:final condition}). The result can be seen in Figure \ref{fig:SimulationViolating}.  In figure \ref{fig:SimulationViolatingA} we see the result when $I_{l}<\beta-r$, the result is one clique of both insured and non-insured nodes. In figure \ref{fig:SimulationViolatingB} the insurance cost is $I_{l}>\beta$, and as we see only non-insured nodes connect to each other, because the insurance cost per link cost more than the benefit given from connecting to a new node, i.e. the insured ones choose not to establish any connections. 
  
....SLETTE DETTE NEDENFOR?!?!  
do this we have to analyze the different scenarios. 
When non-insured nodes connect to each other, they both end up with the payoff shown in Eq. (\ref{eq:notinsured payoff}). If they do not establish connection, they both receive the payoff shown in Eq. (\ref{eq:notinsured payoff2}).
\begin{equation}
U=- 2*r +\beta
\label{eq:notinsured payoff}
\end{equation}

\begin{equation}
U=- r
\label{eq:notinsured payoff2}
\end{equation}

From this equation we see that if $r>\beta$ then no connections will be made between non-insured nodes, because they would strictly prefer the payoff from not connecting.
If an insured node connects to a non-insured one, he will end up with the payoff in Eq.(\ref{eq:insured-noninsured0}). If he do not connect he will receive the payoff shown in Eq.(\ref{eq:insured-noninsured2}).

\begin{equation}
U= - I_{0} - I_{l} - r + \beta
\label{eq:insured-noninsured0}
\end{equation}

\begin{equation}
U= - I_{0}
\label{eq:insured-noninsured2}
\end{equation}

If $I_{l}+r>\beta$ the insured one will prefer not to connect, else he would prefer to establish a connection, and since the non-insured agent is always better off when connected to an insured agent, he will accept the establishment. This can be shown when comparing the two payoffs. Payoff of connecting to an insured one: $U_{i}=- r + \beta$ is allways higher than not connecting: $U_{i}= - r$, as long as $\beta>0$.


.... UNDER HER ER TING GREIT. 

\section{Forcing non-insured nodes to buy insurance(FIX!!!!!!!!)}

In this section we try to find a condition which gives all nodes incentive to buy insurance. The basic idea is to create a scenario where it is beneficial for a node to be insured, i.e the non-insured nodes wants to to purchase insurance. 
This scenario will also benefit the insurer, obviously because more nodes purchases insurance. In addition, the insurer now have incentive to handle the problem with asymmetry. In previous models, the insurer would have difficulty obtaining sufficient information to calculate a node's risk. Because the nodes would did not have incentive to provide information to the insurance company. Hence the insurance company could enter into risky contracts. Now, we have a different scenario. Since non-insured nodes want to be insured, they can be forced to give up information about their current condition, both financial and list possible risks. From the market survey, we found that companies offering cyber-insurance actually required this information. The information received can be analyzed in different ways. If the nodes provide enough information, the insurer are able to calculate the risk, and offer a premium. On the other hand, if a node acts suspiciously and tries to hide information from the insurer, the insurer have reasonable cause to not chose to insure the node. Either way, it is a seller's market, where the insurance company can dictate the outcome of the network by pricing the insurance according to equations provided in this section.  

Initial conditions are equal to the previous, where every node are randomly chosen to be either insured or not. Hence we could use the same payoff matrix as shown in Figure \ref{fig:FirstGameTheoryModel} to analyze how we could force the non-insured nodes to purchase insurance. In order to give incentive for a non-insured node to purchase insurance, the payoff has to always be higher, i.e. 
\begin{equation}
\textit{Utility insured node > Utility non-insured node}
\end{equation}



This means that we need to make sure that a non-insured node in any circumstances will benefit from purchasing insurance. From the payoff matrix, Figure \ref{fig:FirstGameTheoryModel} we find the different conditions. When a connection is established, we need the payoff for insured nodes to be higher than non-insured nodes: 


\begin{eqnarray}
\beta - I_{0} - I_{l} > \beta - r \nonumber \\ 
\llap{$\rightarrow$\hspace{50pt}}  I_{0} + I_{l} < r 
\label{eq:model2extention:1}
\end{eqnarray}

For the other case, when the nodes have not established any connections the following has to hold: 
\begin{eqnarray}
 - I_{0} > -r \nonumber \\ 
\llap{$\rightarrow$\hspace{50pt}}  I_{0} < r 
\label{eq:model2extention:2}
\end{eqnarray}

In addition we need to make sure that it is not beneficial for insured nodes to connect to non-insured nodes: 

\begin{equation}
I_{l}+r < \beta
\end{equation}

If these conditions are met, we are guaranteed to get a network consisting of only insured nodes. Because in any case, the non-insured nodes will get a higher payoff from purchasing insurance. It is interesting to see that both conditions are completely dependent upon how the insurance company chooses to price their products. If the insurer collects enough information to calculate an accurate risk, he could price both $I_{0}$ and $I_{l}$ to meet the conditions. Hence he forces the network to end up with every node having incentive to purchase insurance.

\subsection{Violating the conditions} 
Since the risk are difficult to calculate, there is a possibility of ending up in states where a node would actually benefit from doing the opposite. If the actual scenario ends up with the following conditions: 


\begin{eqnarray}
I_{0}< r \nonumber \\ 
I_{l}> r
\label{eq:model2extention:3}
\end{eqnarray}

Now we will have a situation where it first looks beneficial to be purchase insurance. However, as the nodes adds more connections, and pays $I_{l}$ pr connection, the node would actually be better of with not being insured. This demonstrates the importance of being able to accurately calculate the risk. 








\section{Model 3 - Including maximum node degree and bonus}
We keep adding new features to the model, here we introduce a maximum node degree per node, and a payoff bonus when reaching this level. This is done to make a more applicable model in certain scenarios. For example, lets consider a software company who want to develop a new product. However, they do not have the required resources or knowledge to complete the product, and will therefore need help from other companies with the desired knowledge or resources. When the product is finished the company get paid, but not before, to finish the product they need to cooperate with others.
To model this scenario we added a bonus $\gamma$, which represents the payoff when a node reach their desired number off established connections, i.e. their maximum node degree($m$). Except from this fact the game is as before, nodes connect to other nodes if they can reach a higher payoff by doing so. 



\subsubsection{Four different scenarios}
To further analyze this model, lets take a closer look on the four different scenarios of the game, a insured node wants to connect to another insured node, insured tries to connect to a non-insured node, non-insured tries to connect to another non-insured node, and non-insured who tries to connect to a insured node.
\subparagraph{Insured to insured}
When an insured node tries to connect to another insured node, the node decision depends on whether he can increase his payoff. 
Let $U_{i}$ denote the payoff of a node with node-degree $i$. When adding a link the payoff the node receives is as follows:
\begin{equation}
    U_{i+1}= 
\begin{cases}
    \alpha + \beta - I_{0} - I_{l},& \text{if } i = 0\\
    U_{i}+\beta -I_{l},& \text{if }  i>0\\
    U_{i}+\beta -I_{l}+\gamma,& \text{if } i=m
    
\end{cases}
\label{eq:itoi}
\end{equation}
For insured nodes to connect to each other, $U_{i+1} > U_{i}$. This model is very similar to the earlier model, but we need to consider the received bonus when reaching the maximum node degree, $m$.
We model this by adding, the bonus divided on the current degree, in the decision process of establishing link or not. 
\begin{eqnarray}
U_{i}+\beta-I_{l}+\frac{\gamma}{m-i}&>U_{i} \nonumber \\ 
\beta-I_{l}+\frac{\gamma}{m-i}&>0 \nonumber \\ 
\llap{$\rightarrow$\hspace{50pt}} \beta+\frac{\gamma}{m-i}&>I_{l} 
\label{eq:itoi2}
\end{eqnarray}
It should be noted that the actual bonus is not added to the current payoff before the node reach the maximum degree, as shown in Eq.(\ref{eq:itoi}). The equation $\frac{\gamma}{ m-i}$ reflects the expected payoff from taking the risk of establishing a connection. This value increases linearly according to how close you are to reach $m$.
In this way the model will change from the former models, because now the nodes have more incentive to connect to other nodes, and in some scenarios they will be willing to take the risk of connecting to a non-insured node in order to reach the expected bonus. The model now introduces a risk factor, because it is not certain that the nodes will obtain enough links, and if not, they will not receive their bonus, however they are stuck with the established connections. 

\subparagraph{Insured connect to non-insured}
The payoff an insured node receives in this scenario is as follows:
\begin{equation}
    U_{i+1}= 
\begin{cases}
    \alpha + \beta - I_{0} - I_{l} -r,& \text{if } i = 0\\
    U_{i}+\beta -I_{l}-r,& \text{if }  i>0\\
    U_{i}+\beta -I_{l}-r+\gamma,& \text{if } i=m
\end{cases}
\label{eq:itonoti}
\end{equation}
To establish a connection from an insured node to a non-insured one, the following has to hold:
\begin{eqnarray}
U_{i}+\beta-I_{l}-r+\frac{\gamma}{m-i}&>U_{i} \nonumber \\ 
\beta-I_{l}-r+\frac{\gamma}{m-i}&>0 \nonumber \\ 
\llap{$\rightarrow$\hspace{50pt}} \beta+\frac{\gamma}{m-i}-r&>I_{l} 
\end{eqnarray}
\subparagraph{Non-insured to non-insured}
When a non-insured node connect to another not-insured node this is the payoff they receive:
\begin{equation}
    U_{i+1}= 
\begin{cases}
    \alpha + \beta -r,& \text{if } i = 0\\
    U_{i}+\beta -r,& \text{if }  i>0\\
    U_{i}+\beta -r +\gamma,& \text{if } i=m
\end{cases}
\label{eq:noitonoti}
\end{equation}
To establish the connection the following equation has to hold:
\begin{eqnarray}
U_{i}+\beta-r+\frac{\gamma}{m-i}&>U_{i} \nonumber \\ 
\beta-r+\frac{\gamma}{m-i}&>0 \nonumber \\ 
\llap{$\rightarrow$\hspace{50pt}} \beta+\frac{\gamma}{m-i}&>r
\end{eqnarray}
\subparagraph{Non-insured to insured}
\begin{equation}
    U_{i+1}= 
\begin{cases}
    \alpha + \beta,& \text{if } i = 0\\
    U_{i}+\beta,& \text{if }  i>0\\
    U_{i}+\beta +\gamma,& \text{if } i=m
\end{cases}
\label{eq:noitoti}
\end{equation}
As we see, this is a strictly increasing function, and thus a non-insured will always connect to an insured node if given the option, and as long as $\beta$ is positive, which it is, given the rules of the model.

\subparagraph{Limitations to ensure a clique of only insured}
We are interested in finding insurable topologies, one candidate is a sub graph consisting of only insured nodes. By analyzing the different scenarios above we can find limitations on the cost of insuring a link, that will force the network formation game to end up in an insurable sub graph.
We know that an insured node would want to connect to another insured node if  Eq.(\ref{eq:itoi2}) is satisfied. 
In the equation we see that the expected bonus per established link is increasing, i.e. if an insured node of degree zero is willing to connect to another insured node, then every node with a degree higher than zero also would like to connect to another insured node. Thus to ensure that insured nodes connect to each other this equation has to hold:
\begin{equation}
\beta+\frac{\gamma}{m}>I_{l}
\label{eq:conditionitoi}
\end{equation}
We also want to ensure that insured nodes never establishes links with non-insured nodes, from \ref{eq:itonoti} we see that this has to hold:
\begin{equation}
\beta+\frac{\gamma}{m-i}-r < I_{l}
\label{eq:conditionitonoti}
\end{equation}
This can be simplified, since we know the insured nodes with degree $m-1$ get the highest expected bonus when establishing a new link, i.e. if we can ensure that nodes with this degree do not establish links to non-insured nodes, we also know that every node with degree less than $m-1$ will not establish links to non-insured nodes. From this we get the equation:
\begin{eqnarray}
\beta+\frac{\gamma}{m-(m-1)}-r&<I_{l} \nonumber\\
\llap{$\rightarrow$\hspace{50pt}} \beta+\gamma-r&<I_{l}
\label{eq:condition-i-to-noti}
\end{eqnarray}
To summarize, Eq.(\ref{eq:conditionitoi}) and Eq.(\ref{eq:conditionitonoti}) gives the final limitation on the link insurance cost:
\begin{equation}
\beta+\gamma-r<I_{l}<\beta+\frac{\gamma}{m}
\label{eq:final-insurance-clique-condition}
\end{equation}
Additionally, in order to make it possible for the game to end up in an insurable topology Eq.(\ref{eq:noinsuredcliqueconditon}) has to be satisfied. As we see from the equation, if the risk to bonus ratio gets to small it gets more and more unlikely to ensure an insurable topology. If we think about a real world scenario where you get a bonus for establishing a fixed number of equations, you would be more willing to take a risk if the possible reward of doing so is large, this is what this equation express. 
\begin{eqnarray}
\gamma-r &<\frac{\gamma}{m}\nonumber \\
1-\frac{r}{\gamma}&<\frac{1}{m} \nonumber \\
\llap{$\rightarrow$\hspace{50pt}}1-\frac{1}{m}&<\frac{r}{\gamma}
\label{eq:noinsuredcliqueconditon}
\end{eqnarray}

\subsection{Simulations}
We simulate how networks form when the conditions from Eq.(\ref{eq:final-insurance-clique-condition}) are met, and what happens when they are not. In addition we use the simulation to look at the consequences with regards to the payoff, when the required number of connections aren't met. 

\subparagraph{Simulation when conditions are met}
First we simulate a network formation game when the link insurance cost satisfied the Eq.(\ref{eq:final-insurance-clique-condition}), using the parameters in table \ref{tbl:maxdegrevalues}. 
\begin{table}[h]
\centering
\begin{tabular}{lc}
 \hline
  $\alpha=10,
  \beta=10,
  I_{o}=5,
  I_{l}=9,
  r=8,
  \gamma=5,
  m=5
  $
  \\
  \hline
\end{tabular}
\caption{Parameters used in the simulation \label{tbl:maxdegrevalues}}
\end{table}
\begin{figure}[h]
\centering
  \includegraphics[width=0.8\linewidth]{../Figures/BonusGameInsuredClique.png}
  \caption{\label{fig:bonusoptimal} Two clustered fully connected networks, created by simulating with the parameters from table \ref{tbl:maxdegrevalues} One consisting of insured agents the other consists of non-insured. }
\end{figure}
As we see in Figure \ref{fig:bonusoptimal} the results where as expected, the cost of insuring a link satisfied the conditions found earlier and thus the result where two cliques, one consisting of only insured and the other of non-insured nodes.

\subparagraph{Simulation when the parameters violates the conditions}
If we change the link insurance cost, so it is just below the limit, $I_{l}=6$, the result is quite different as depicted in Figure \ref{fig:bonusviolating}. Here we see that eventhough non-insured nodes can fail and accumulate negative payoff, some of the insured nodes have taken a risk by connecting to non-insured nodes in order to receive the bonus. 

\begin{figure}[h]
\centering
  \includegraphics[width=0.8\linewidth]{../Figures/BonusGameViolating.png}
  \caption{\label{fig:bonusviolating} Simulation when the cost of insuring a link is just below the limits. }
\end{figure}


\subparagraph{Consequences of not reaching required number of edges}
Both of these scenarios ends up in a situation where the nodes reach their maximum node degree and they all receive the bonus. However each time a node chooses to connect to another node, except when $i=m-1$, it does not know whether it will reach the maximum node degree. Hence the node might take a risk of connecting to other nodes without being able to reach their required number of connections. This means that in certain situation one might end up getting a total $i<m$ connections which results in a much lower payoff than expected. If the variables are set close to a worst-case, such as in table \ref{tbl:maxdegrevalueswitherror} , we end up with two cliques with payoffs close too or worse than the payoff each node had initially. 

\begin{table}[h]
\centering
\begin{tabular}{lc}
 \hline
  $\alpha=10,
  \beta=10,
  I_{o}=5,
  I_{l}=11,
  r=8,
  \gamma=25,
  m=11
  $
  \\
  \hline
\end{tabular}
\caption{Parameters used in simulation \label{tbl:maxdegrevalueswitherror}}
\end{table}


\begin{figure}[h]
\centering
  \includegraphics[width=1.0\linewidth]{../Figures/payoffWhenFulfillingAndViolatingConditionsInMaxDegreeSimulation.png}
  \caption{\label{fig:payoffMaxDegreeSimulation} The different payoffs received when fulfilling and violating the conditions given in \ref{eq:final-insurance-clique-condition}, using the parameters from table \ref{tbl:maxdegrevalues} and \ref{tbl:maxdegrevalueswitherror}. }
\end{figure}

As depicted in Figure \ref{fig:payoffMaxDegreeSimulation}, with the variables from table \ref{tbl:maxdegrevalueswitherror} the resulting payoffs from Figure \ref{fig:bonusviolatingWithErrors} equals $6$ for each of the non-insured nodes and $-1$ for the insured nodes. In comparison, the same figure using the variables from table \ref{tbl:maxdegrevalues} results in a positive payoff $15$ for each of the nodes in the insured clique. This comparison shows the consequences of failing to achieve the required amount of connections, which might be the case if a company is trying to complete a project which requires too many external suppliers. 

\begin{figure}[h]
\centering
  \includegraphics[width=0.6\linewidth]{../Figures/BonusGameInsuredCliqueWithErrors.png}
  \caption{\label{fig:bonusviolatingWithErrors} Simulation when the cost of insuring a link is just below the limits and the maximum node degree is high. }
\end{figure}

\section{Model 4 - Including bulk insurance discount}

Insurance companies often interpret a quantum discount when purchasing multiple products. From convenience stores we are used to the slogan "buy one get one for free". It seems to be common for insurance companies to offer discount to their customers if they choose to collect some or all of their insurances with them. Several insurance companies in Norway, such as Sparebank 1 offers customers up to 25 $\%$ discount according to the following rules \cite{sparebank1}. 

\begin{itemize}

\item 10$\%$ discount if the person has signed three different insurances
\item 15$\%$ discount if the person has signed four different insurances
\item 20$\%$ discount if the person has signed five or more different insurances
\item Plus additional 5$\%$ discount if the person is a customer of the bank. 

\end{itemize}

The insurance offered is intended to the individual market and includes among others: travel insurance, household insurance, car insurance, house insurance, insurance of valuable items and yacht insurance. Since this seems to be the trend for marketing insurance products, it is reasonable to believe that several bonus options would be included in a cyber-insurance product. Since our model so far reflects that a company have to insure each of the connections to other nodes in their network, it is assumed that a similar discount rate following the number of established would be implemented. 

How insurance companies choose to formulate their discount rate might vary. One solution might be to follow a strict 5$\%$ discount per new connection, similar to the one from Sparebank 1, or let the discount follow a power law. However, we choose to follow a discount rule which directly reflects the number of connections the company have established. 
The price for adding a new connection follows the equation:

\begin{equation}
\frac{I_{l}}{i+1}
\label{eq:discount0}
\end{equation}

Here, $i$ is the current number of established connections. This means that the more connections a company acquire the cheaper the connections will be. 
If we add the new rule to the Eq.(\ref{eq:itoi}) which shows the connection between two insured nodes, we get the following equations: 

\begin{equation}
    U_{i+1}= 
\begin{cases}
    \alpha + \beta - I_{0} - I_{l},& \text{if } i = 0\\
    U_{i}+\beta -\frac{I_{l}}{i+1},& \text{if }  i>0\\
    U_{i}+\beta -\frac{I_{l}}{i+1}+\gamma,& \text{if } i=m
    
\end{cases}
\label{eq:discount1}
\end{equation}

As described, for insured nodes to connect to each other, $U_{i+1} > U_{i}$. Building on the already existing model, the quantum discount slightly changes the decision process:
\begin{eqnarray}
U_{i}+\beta-\frac{I_{l}}{i+1}+\frac{\gamma}{m-i}&>U_{i} \nonumber \\ 
\beta-\frac{I_{l}}{i+1}+\frac{\gamma}{m-i}&>0 \nonumber \\ 
\beta (i+1)+\frac{\gamma}{m-i}&>\frac{I_{l}}{i+1} \nonumber \\
\llap{$\rightarrow$\hspace{50pt}}  \beta (i+1)+\frac{\gamma (i+1)}{m-i}&>I_{l}
\label{eq:discount2}
\end{eqnarray}

Similar calculation can be done for the other three scenarios in the game, and they all result in almost the same outcome. First of all results from having quantum discounts on new connections results in a overall higher payoff for the nodes, as long as $i>1$. Since the cost of insuring a new link becomes cheaper. This means that the nodes will have a higher incentive to create links to each other, because the left side of Eq.(\ref{eq:discount2}) yields a higher payoff than before. Which leads to a consequence that more insured nodes could connect to non-insured nodes. 
Building on the final condition \ref{eq:final-insurance-clique-condition} in previous section, we now get:

\begin{eqnarray}
\beta+\gamma-r<\frac{I_{l}}{i+1}<\beta+\frac{\gamma}{m} \nonumber \\
\beta(i+1)+\gamma(i+1)-r(i+1)<I_{l}<\beta(i+1)+\frac{\gamma(i+1)}{m}
\label{eq:final-insurance-clique-condition-with-discount}
\end{eqnarray}


From Eq.(\ref{eq:final-insurance-clique-condition-with-discount}) we see that the initial variable $I_(l)$ has to be priced higher, in order to ensure that only insured nodes connects to other insured nodes. However, beside this drawback, we also experience some positive effects from the modification, since the the purchase of more connections is cheaper the product might be more attractive for potential customers. 
\section{Model 2b: Model with incomplete information}
\label{Model with incomplete information}
An interesting scenario to model is when the nodes lack information about the other nodes type. The way we model this is by letting nature selecting whether a player is insured or not, a node is insured with probability $p$, and not insured with probability $1-p$. 
All nodes know their own type, but in the link establishment process there are only one node who knows the type of the other. The other node only know the probability of the other node being insured or not. 
What we want to find is if it possible for the nodes with incomplete information to distinguish a insured node from a non-insured one. In order to form insurable topologies although we have a scenario with incomplete information. 
\subsection{Analyzis}
When facing a game like this, there exists two types of equilibriums, one where node 2 is able to seperate node 1's type, seperating equilibrium. The another where he can seperate them, pooling equilibrium. 
In this game we have two types of node, type 1 $(t1)$: insured and type 2 $(t2)$: not insured. 
\subparagraph{Node 2 is insured.}
Since every node knows their own type, there are two different games to model, one where node 2 is insured, and the other where he is not insured. We start with the one where he is insured.
Node 1's type is chosen randomly by nature, with probability $p$ of being type 1 and $1-p$ of being type 2.
\begin{figure}[h]
\centering
  \centering
\includegraphics[width=1\linewidth]{../Figures/SignalingGameInsured.png}

\caption{Signalling game with two nodes, node 1's type choosen by nature, node 2 is insured. Node 1 have complete information, node 2 suffer from incomplete information, and act on best response functions based on beliefs. \label{fig:signalingInsured}}

\end{figure}

In the extensive-form shown in Figure \ref{fig:signalingInsured}, we see that $t2's$ strategy L dominates N, and thus $t2$ will never play $N$.
\subparagraph{Separating equilibrium.}
Since node 1 will never play $N$ as type 2, there are only one possible separating equlibrium, type 1 plays $L$ and type 2 plays $N$. Hence node 2's beliefs are as in Eq.(\ref{eq:node2belief}).
\begin{equation}
    \sigma_{1}(t_{i})= 
\begin{cases}
   N,& \text{if } t1\\
   L,& \text{if } t2  
\end{cases}
\label{eq:node2belief}
\end{equation}
Let $\mu_{1}(t_{i} | N )$, denote the probability that node 1 is of type $t_{i}$. By using bayes rule we get this equation:
\begin{equation}
\mu_{1}(t_{1} | N )=\frac{P(N|t_{1})P(t_{1})}{P(N)}=\frac{P(N|t_{1})P(t_{1})}{P(N|t_{1})P(t_{1})+P(N|t_{2})P(t_{2})}
\end{equation}

With node 2's belief, we get that $\mu_{1}(t_{1} | N )=1$ and $\mu_{1}(t_{2} | L )= 1 $. We can now calculate node 2's expected utility from playing L and N:
\begin{eqnarray}
EU_{2}(L,L)=\mu_{1}(t_{1} | L )U_{2}(L,L;t_{1})+\mu_{1}(t_{2} | L )U_{2}(L,L;t_{2}) \nonumber\\
\llap{$\rightarrow$\hspace{50pt}}EU_{2}(L,L)=U_{i}+\beta-I_{l}-r \\
EU_{2}(N,L)=\mu_{1}(t_{1} | L )U_{2}(N,L;t_{1})+\mu_{1}(t_{2} | L )U_{2}(N,L;t_{2})\nonumber\\
\llap{$\rightarrow$\hspace{50pt}}EU_{2}(N,L)=U_{i}
\end{eqnarray}
From these two equations we see that the best response of node 2($BR_2$) when he observes the other node choosing action $L$ is:
\begin{equation}
BR_{2}(L)=
\begin{cases}
L, & \text{if }\beta - r \geq I_{l}\\
N, & \text{if } \beta -r<I_{l}
\end{cases}
\label{eq:insuredBR}
\end{equation}
Node 2's expected utility when type 1 chooses N, is easily seen to be $U_{i}$. 
To confirm if this is a separating equilibrium we must see if node 1 has any incentive to deviate from the strategies in node 2's belief.
Type 2 will never deviate, so lets investigate type 1.
In order to get node 1 to be willing to play N when he knows node 2's best response function, the following must hold: $\beta<I_{l}$. If this is true, then node 2's best response is to play N. I.e. the only separating equilibrium is the following:

\begin{eqnarray}
\beta<I_{l}\\
 \sigma_{1}= 
\begin{cases}
   N,& \text{if } t1\\
   L,& \text{if } t2  
\end{cases}\\
BR_{2}(\sigma_{1})=N
\end{eqnarray} 
This means that in a separating equilibrium, the game will end up with no link establishment.
\subparagraph{Pooling equilibrium.}
In a pooling equilibrium node 2 will not be able to distinguish the two types, and since $t1$'s strategy $L$ dominates $N$, i.e. there is only one possible equilibrium, the one where both types of node 1 plays $L$.
\begin{equation}
    \sigma_{1}(t_{i})= 
\begin{cases}
   L,& \text{if } t1\\
   L,& \text{if } t2  
\end{cases}
\label{eq:node2beliefpooling}
\end{equation}
By using bayes rule we get that $\mu(t_{1}|L)=p$ and $\mu(t_{2}|L)=1-p$.
Node 2's expected utility is then:
\begin{eqnarray}
EU_{2}(L,L)=p(U_{i}+\beta-I_{l})+(1-p)(U_{i}+\beta-I_{l}-r)\nonumber\\
\llap{$\rightarrow$\hspace{50pt}}EU_{2}(L,L)=U_{i}+\beta-I_{l}-r+pr\\
EU_{2}(N,L)=U_{i}
\end{eqnarray}
From this we get node2's best response:
\begin{equation}
BR_{2}(L)=
\begin{cases}
L ,& \text{if } \beta + rp-r\geq I_{l} \\
N ,& \text{if } \beta +rp -r < I_{l} 
\end{cases}
\end{equation}
By using this best response function, node 1 sees that as long as $\beta>I_{l}$ he will never deviate from node 2's beliefs. And it is a pooling equilibrium where both nodes choose $L$, as long as $\beta>I_{l}$ and $\beta +rp-r>I_{l}$.
We also know that: $rp-r\leq0$ is allways true, and thus there also exists a pooling equilibrium where node 1, plays $L$, and node 2, plays $N$. This equilibrium will occur when $\beta>I_{l}$ and $\beta+rp-r<I_{l}$.
\subparagraph{Node 2 not insured.}
Here we will analyze the game when node 2 is not insured.
The rules of the game are as before, the only thing that has changed is the type of node 2, and thus the payoffs are different and we need to see if there exists separating and pooling equilibrium in this game as well.
\begin{figure}[h]
\centering

  \centering
\includegraphics[width=1\linewidth]{../Figures/SignalingGameNotInsured.png}

\caption{Signalling game with two nodes, node 1's type choosen by nature, node2 is not insured. Node 1 have complete information, node 2 suffer from incomplete information, and act on best response functions based on beliefs. \label{fig:signalingNotInsured}}

\end{figure}
\subparagraph{Separating equilibrium.}
In this game there is no dominant strategy for node 1, thus we have to check for the two possible separating equilibriums.
We start with the separating equilibrium with the beliefs shown in Eq.(\ref{eq:node2beliefnotinsured}).
\begin{equation}
    \sigma_{1}(t_{i})= 
\begin{cases}
   L,& \text{if } t1\\
   N,& \text{if } t2  
\end{cases}
\label{eq:node2beliefnotinsured}
\end{equation}
With the beliefs in Eq.(\ref{eq:node2beliefnotinsured}), this is node 2's expected payoffs:
\begin{eqnarray}
EU_{2}(L,L)=(U_{i}+\beta) \\
EU_{2}(N,L)=(U_{i})
\end{eqnarray}
From this we see that his best response when node 1's action is L, is to allways play $L$: \begin{equation}
BR_{2}(L)= L
\end{equation}
To see if this is an equilibrium, we have to see if node 1 has any incentive to deviate. 
We need to check for the two types of node 1:
If $\beta>r$ then type 2 would deviate, because he could achieve a higher payoff by playing $L$, given the beliefs of node 2 in Eq.(\ref{eq:node2beliefnotinsured}). So we know that for this to be an equilibrium, \begin{equation}
\beta < r
\label{eq:sepcondition}
\end{equation}  
When analyzing from node 1 type 1's perspective, for him to play L, this has to hold: $U_{i}+\beta-I_{l}-r > U_{i}$. The only way this can hold is if $\beta>I_{l}+r$. We see that Eq.(\ref{eq:sepcondition}) is violating this condition, and thus we have no separating equilibrium with the beliefs in Eq.(\ref{eq:node2beliefnotinsured}).

Now lets look at the other possible separating equilibrium, see Eq.(\ref{eq:node2beliefnotinsured2}).
\begin{equation}
    \sigma_{1}(t_{i})= 
\begin{cases}
   N,& \text{if } t1\\
   L,& \text{if } t2  
\end{cases}
\label{eq:node2beliefnotinsured2}
\end{equation}
Node 2's expected payoffs are as follows:
\begin{eqnarray}
EU_{2}(L,L)=U_{i}+\beta-r \\
EU_{2}(N,L)=U_{i}
\end{eqnarray}
From this we get the best response function:
\begin{equation}
BR_{2}(L)=
\begin{cases}
L ,& \text{if } \beta\geq r \\
N ,& \text{if } \beta<r 
\end{cases}
\end{equation}
For this to be a separating equilibrium, we need to see if node 1 would deviate from node 2's beliefs. 
Type $t1$ will not deviate as long as $\beta<I_{l}+r$. Type $t2$ will not deviate if $\beta \geq r$, if this condition is true, we see that node 2 will play $L$. I.e. the only separating equilibrium that exists is when node 2 plays $L$, node 1 of type $t1$ plays $N$ and node 1 of type$t2$ plays $L$.
For this to happen we get this condition on $\beta$. \begin{equation}
I_{l}+r>\beta>r
\label{eq:conditionseparatingequilibrium}
\end{equation}
\subparagraph{Pooling equilibrium.}
Two possible, one where both types of node 1 plays $L$, and one where both types plays $N$. Lets first analyze the one where both types of node 1 plays $L$.
\begin{equation}
    \sigma_{1}(t_{i})= 
\begin{cases}
   L,& \text{if } t1\\
   L,& \text{if } t2  
\end{cases}
\label{eq:node2beliefnotinsuredpooling}
\end{equation}
With the beliefs shown above, node 2's expected payoffs are: \begin{eqnarray}
EU_{2}(L)=p(U_{i}+\beta)+(1-p)(U_{i}+\beta-r) \nonumber \\
EU_{2}(L)=U_{i}+\beta-r+pr \\
EU_{2}(N)=U_{i}
\end{eqnarray}
From this we get the best response function :
\begin{equation}
BR_{2}(L)=
\begin{cases}
	L,& \text{if } \beta\geq r-pr\\
   N,& \text{if } \beta<r-pr  
\end{cases}
\end{equation}
Will node 1 deviate knowing this?
Type $t1$ will not deviate as long as: $\beta - I_{l} \geq r$, and type $t2$ will not deviate as long as $\beta >r$.
From this we get this final condition, if $\beta-I_{l}\geq r$ then there exists a pooling equilibrium where both types of node 1 plays $L$ and node 2 also play $L$.
From this we see that the other pooling equilibrium where both types of node 1, plays $N$, will only occur when $\beta<r \text{ and } \beta<I_l+r$.

\subparagraph{Result and findings.}
When one player lack knowledge about the other player, we only found two scenarios where he could separate the two types of the other node. This is possible when player 2 is insured and $\beta<I_{l}$. He can then separate the insured and non-insured types of the other node, because it is only the non-insured node who would want to establish link. Since $\beta<I_{l}$ his best response is to not establish any link.

The other scenario where the node with incomplete information are able to seperate is when he is not insured, and $r<\beta<I_{l}+r$. In this scenario it is only the non-insured node who would want to establish a link, and this is beneficial for both. Thus in this scenario the game will end up with a link between two non-insured nodes.


We where also able to find some pooling equilibriums, if the node with incomplete information is insured, a link will be established if $\beta+rp-r>I_{l}$. However, if $I_{l}<\beta \text{ but } I_{l}>\beta+rp-r$, then the pooling equilibrium will be that node 1 wants to establish link, but node 2 rejects.
A pooling equilibrium where both nodes want to establish a link, occur when node 2 is not insured and $\beta-I_{l}>r$. If $\beta<r$ there will be a pooling equilibrium where both players choose not to establish link. 

What this shows us is that when one player suffer from incomplete information, it is no longer possible for the insurer to force a network to evolve into a clique of only insured nodes. It will also be harder to establish links, because one player must act on beliefs.  




\chapter{Further work on the model}
\label{secondPhaseOfModelingCyberInsurance} 

So far our model have only considered the affects from direct connections. Now we are introducing a way of analyzing how network formation will  be affected from indirect connections. 

In the previous models the utility equation for each node have only been affected by direct variables such as $\beta$. In a real world scenario a node will be strongly affected by network externalities. Our idea is based on the paper from Jackson and Wolinsky \cite{jackson1996strategic} and a network formation game in \cite{jackson2005survey}. 

\section{The connection game}
We consider a connection game which reflects not only the benefit from establishing connections to other people, but also the benefits from indirect connections. Meaning, in addition to the benefit from the direct connection, a node will also benefit from "a friend of a friend", although the benefit will be a factor lower than the direct connection, also "friends of a friend of a friend" will generate benefit and so forth. The payoff will be calculated relative to the distance between different nodes.


\begin{figure}[h]
\centering
  \includegraphics[width=0.2\linewidth]{../Figures/connectionGame.png}
  \caption{\label{fig:connectionGame} Four nodes interconnected with each other.}
\end{figure}
For instance, node 1 in the network depicted in figure \ref{fig:connectionGame}, will benefit $\beta$ from node 2, $\beta^{2}$ from node 3 and $\beta^{3}$ from node 4. The benefit will decrease relative to the shortest path between two nodes as long as $\beta < 1$. Hence the payoff a node receives from the network equals: 

\begin{equation}
\sum_{j\neq i}^{} \beta_{ij}^{d(ij)} - \sum_{j:ij\in g}^{} {I_{l}}_{ij}, 
\label{connecetionGame}
\end{equation}

where $d(ij)$ represents the shortest path between node $i $ and node $j $, and ${I_{l}}_{ij}$ represents node i's cost of insuring a link between the two nodes. To simplify the model we choose a symmetric connection process where $\beta$ and $I_{l}$ is set to a fixed global value. 

To analyze the different outcomes of the game, one might consider making the network as efficient as possible or focus on creating a stable network. An efficient network means ending up with a network which generates the most total value for the players. Intuitively, this network is preferable if it is stable. However, as we shall see there might be some conflict areas between efficiency and stability. 

The paper from \cite{jackson1996strategic} showed that the following propositions for analyzing the game as an efficient network structure:
\begin{enumerate}
\item \textit{a complete graph $g^N$ if $I_{l}<\beta - \beta^2$,}
\item \textit{a star encompassing every node if $\beta - \beta^2 < i_{l} < \beta + \frac{(N-2)}{2}\beta^2$,}
\item \textit{no links if $\beta + \frac{(N-2)}{2}\beta^2 < I_{l}$.}
\end{enumerate}

Generally it means that when the cost of insuring a link is low, it would be more beneficial to have a direct connection to a node than indirectly benefiting from it. When the insurance cost is high, it would not be more beneficial not to establish any connections at all.
The most efficient structure is created in the intermediate cost of insuring links, and ends up in a star structure which encompasses every node. A star structure have the characteristics of minimizing the average path length and uses only a minimal number of links. Indisputable this structure provides the highest overall payoff for the network, however, this network is not stable.
The reason is because the center node of the star will bear the cost of insuring every link connected, which generates a huge cost compared to the other nodes. This scenario is quite unfair, since the center node generates a great deal of network externalities for the other nodes, without being compensated. Hence a number of connections to the center node will not be pairwise stable, and the network will not be stable. 

The conditions has to be changed in order to met the requirements for stability, Jackson and Wolinsky presents the following proposition:

\begin{enumerate}

\item \textit{a pairwise stable network consists of at most one (non-empty) component,}
\item \textit{if $I_{l}<\beta - \beta^2$, the unique pairwise stable network will be a complete graph $g^N$, }
\item \textit{if $\beta - \beta^2 < I_{l} < \beta $, a star encompassing every node will be pairwise stable, although not unique,}
\item \textit{if $\beta < I_{l}$, any pairwise stable network which is nonempty is such that each player has at least two links and thus be inefficient. }
\end{enumerate}

As we can see, the conditions for high and low insurance cost is similar to the previous outcome in the efficient network structure. In addition, the case of intermediate insurance cost will become a stable network, since every new link connected will result in a higher payoff, due to $I_{l} < \beta$. However, it should be noticed that it would be more beneficial for a node to operate at a leaf node in the network, instead of being a center node, due to the cost of insuring each new link. In a perfect star structure, a leaf node will only have to insure the node to the center node, and will benefit indirectly for each node connected to the center node. The center node will benefit from each new connection, however, the payoff will only be $\beta - I{l}$ for each connection. Therefore it is desirable to be a leaf node. 

To solve the problem, one has to compensate the center node for the extra cost of creating network externalities. As described in the previous model considering bulk insurance discount, the insurance company could implement a bonus which lowers the cost for each new connection. This would lower the extra cost for the center node significantly, as it is expected that the number of nodes might be high and therefor result in a significant discount. 
Using the discount calculation from previous models, we end up with the following equation for the star topology:

\begin{equation}
\beta-\beta^2<\frac{i_{l}}{i+1}< \beta
\end{equation}



where $i $ is represent the number of connections a node have established. 

Although this enhancement ensures that the cost for the center node is lowered, it does not guarantee that the center node is fully compensated. To accomplish this, the insurance companies have to directly compensate the center nodes. As described earlier a star topology will create a super critical payoff, and a star topology would be beneficial for the insurance companies. Hence the insurance companies will have incentive to compensate the center nodes additionally to enable the star structure to be generated. 







\
\
......FROM HERE, it's only notes.......
\section{DETTE ET ANNET STED KANSKJE? BLIR LITT RART Å HOPPE INN I DET HER}
By adjusting the parameter one can assure that only insured agents connects to other insured agents, and the opposite,
that only uninsured agents connects to each other. Hence as we can see from the Figure \ref{fig:fincont} clustered
networks of insured agents (red) are created, and  as the paper \cite{contagion} showed, these clustered trusted
networks, can achieve higher, super-critical, payoff by increasing their node degree past the critical point.

\begin{figure}[h]
\centering
\begin{subfigure}{.5\textwidth}
  \centering
  \includegraphics[width=0.8\linewidth]{../Figures/financialContagion1.png}
  \caption{\label{fig:fincont1} Initial graph with 10 agents.}
\end{subfigure}
\quad
\begin{subfigure}{.46\textwidth}
  \centering
  \includegraphics[width=0.8\linewidth]{../Figures/financialContagion2.png}
  \caption{\label{fig:fincont2} Insured agents (red) forms a network}
\end{subfigure}
\caption{\label{fig:fincont} shows how insured agents connects with each other to form a network to achieve super-critical payoffs.}
\end{figure}

, because the nodes can thus receive a super-critical payoff, and they are also insured against contagious risk.  

Figure \ref{fig:GTmodel1equations} presents the individual payoffs in a formation game between two agents in the described model. It is assumed that both agents has to have a desire to establish a connection in order to create a link between them. This is reasonable since a company would not prefer to enter into an agreement with negative expected payoff. As in this case would be the result when an insured agent is requested a connection with someone without insurance. 
 
 
\begin{figure}[h]
\centering
\begin{tabular}{@{}c@{}}
\includegraphics[width=1.0\textwidth]{../Figures/gameTheoryModel1WithEquations.png}
\end{tabular}
\caption[Caption for LOF]{\label{fig:GTmodel1equations} Normal form game between two agents individually choosing to purchase insurance and express desire to connect to the other  \footnotemark }
\end{figure}
\footnotetext{A link will only be created if both agents wishes to establish a connection.}

If we give value to the variables in Figure \ref{fig:GTmodel1equations} one can observe the model's different equilibrium's. It is difficult to know exactly how the variables are set and this would vary considerably between different markets. In a real worlds scenario the variables would also be different for each agent. However in Figure \ref{fig:GTmodel1} we decided to set a fixed value (which is assumed to be corresponding to the real values) for each variable in order to show a concept of how cyber-insurance can be used to create beneficial payoffs.
The following values where used: $\alpha$ = 10, $\beta$ = 10, $I_{o}$ = 5, $I_{l}$ = 2, $r$  = 20, $q$ = 0.5.


\begin{figure}[h]
\centering
\begin{tabular}{@{}c@{}}
\includegraphics[width=0.6\textwidth]{../Figures/gameTheoryModel1WithNumbers.png}
\end{tabular}
\caption{\label{fig:GTmodel1} Shows equilibrium's in the resulting payoff matrix.}
\end{figure}

From the payoff matrix \ref{fig:GTmodel1} we observe two different Nash equilibrium's: One when both agents are insured and wants to connect to the other agent, and one when both are insured but does not want to establish a connection. These are the possible outcomes between the two agents, however as we can see it the social optimal solution would be for two insured agents to connect with each other, i.e they would both receive a significantly higher payoff. This demonstrates that a cluster of insured nodes would achieve higher payoffs.  

