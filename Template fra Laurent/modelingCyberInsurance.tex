\chapter{Modeling Cyber-Insurance }
\label{chp:modelingCyberInsurance} 


\section{Network Formation}

-risk kan spre seg, insentiv for å sørge for at naboene dine er forsikret.
-clustered vs anonymous networks. Krever at man har informasjon om naboene sine, i dette tilfellet om man har forsikring. Fører til at man vil få supercritical payoff. Dette viser at CI kan bringe frem en tilstand i nettverket som øker avkastningen til selskapene i clique'n.
-et eksempel på en modell hvor kun insured vil connecte. +GT tabell 
-oppsummering, hva er det et slikt nettverk kan få igjennom? økt sikkerhet (økonomisk), bedre tillit til samarbeidspartnerne sine, økt payoff. 


In many scenarios agents seeks to create networks in order to directly benefit from each other. The established links might represent companies out sourcing part of their manufacturing, or cooperative agreements in the development of new software products. In addition to increase the trade-off, each of the established links represents risk of being a victim of cascading failures. The intuitive example is the spread of epidemic diseases, also  (node failures of a power grid and) financial contagion such as the one back in 2008 was a result of cascading failures. Strategic network formation using cyber-insurance can be used to prevent such situation in addition to increase the overall payoff of participants in a clustered network.


When deciding whether to establish connection to a neighbor agent, the payoff has to be a balance between the expected earnings and the risk of the other party failing to complete the transaction. This is the reason why we seek to only download content from trusted peers and outlaw MC-gangs are consistently skeptical to enter into new agreements despite promising increased earnings, since the risk of undercover police are too high. 


The paper \cite{contagion} describes a model which seeks to capture the underlying trade-off between the benefits of adding new links and the problem with increased contagious risk. Results from the model describes a situation where clustered graphs achieve a higher payoff when connected to trusted agents. This phenomena is called super-critical payoffs. Unlike in anonymous graphs, which are completely random, nodes in these graphs share some information with their neighbors, which is used when deciding whether to connect or not. The cliques, forms a clustered network of agents which trust each other, consequently the risk of cascading failures are lower.
Inspired by this model, we created a model which shields light on how cyber-insurance can be used in network formation to prevent cascading failures and increase an agents payoff.  

\subsection{Model of handling contagion risk}
The model is simplified in order to show the concept of using cyber-insurance to encounter the problems with contagious risk. 