\chapter{Similar insurance products}
\label{chp:similarInsuranceProducts} 


\section{Discount by collecting insurances}

It seems to be common for insurance companies to offer discount to their customers if they choose to collect some or all of their insurances with them. Several insurance companies in Norway, such as Sparebank 1  offers customers up to 25 $\%$ discount according to the following rules \cite{sparebank1}. 

\begin{itemize}

\item 10$\%$ discount if the person has signed three different insurances
\item 15$\%$ discount if the person has signed four different insurances
\item 20$\%$ discount if the person has signed five or more different insurances
\item Plus additional 5$\%$ discount if the person is a customer of the bank. 

\end{itemize}

The insurance offered is intended to the individual market and includes among others: travel insurance, household insurance, car insurance, house insurance, insurance of valuable items and yacht insurance. Since this seems to be the trend for marketing insurance products, cyber-insurance will also have to offer some kind of discount. As we shall see from our model, a company have to insure each of the connections to other actors in their network, in addition to their own insurance. The reason companies has to insure connections to other companies, is due to the cascading failure effects. If a company is dependent on a resource from another company, and that company goes bankrupt, it will affect the operations of the company. A similar discount rate following the number of established connections might work as incentive for customers to purchase more cyber-insurance, which is believed to be good for both the issuer and buyer. 


\section{Reassuranse}
http://no.wikipedia.org/wiki/Reassuranse

This is something we should look into, from the insurance companies point of view. This will help them to lower the overall risks. 