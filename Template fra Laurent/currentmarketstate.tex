\section{The cyber-insurance market}
The market for cyber-insurance emerged in the late 80's, when security software companies began collaborating with insurance companies to offer insurance policies together with their security products. From a marketing perspective, adding insurance helped highlighting the supposedly high quality of the security software. Nevertheless, this new product was a comprehensive solution, which dealt with both risk reduction and residual risk \cite{bolot2008new}. Continuing into the beginning of the new millennium, several companies started offering standalone cyber-insurance, which sat the frame for the current insurance product. In Norway, startup-companies, such as Safensure AS were established with the goal to deliver cyber-insurance to the Norwegian and European market \cite{digi}. In addition, already well-established insurance companies, such as Gjensidige Nor, started offering insurance products intended for the web-site market. These insurances were created to insure lost income due to malicious hackers, denial of service and other well know cyber-attacks at that time. In 2001 Gjensidige Nor, in cooperation with the German company Tela Versicherung, offered businesses insurance against financial losses due to hacker attacks and sabotage in a range up to 5 million NOK, given that the companies could provide proof that specified security measures were taken \cite{dagensithackerforsikring}. 
 

Despite the fact that cyber-insurance has been around for a couple of decades, the market still struggles to gain a foothold. Safensure AS does not longer exist and Gjenside Nor does not advertise a cyber-insurance product anymore. There seems to be many challenges for both buyers and sellers. Buyers face tremendous confusion about cyber risks and their potential impacts on business. 
The paper \cite{Cyberworkshop} points out that people do not know or understand what kinds of risks the cyber-space involves, and how fatal the losses can be. Even when companies have decided to purchase a cyber-insurance, they are confused with what kind of insurance they should purchase, it is difficult to see what it covers, what is a reasonable price etc. Thus, the market for cyber-insurance tends to become a lemon.s market, where the buyer lacks knowledge, and struggles to see the differences between the different insurance contracts.


\subparagraph{The UK and US markets}
We wanted to reveal the current status of the cyber-insurance market. We limited our survey to the UK and US markets, in addition to the Norwegian market. 
The first impression reveals that there are several different results and opinions regarding the health of the global cyber-insurance market. 
The paper \cite{ccost} studied a sample of 50 organizations in various industry sectors, located in the United States. They showed that on average every company suffered more than one successful attack every week, and argued that successful cyber-attacks could have serious financial consequences. They found that the median cost of cyber-crime in the U.S is \$5.9 million per year, ranging from \$1.5 million to \$36.5 million per company, which is a 56 $\%$ increase from 2010. 
 
 Another paper \cite{evolvingcyber} collected statistics about cyber-attacks in the UK, and claims that the costs are expected to be \pounds 27 billion a year, which makes cyber-crime one of UK's biggest emerging threats. In addition, the paper pointed out that the victims are not only large companies like Google and PlayStation, but also small businesses. Despite these numbers, only 35 $\%$ of the companies in the survey had purchased cyber-insurance. This is surprisingly low, since they found no shortage of providers.It was revealed that there are nine insurers with specialists in cyber-insurance in the UK, and in the US around 30-40 actors.  
 
 
 An article from CFC underwriting \cite{CFCunder}, a UK firm offering insurance to small and medium sized businesses, claims promising numbers for the US cyber-insurance market. On US soil, 20-50$\%$ of the businesses purchased either standalone cyber-insurance or benefits from coverage provided in their existing insurance. However, despite recent years' focus on the increasing cyber-crime activity and the catastrophic consequences of having weak security, only 1$\%$ of European businesses are enrolled in an insurance program covering cyber-threats.
A more optimistic survey, \cite{compworld}, pointed out that more and more insurance companies offer cyber-insurance. Yet, of the 13000 companies, only 46 percent reported that they were insured against the economic consequences of cyber-attacks . 
 
 The media coverage on corporate threats such as Stuxnet and the attacks on Playstation, which lead to a compromise of 77 million user accounts including credit card numbers \cite{playstation}, shows that the cyber-threats are growing, and one would assume that we are in need of cyber-insurance. However, even though the number varies, the surveys show that a large share of companies have chosen not to protect themselves against the residual risk of cyber-attacks, by buying cyber-insurance.

\subparagraph{The Norwegian market}

In comparison, our survey of the Norwegian insurance market revealed that specialized cyber-insurance companies, such as Safensure AS, do not exist anymore. Only one out of the five biggest insurance companies\footnote{Gjensidige, If Skadeforsikring, DNB, TRYG, Storebrand} offers something similar to a cyber-insurance. Gjensidige Nor offers what they call operation-loss-insurance, which covers expenses due to reconstruction of files and reinstalling software and denial of service attacks. In addition, it is also possible to insure against hacking and sabotage \citep{gjensidige}. From email correspondence with Gjensidige Nor it was clear that they needed lots of information regarding the company to be able to calculate the insurance premium. They required extensive information about the economic health of the company, and a model of what kind of software and hardware where used with estimated values on each component. Unfortunately, we were not able to obtain the cost of such an insurance.
  
\subparagraph{Future market}
The survey from \cite{CFCunder} claimed that the US cyber-insurance market was much more mature than the European market. A possible reason is the breach notification laws. In the US, 46 states have mandatory breach notification laws, combined with significant penalties for companies failing to protect sensitive data. This means that the US government is creating incentives for firms to buy cyber-insurance.
 In Europe, only Germany and Austria have similar laws, forcing companies to notify affected customers of data leakage. A recent proposal of the EU wants to introduce the notification law in Europe, and also include penalties for serious data breaches, which could be set as high as 2 $\%$ of a company's global revenue \cite{CFCunder}. It is proposed that the law should take effect in 2014, although this is highly unlikely, considering the complexity of the effects of this law. Undoubtedly such a law would be a healthy injection to the cyber-insurance market. However, a market based on fear of the consequences of not being insured is not desirable. The ultimate goal for cyber-insurance is to correlate the purchase of cyber-insurance with companies growing desire to invest in more security, and hence lower the risk of being a victim of cyber-crimes. 
The article claims that the way to meet this goal, is to focus on the serious brand damage a company will experience and not just on the financial loss. 

In summary, the cyber-insurance market seems to have a huge potential, but needs some new thinking to fully take advantage of it. We will take another approach and focus on finding network structures that will help the insurers offer fair contracts, which is beneficial for both the customers and the suppliers. Hopefully, this can help in the process of establishing a healthy cyber-insurance market.

   
   
   
   
   
   
   
  

