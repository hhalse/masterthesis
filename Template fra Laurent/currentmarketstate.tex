\chapter{Current market}
\label{chp:hvahardenneaasi??} 
\section{Current market state}
Carriers in London, New York, Zurich, Bermuda, Europe, the U.S. and 
elsewhere developing cyber-security insurance products for their clients.
In UK there are 9 insurers with specialists in cyber devisions, 
in the US it is 30-40. \cite{evolvingcyber}
There are lots of challenges both for buyers and sellers. 
Buyers face tremendous confusion about cyber risks and their potential impacts on business. 
People dont know or understand what kinds of risk cyber inlcudes, 
how large losses can be and why should they care about externalities?
\cite{Cyberworkshop} 
Even when companies have decided to purchase a cyber insurance, they are confused of what kind of insurance they should purchase.
The market of cyber insurance becomes a lemons market, where the buyer have little knowledge to choose between the different insurances. 
Therefore, people will buy the cheapest insurance, which probably won't cover the expenses when the incident occurs. 

Despite the widespread awareness of cyber crimes, cyber attacks occur frequently. 
The companies studied in \cite{ccost} experienced successfull every week.
 A successfull cyber attack can result in serious financial consequences. 
 And the longer it takes to resolve the attack, the more costly it get. 
 This paper found that the median cost of cyber crime is \$5.9 million per year, 
 ranging from \$1.5 million to \$36.5 millon per company, 
 which is an 56 percent increase from the last year. This was in the US market only. 
 With these numbers in mind, cyber insurance should be a very attracting security investment. 
 More and more insurance companies offering cyber protection,
  but there are still many companies not utilizing them, in a survey of 13000 companies, 
  46 percent said they had a cyber insurance. \cite{compworld} 

Another paper \cite{evolvingcyber} collected statistics in the UK, which said it costs \pounds 27
 billion a year, and it is one of UKs biggest emerging threats. They found similar results as in US,
  the number of security breaches continue to increase.  It is not only large companies like google
   and playstation that suffer from attacks, but also small businesses. Despite these numbers
    there where only 35 percent of the companies in the survey who purchased cyber insurance. 

A lot of companies are trusting their own IT-department to handle cyber risk, 
and do not think they need a cyber insurance, despite the increasing cyber threats. \cite{twatson}

We conducted a market survey of the Norwegian cyber insurance market. Compared to the US and UK market there are little information about it current state. 
In Norway there are few actors offering any kind of cyber-insurance, in addition they wasn't eager to share any information from their customer base.
Therefore it was not possible to get any estimates on how big the current Norwegian market is. Despite today's low activity, the survey revealed that around year 2000 there was taken steps towards establishing a cyber-insurance market in Norway. Startup companies such as Safensure AS, 
dedicated to deliver cyber-insurance to the Norwegian and European market, and major insurance companies, such as Gjensidige Nor
started offering insurance against loss income due to malicious hacker attacks, denial of service and other characteristics of cyber-insurance. 
In 2001 Gjensidige Nor in cooperation with the German company Tela Versicherung offered businesses insurance against financial losses due to hacker attacks and sabotage for up to 5 million NOK, given that specified security measures are taken by the company (REF: http://www.dagensit.no/arkiv/article1345297.ece). Today, the same company 
offers something they call operation-loss-insurance which covers expenses due to denial of service, software-insurance which covers expences regarding reconstruction of files and reinstalling software, it is also possible to insure against hacking and sabotage. Unfortunately details specifying what's insured and the cost is not known. However, a similar insurance is offered by RTM Insurance Brokers, a Danish company, below is the offered premiums. This gives an indication of the cost of cyber-insurance in the Norwegian market. http://www.hackerforsikring.dk/index.html 






When facing a security breach there are two potential loss classes:
primary losses or first-degree loss: direct loss of information or data and operating loss. 
These arises from unuse, disuse, abuse and misuse of information.
 And the cost of these arise from revovering, loss of revenue, 
 PR and information sharing, hiring of IT-specialists etc. 
Secondary loss is indirectly triggered. Such ass loss of reputation, goodwill, 
consumer confidence, competitive strength, credit rating and customr churn. 
These claims arise from loss of external parties, sensitive data, 
and generally contribute to an even higher cost. \cite{bandyopadhyay2009managers} 

These two loss classes can be covered by cyber-insurance, 
usually are these contract based on the same two classes, i.e you have to get an insurance for both. 
Here is an example contract from \cite{travelers}.
\section{Contract structure}
Travelers cyber insurance:
\begin{itemize}
\item Liability insurance. \begin{enumerate}
\item Network and Information Security Liability
\item communications and Media Liability
\item Regulatory Defense Expenses
\end{enumerate}
\item First party insuring agreements: \begin{enumerate}
\item Crisis managment event expenses
\item Security breach remediation and notification expenses
\item computer program and electronic data restoration expenses
\item computer fraud
\item fund transfer fraud
\item e-commerce extortion
\item business interruption and additional expenses
\end{enumerate}
\end{itemize}
\section{Economics}
Traditional security is a public good and are usually provided by the government. 
The threats are also originating from a small number of actors. 
What about internet security, should it be handled by the government.
 We do not have anti-tank gear in every house, should we have anti virus software on every computer?
 there are strong externalities involved, if a unsecured computer joins the internet, 
 it end up dumping costs on others, just like pollution. 
 Lemons problem, antivirus software. because the customer cant see the difference.
 Asymmetric information explains many market failures, low prices in lemons-markets, why sick people struggle with getting to buy insurance.
 A good example of misaligned incentives is bank frauds in US and UK, in US the banks are the ones hold responsible, in UK it is the customers. One would think the banks in UK was better off, but they are not. Similar problems can be found in other systems, and the problem is  security failing because the people guarding a system are not the poeple suffering the costs of failure. 
 \section{Epidemics}
 \cite{easley2012networks}
 The social network within a population, has a big say in determining how diseases is likely to
  spread. it can only spread if there are contact between to persons(Nodes), the contact network.
  The contact network for to different diseases can differ radically, e.g java viruses versus worm
   propagating through another vulnerability. Or internet viruses versus viruses that spread through short-range wireless communication. 
   \subsection{modeling contagion}
   \subparagraph{branching processes}
   first wave, a person carrying a new disease enters a network, and transmits to everyone he meets with a probability of p, he meets k-people.
second wave, each person from the first wave now meets k new people, i.e a total of k times k and if infected passes the disease on with probability p.
further waves are formed in the same way.
With this simple modeling approach, we get a tree, with a root node which creates branches to new
lvls of the tree. With low contagion probability, the infection is likely to die out quickly. 
If the disease in a branching process ever reaches a wave where it fails to infect anyone, then it has died out. It is only two possibilities for the disease in a branching model, 
either it dies out, or it continue to infect infinitely  many waves. These two possibilities can be
 differentiaded by a quantity called the basic reproductive number.  $R_{0}$, this is the expected
  number of new cases of the disease caused by one person/node. In this basic model this number is:
   $p*k$. If $R_{0}<1$ then with probability 1 the disease dies out after a finite number of waves, if $R_{0}>1$ then it continues to infect atleast one person each wave with a probability greater than 0.
   A interesting thing to notice about these statements, is if the $R_{0}$ is close to 1 in 
   either way, then a small shift in the probability will change the 
   disease status from terminating to widespread or visa versa. This suggests that around the critical value $R_{0}=1 $ it can be worht investing large amounts of effort to produce small shifts in R. 
\subparagraph{SIR epidemic model}
Can be applied to any network structure, preserve the basics of the branching process at the level of individual nodes, but generalize the contact structure. 
A node goes through three potential stages: \begin{enumerate}
\item Susceptible(S): Before the node has caught the disease.
\item Infectious(I): once the node has caught the disease, it is infectious and can infect other susceptible neighbors with probability p.
\item Removed(R): After a node has experienced the full infetious period, it is removed from consideration, since it no longer poses a threat.
\end{enumerate}
Network with directed edges. The progress of the epidemic is ontrolled by the contact network
 structure, probability of contagion and $t_{I}$ the length of infection.
When a node enters the I state, it remains infectious for a fixed number of steps $t_{I}$. 
During each of these steps it has a probability of infecting its neighbours. 
After $t_{I}$ it is removed(R). Good model for disease you can only catch once in a lifetime. 
Important to note that in networks that do not have tree structure, the claim made earlier about $1>R_{0}>1$ does not necesarily hold anymore. The network structure is very important, 
it can decide if a disease will spread or not.  Narrow channel example. 
\subparagraph{Extension to SIR}
The SIR model is simple, to make it more realistic we can add probability q of recovery, and also add different probabilities for contamination between nodes, due to stronger contact. We add periods to the infection time, early, middle and late and allow different probabilities for infecting in each of these states. 

\subparagraph{Model from dynamic to static(Percolation)}
assing a probability of infecting on every edge, calculate this at the beginning, and thus an infected
 node has to connected to another infected node by an open edge. Think of it as fluid running through open and closed pipes. Its only the open ones who can be affected. 
 
 \subparagraph{SIS epidemic model}
 Nodes can be reinfected. Only two states, susceptible and infectious. Researchers have proved "knife-edge" results on these networks as well.
 A SIS epidemic can be represented by a SIR model by using a "time-expanded" network. Duplicate the nodes to the next time-frame.
 \subparagraph{SIRS Epidemic model}
 Remain removed(immune for a fixed period of time) $t_{R}$, this model fits good with many real 
 world diseases. It can produce oscillations in  very localized parts of the network, with patches of
  immunity following large numbers of infections in small areas. 

\section{Incentives and Information Security}


People have realized that security failure is not only caused by technical mistakes but also misaligned incentives. When the person guarding them is not the one who suffers when the system fail, there are strong misaligned incentives. As the book \cite{anderson2010security} states, the tools and concepts of game theory and microeconomic theory are becoming just as important as the mathematics of cryptography.
\subparagraph{Informational asymmetries}
peer-to-peer network, these exploit network externalities to the fullest by having large member populations with a flat topology. Joining creates the possibility of collaboration with everyone. 
it is easy to cheat.One solution, change the network topology, create clubs of nodes, 
one need to establish trust with the club, then you can connect with outside groups through
 your group. Social networks can also be used to create better topologies, when honest players 
 can select their friends as neghbors., they minimize the information asymmetry present during
  neighbor interactions. 
Another information asymmetry in security, is due to our inability to measure software security. 
Network science and information security, the network topology can strongly influence conflict dynamics.
Externalities makes security problems reminiscent of environmental pollution, public goods. 


   
    
   
   
   
   
   
   
   
   
   
   
   
   
  

