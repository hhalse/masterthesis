\chapter{Current market}
\label{chp:hvahardenneaasi??} 
\section{Current market state}
Several companies in London, New York, Zurich, Bermuda, Europe, the U.S. and 
elsewhere offer cyber-security insurance products for their clients.
In UK there are 9 insurers with specialists in cyber-insurance, 
in the US it is 30-40 \cite{evolvingcyber}.


A survey of the Norwegian insurance market relieved that only one out of the five biggest actors \footnote{Gjensidige, If Skadeforsikring, DNB, TRYG, Storebrand} where even considering to offer something similar to cyber-insurance. From mail correspondence with Gjensidige it was clear that normally this was a typical risk they would like to insure, however extensive information about both what software and hardware where used. In dept, the requested information was related to a companies revenue from a website, and a model describing the architecture of the server and it's value. 
[Email from: Arild Hjelde, Gjennsidige Nor.]


However, there are lots of challenges both for buyers and sellers. 
Buyers face tremendous confusion about cyber risks and their potential impacts on business. 
People dont know or understand what kinds of risk the cyber space inlcude, 
how large losses can be and why should they care about externalities \cite{Cyberworkshop}?
Even when companies have decided to purchase a cyber-insurance, they are confused with what kind of insurance they should purchase.
The market of cyber-insurance becomes a lemons market, where the buyer have little knowledge to choose between the different insurances. 
Hence, people will buy the cheapest insurance, although it would probably not cover the expenses when the incident occurs. 

At the same time, people are notified with the increasing occurrence of cyber crimes and cyber attacks.
The companies studied in \cite{ccost} experienced successful attacks every week.
 A successful cyber attack can result in serious financial consequences. 
 And the longer it takes to resolve the attack, the more costly it get. 
 This paper found that the median cost of cyber crime int the U.S is \$5.9 million per year, 
 ranging from \$1.5 million to \$36.5 million per company, 
 which is an 56 percent increase from last year. This was in the US market only. 
 With these numbers in mind, cyber-insurance should be a very attractive security investment. 
 More and more insurance companies are offering cyber protection,
  but there are still many companies not utilizing them, in a survey of 13000 companies, 
  only 46 percent said they had some kind of cyber-insurance \cite{compworld}. 

Another paper \cite{evolvingcyber} collected statistics about cyber attacks in the UK, and the results said it costs \pounds 27
 billion a year, and it is one of UKs biggest emerging threats. They found similar results as in US,
  the number of security breaches continue to increase, and it is not only large companies like Google
   and Playstation that suffer from attacks, but also small businesses. Despite these numbers
    there where only 35 percent of the companies in the survey who had purchased cyber-insurance. 

A lot of companies are trusting their own IT-department to handle cyber risk, 
and do not think they need a cyber-insurance, despite the increasing cyber threats. \cite{twatson}


When comparing the norwegian cyber-insurance market up against the US and UK, 
there is little information available about its current state. We did a survey and contacted a number of Norwegian insurance companies. Although the majority of the companies did not offer any kind of cyber-insurance, the few who did where not eager to share information about the current market, such as, number of customers, typical customer i.e. small/big companies etc.  
Despite today's low activity, the survey revealed that around year 2000 there were taken steps towards
 establishing a cyber-insurance market in Norway. Several startup companies, such as Safensure AS \cite{digi},
  where created with the idea of delivering cyber-insurance to the Norwegian and European market. Big actors such as Gjensidige Nor, where also offering an insurance aimed for web-sites. 
The first insurances where created to ensure lost income due to malicious hacker attacks, denial of service and other well know cyber-attacks. 
In 2001 Gjensidige Nor in cooperation with the German company Tela Versicherung offered businesses
 insurance against financial losses due to hacker attacks and sabotage for up to 5 million NOK, given that specified security measures were taken by the company \cite{dagensithackerforsikring}. 
 Today, the same company 
offers something they call operation-loss-insurance which covers expenses due to denial of service,
 software-insurance which covers expenses regarding reconstruction of files and reinstalling software.
 In addition, it is also possible to insure against hacking and sabotage \citep{gjensidige}. 
 Unfortunately details specifying what aspects of the business is insured and the cost is not known. However,
  a similar insurance is offered by RTM Insurance Brokers, a Danish company, with premiums ranging from 3400DKK for insuring a loss up to 2.5 million DKK, to 12900DKK for insuring a loss valuated to 25 million DKK \cite{RTM}. This gives an indication of the cost of the current cyber-insurance in the Norwegian market. 

There are several different opinions regarding the health of the global cyber-insurance market. 
An article from CFC underwriting \cite{CFCunder}, a UK firm offering insurance to small 
and medium sized business, claims promising numbers for the US cyber-insurance market. 
On US soil, 20-50$\%$ of businesses purchases either stand alone cyber-insurance or benefits from
 coverage provided in their already exciting insurance. However, despite recent years focus on the increasing
  cyber-crime activity and the catastrophic consequences of having weak security, 
  only 1$\%$ of European businesses are enrolled in an insurance program covering cyber-risks. 
One possible reason could be the different environments of the US and European market. 
In the US, 46 states have mandatory breach notification laws, combined with significant penalties 
for companies failing to protect sensitive data. This means that the US government are creating incentives for firms to buy cyber-insurance.
 In Europe, only Germany and Austria have similar breach notification laws,
 forcing companies to notify affected customers of data leakage. A recent proposal of the EU wants to introduce the notification law in Europe, and also include penalties for serious data breaches, these could be as high as 2 $\%$ of a companies global revenue \cite{CFCunder}. It is proposed that the law should take effect in 2014, although this is highly unlikely regarding the complexity of the effects of this law. Undoubtedly this law would be a health injection to the rise of the cyber-insurance market, however, a market based on fear of the consequences of not being insured is not desirable. The ultimate goal for cyber-insurance, is to correlate the purchase of cyber-insurance with companies growing desire to invest in more security, and hence lower the risk of being a victim of cyber-crimes. 
The article claims that the way to meet this goal, is to the focus on the serious brand damage a company will experience and not just the financial loss. 


   
   
   
   
   
   
   
   
   
   
   
   
   
   
  

