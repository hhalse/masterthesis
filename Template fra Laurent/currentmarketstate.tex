\chapter{Current market}
\label{chp:hvahardenneaasi??} 
\section{Current market state}
Carriers in London, New York, Zurich, Bermuda, Europe, the U.S. and 
elsewhere developing cyber-security insurance products for their clients.
In UK there are 9 insurers with specialists in cyber devisions, 
in the US it is 30-40. \cite{evolvingcyber}
There are lots of challenges both for buyers and sellers. 
Buyers face tremendous confusion about cyber risks and their potential impacts on business. 
People dont know or understand what kinds of risk cyber inlcudes, 
how large losses can be and why should they care about externalities?
\cite{Cyberworkshop}

Despite the widespread awareness of cyber crimes, cyber attacks occur frequently. 
The companies studied in \cite{ccost} experienced successfull every week.
 A successfull cyber attack can result in serious financial consequences. 
 And the longer it takes to resolve the attack, the more costly it get. 
 This paper found that the median cost of cyber crime is \$5.9 million per year, 
 ranging from \$1.5 million to \$36.5 millon per company, 
 which is an 56 percent increase from the last year. This was in the US market only. 
 With these numbers in mind, cyber insurance should be a very attracting security investment. 
 More and more insurance companies offering cyber protection,
  but there are still many companies not utilizing them, in a survey of 13000 companies, 
  46 percent said they had a cyber insurance. \cite{compworld} 

Another paper \cite{evolvingcyber} collected statistics in the UK, which said it costs \pounds 27
 billion a year, and it is one of UKs biggest emerging threats. They found similar results as in US,
  the number of security breaches continue to increase.  It is not only large companies like google
   and playstation that suffer from attacks, but also small businesses. Despite these numbers
    there where only 35 percent of the companies in the survey who purchased cyber insurance. 

A lot of companies are trusting their own IT-department to handle cyber risk, 
and do not think they need a cyber insurance, despite the increasing cyber threats. \cite{twatson}

When facing a security breach there are two potential loss classes:
primary losses or first-degree loss: direct loss of information or data and operating loss. 
These arises from unuse, disuse, abuse and misuse of information.
 And the cost of these arise from revovering, loss of revenue, 
 PR and information sharing, hiring of IT-specialists etc. 
Secondary loss is indirectly triggered. Such ass loss of reputation, goodwill, 
consumer confidence, competitive strength, credit rating and customr churn. 
These claims arise from loss of external parties, sensitive data, 
and generally contribute to an even higher cost. \cite{bandyopadhyay2009managers} 
These two loss classes can be covered by cyber-insurance, 
usually are these contract based on the same two classes, i.e you have to get an insurance for both. 
Here is an example contract from \cite{travelers}.
\section{Contract structure}
Travelers cyber insurance:
\begin{itemize}
\item Liability insurance. \begin{enumerate}
\item Network and Information Security Liability
\item communications and Media Liability
\item Regulatory Defense Expenses
\end{enumerate}
\item First party insuring agreements: \begin{enumerate}
\item Crisis managment event expenses
\item Security breach remediation and notification expenses
\item computer program and electronic data restoration expenses
\item computer fraud
\item fund transfer fraud
\item e-commerce extortion
\item business interruption and additional expenses
\end{enumerate}
\end{itemize}
\subparagraph{How to measure damage?}

