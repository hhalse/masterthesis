\chapter{The cyber-insurance market}
\label{chp:hvahardenneaasi??} 



The market for cyber-insurance emerged in the late 90's when security software companies partnered with insurance companies and started offering insurance policies together with their security products. From a marketing perspective, adding the insurance helped highlighting the supposedly high quality of the security software. Regardless, the new product was a comprehensive solution, which dealt with both risk reduction and residual risk \cite{bolot2008new}. Continuing into the beginning of the new millennium, several companies started offering standalone cyber-insurance, which sat the frame for the current insurance product. In Norway, starup companies, such as Safensure AS where established with respect to deliver cyber-insurance to the Norwegian and European market \cite{digi}. Also established insurance companies such as Gjensidige Nor, started offering insurance products aimed for Internet web-sites. These insurances where created to insure lost income due to malicious hacker attacks, denial of service and other well know cyber-attacks at that time. E.g. in 2001 Gjensidige Nor in cooperation with the German company Tela Versicherung offered businesses insurance against financial losses due to hacker attacks and sabotage for up to 5 million NOK, given that specified security measures were taken by the company \cite{dagensithackerforsikring}. 
 
\section{Current market state}
Despite the fact that cyber-insurance has been around for over a decade, the market still struggles to gain a foothold. Safensure AS does not exist anymore and Gjenside Nor does not advertise a cyber-insurance product. It seems to be lots of challenges for both buyers and sellers. Buyers face tremendous confusion about cyber risks and their potential impacts on business. 
In general, \cite{Cyberworkshop} points out that people do not know or understand what kinds of risks the cyber space involves, 
and how large the losses can be especially due to network externalities.
Even when companies have decided to purchase a cyber-insurance, they are confused with what kind of insurance they should purchase.
The market of cyber-insurance tend to become a lemons market, where the buyer have little knowledge to choose between the different insurances. 
Hence, people will buy the cheapest insurance, although it sometimes does not satisfies their requirement. 

\subsection{The UK and US market}
The media coverage on corporate threats such as Stuxnet\footnote{Stuxnet, SKAL VI BESKRIVE HVA DET ER? ?? } and the attacks on Playstation, which lead to a compromise of 77 million user accounts including credit card numbers \cite{playstation}, shows that the cyber-threats is growing. 
There are several different results and opinions regarding the health of the global cyber-insurance market. 
Companies studied in \cite{ccost} experienced successful attacks every week, and showed that successful cyber attacks could result in serious financial consequences. They found that the median cost of cyber crime in the U.S is \$5.9 million per year, 
 ranging from \$1.5 million to \$36.5 million per company, which is an 56 percent increase from last year. 
 
 Another paper \cite{evolvingcyber} collected statistics about cyber attacks in the UK, and the result claims that the costs is expected to be \pounds 27 billion a year, and that it is one of UKs biggest emerging threats. In addition, they pointed out that the victims is not only large companies like Google and Playstation, but also small businesses. Despite these numbers only 35 $\%$ of the companies in the survey had purchased cyber-insurance. Although there is no shortage of providers,-they found that there are 9 insurers with specialists in cyber-insurance in the UK, and in the US around 30-40 actors.  
 
 
 An article from CFC underwriting \cite{CFCunder}, a UK firm offering insurance to small 
and medium sized businesses, claims promising numbers for the US cyber-insurance market. 
On US soil, 20-50$\%$ of businesses purchased either standalone cyber-insurance or benefits from
 coverage provided in their already exciting insurance. However, despite recent years focus on the increasing
  cyber-crime activity and the catastrophic consequences of having weak security, 
  only 1$\%$ of European businesses are enrolled in an insurance program covering cyber-threats.
 A more optimistic survey pointed out that more and more insurance companies offered cyber-insurance. Yet, of the 13000 companies, only 46 percent said they where insured against cyber-attacks \cite{compworld}. 
 
 The numbers vary between the different surveys. However, all of them concludes that a large share of the companies are not protected against the residual risk of cyber attacks. 
    
    
\subsection{The Norwegian market}

 In comparison, our survey of the Norwegian insurance market relieved that specialized cyber-insurance companies such as Safensure AS does not exist anymore. Additionally, only one out of the five biggest actors\footnote{Gjensidige, If Skadeforsikring, DNB, TRYG, Storebrand} offer something similar to a cyber-insurance. Gjensidige Nor offers something they call operation-loss-insurance which covers expenses due to reconstruction of files and reinstalling software and denial of service attacks. In addition, it is also possible to insure against hacking and sabotage \citep{gjensidige}. From mail correspondence with Gjensidige Nor it was clear that they needed information to be able to calculate the insurance premium. They required extensive information about the economic health of the company, and a model of what kind of software and hardware where used with estimated values on each component. [Email from: Arild Hjelde, Gjennsidige Nor.] Unfortunately we were not able to obtain the cost of such insurance. However, a similar insurance is offered by RTM Insurance Brokers, a Danish company, with premiums ranging from DKK3400 for insuring a loss up to DKK2.5 million , to DKK12900 for insuring a loss valuated to DKK25 million  \cite{RTM}. This gives an indication of the cost of the current cyber-insurance in the Norwegian market. 
  

\section{Future market}
The survey from \cite{CFCunder} claimed that the US cyber-insurance market was much more mature compared to the European. A possible reason is the breach notification laws. In the US, 46 states have mandatory breach notification laws, combined with significant penalties for companies failing to protect sensitive data. This means that the US government are creating incentives for firms to buy cyber-insurance.
 In Europe, only Germany and Austria have similar laws, forcing companies to notify affected customers of data leakage. A recent proposal of the EU wants to introduce the notification law in Europe, and also include penalties for serious data breaches, these could be as high as 2 $\%$ of a companies global revenue \cite{CFCunder}. It is proposed that the law should take effect in 2014, although this is highly unlikely regarding the complexity of the effects of this law. Undoubtedly this law would be a health injection to the rise of the cyber-insurance market, however, a market based on fear of the consequences of not being insured is not desirable. The ultimate goal for cyber-insurance, is to correlate the purchase of cyber-insurance with companies growing desire to invest in more security, and hence lower the risk of being a victim of cyber-crimes. 
The article claims that the way to meet this goal, is to the focus on the serious brand damage a company will experience and not just the financial loss. 


\subparagraph{notes...}
   One reason for why the number of insured companies are low could be the fact that a lot of companies are trusting their own IT-department to handle cyber risk. Hence they believe that they would not need a cyber-insurance \cite{twatson}. 
   
   
   
   
   
   
   
   
   
   
   
   
   
  

