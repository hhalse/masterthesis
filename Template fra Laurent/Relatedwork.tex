\chapter{Relatedwork}
\label{chp:relatedwork} 



\begin{figure}
\centering
% dummy figure replacement 
\begin{tabular}{@{}c@{}}
\rule{.5\textwidth}{.5\textwidth} \\
\end{tabular}
\caption{\label{fig:example}A figure}
\end{figure}

\section{Towards Insurable Network Architectures}\label{sec:first_section}
A trusted component or system is one you can insure.
Cyber insurance gives an incentive to better secure your network, and will thus reduce the overall threat for both first and third parties. It will also promote gathering and sharing of information related to security incidents. All in all this will increase the social welfare by decreasing the variance of losses. 
But even if cyber insurance seems very profitable for everyone, it has failed to evolve as much as expected.
Some reasons for this, could be:
\begin{itemize}[topsep=-1em,parsep=0em,itemsep=0em] 
 \item lack of data to calculate premium. \item Underdeveloped demand due to missing awareness for cyber risks. \item legal and procedural hurdles in substaining claim.
\end{itemize}
A more economic model to describe why cyber insurance is still such a niche market.
\subparagraph{Interdependent security}
Expected loss due to security breach at one agent is not only dependent on this agents lvl of security, but also by other agents security investment. A good example is spam, it is dependent of number of compromised computers. This also generates an externality and encourages to free riding. which then leads to underinvestment in security.
\subparagraph{Correlated risks}
Many systems share common vulnerabilities, which can be exploited at the same time. This leads to a more likely occurrence of extreme and catastrophic events, which will result in uneconomical supply of cyber insurance.
\subparagraph{information asymmetry}
Since measuring security strength is very hard, people have a high incentive for hiding info. This leads to information asymmetry. 
All these three form a triple obstacle, which eliminates the market in evolving. All these obstacles evolve from what makes ICT succeed, distribution, interconnection, universality and reuse.
This is why Architecture matters. The obstacles does not arrive from properties of individual agents, but from integration and interaction in networking. Networking is not just physical, but a abstract structure mapping physical, logical and social interconnection. A good example is development tool chains. A web-browser is not just dependent of the security the developers have implemented, but also the security in the tools used, such as libraries.
These three problems have never been analysed together, this is what this paper contributes with. Uses GT to model incentives of the different agents. 
\section{Cyber insurance as an Incentive for Internet Security}\label{sec:first_section}
so far the risk management on the internet has involved methods to reduce the
risks(firewalls, ids, prevention etc.) but not eliminate risk. Is it logical to buy
insurance to protect the internet and its users.
An important thing to notice when insuring internet, 
is that the entities on the internet are correlated, which means insurance claims will likely be correlated. Risks are interdependent, decision by an entity to invest in security affects the risks of others.
Key result: using insurance would increase the overall security. Act as an powerful incentive, which pushes entities over the threshold where they invest in self-protection. Insurance should be an important component of risk management in the internet.

Four typical options available in the face of risks. 
1. avoid the risk 2. retain 3. self protect and mitigate 4. transfer the risk.
Most entities in the internet have choses a mix of 2 and 3. This has lead to 
lots of systems trying to detect threats and anomalies(both malicious and accidental) and to protect the users and the structure from these.
but this does not eliminate risk, threats evolve over time and there is allways accidents.
How to handle this residual risk?
Option 4, transfer the risk to another entity who willingly accept it(hedging), insure in exchange for a fee. Allows for predictable payouts for uncertain events.
But does this makes sense for the internet, benefits, to whom? and to what extent?

How to model insurance and computing premiums. avoid ruin the insurer. Actuarial approach. Economic approach:  premium should be negative correlated to the amount
invested in security by the entity. Users can chose to invest c or not in security
solutions. Shown that in the 2 user case in absence of insurance, there is a NE
in a good state, if c is low enough. These result have been extended to a
network setting. This paper starts out by adding insurance to the two person game, then the n-users network, where damages spread among the users. They show that if premium discriminates about investment in protection. Insurance is a strong incentive to invest in security. Also show how insurance can be a mechanism to facilitate the deployment of security investments by taking advantage network effects such as treshold or tipping point dynamics. Uses simple models. 

Using cyber insurance as a way to handle residual risk started out early in the 90's. Software and insurance sold as packages.  More recently insurance companies started offering standalone products. A challenging problem is the correlation between risks, interdependent risks(risk that depend on the behavior of others).

\subsection{Classical model for insurance}
agents try to maximize some kind of expected utility function, and are risk averse.
\begin{math} u[w_{0}-\pi}]=E[u[w_{0}+X]] \end{math}
Investments for an agent is either self protect and or insurance.
If insurance premium is not negatively correlated to the self protection, we get moral hazard. Because if not, insurance will disincourage self protection.
In this way insurance can co-exist with selfprotection.
\subsection{Interdependent security and insurance}
In presence of interdependent risks, the reward for a user investing in self-protection depends on the security in the rest of the network.
Discrete choice, invest or not. loss occurs directly or indirectly. Cost of investing is c. This avoids the direct loss completely. 
In summary, insurance provides incentives for a
small fraction of the population to invest in self-protection, which in turn induces the rest
of the population to invest in self-protection as well, leading to the desirable state where
all users in the network are self-protected. Furthermore, the parameter y provides a way
to multiply the benefits of insurance, by lowering the initial fraction of the self-protected
population needed to reach the desirable state.
This paper shows that insurance provides significant benefits to network of users facing correlated, interdependent risks. Insurance is a powerful mechanism to promote network-wide changes, i.e lead to self protection.
How to estimate damage? This is very hard on the internet. 
This paper shows how it is economical rational for entities to prefer a relatively insecure system to a more secure, and that the adoption of security investments follows treshold/tipping point dynamics. And that insurance is a powerful incentive to push the users over the treshold.
\section{Modeling cyber-insurance: towards a unifying Framework}
Cyberinsurance, the transfer of financial risk associated with network and computers incidents to a third party, has been researched for several years. But reality continues to disappoint. Sets back by physical accidents such as 9/11 Y2K etc. Clients are for the most SMBs. 
All three obstacles has to be overcomed at the same time to fix the market, to do this we need a comprehensiv framework for modelling cyber-risk and cyber-insurance.
many researchers have lost their optimism about cyberinsurance, but this paper has not.
Breaks the modeling down to five key components: 
\begin{itemize}[topsep=-1em,parsep=0em,itemsep=0em] 
 \item network environment(nodes controlled by agents, who extract utility. The risk comes from here \item demand side(agents) \item supply side(insurers) \item information structure, distribution of knowledge among the players. \item organizational environment. public and private entities whose actions affect netwrok security and agents security decisions.
 
\end{itemize}
what can be answered with models of cyber insurance markets?

Looking at equilibrium we can determine under which conditions will a market for cyber-insurance thrive? or what are the reasons for failure.

What is the effect of an insurance market on aggregate nework security? Will the internet become more secure?

What are the contributions to social welfare?

\subparagraph{function}
Defense function \begin{math} D \end{math} describes how security investment affects the probability of loss \begin{math}p \end{math} and the size of the loss  \begin{math}l \end{math} for individual nodes.  In most general its a probability distribution.
An agent \textit{i} only chooses s_i.

\subparagraph{network topology G}
Describes the relation between elements of an ordered set of nodes.( connectivity)
\begin{itemize}[topsep=-1em,parsep=0em,itemsep=0em] 
\item star-shaped \item tree shaped \item ER \item Structured clusters
\end{itemize}
\subparagraph{Risk arrival}
defined by the relation between network topology \textit{G} and the value of the defense function \textit{D}
Cyber risk is characterized by both interdependent security and correlated risk.
\subparagraph{Attacker model}
existing literature assume attacks are perfomed by "nature" rather than strategic players. But attackers react to agents and insurers decisions. this paper models attackers as players. 
\subsection{Demand side agents}
Make security decisions for one or more nodes. When buying full coverage of risks, permits the agent to exchange uncertain future costs with a predictable premium. 
Agents have node control, they are either heterogen og homogeneous. They only seek insurance if they are risk averse. 
Full or partial insurance. 
A controct is called fair if the expected profit from it is zero(insurers point of view).

Selfprotection creates an externality, ie interdependent security.
Selfinsurance does not generate externality, it only reduces your size of potential loss. 

time, all choices are set only once, but not necessarily at the same time by all the agents. 
\subsection{Supply side, insurers}
It is important to include these as players. Five attributes: market structure, risk aversion, markup, contract design and higherorder risk transfer.
Marketstructure, monopoly, oligopoly or competition. Homegenous or heterogenous? Competitions leads to low MC. Risk aversion to avoid taking excessive risk and bankruptcy, need a safety capital. 
Markup: insurers profit, admin-costs, cost of safety capital. 
Contract design: fixed premium, premium differentiation, contract with fines.
Higher order risk transfer: 
Cyber-reinsurance, catastrophe bonds, exploit derivatives.
\subsection{Information structure}
Symmetric and asymmetric. Leads to adverse selection if the insurer cant distinguish between the agents. Moral hazzard occurs if agents could undertake actions that affect the probability of loss ex post. Also information about security is hard to gather and evaluate,. . All this results in two types of contract scenario, pooling or separation(agents sort them self out).


 
\subsection{First subsection with some \texorpdfstring{$\mathcal{M}ath$}{Math} symbol}\label{sec:first_ssection}
\begin{math} w_{0} \lim_{x \to \infty} \exp(-x) = 0 \alpha, \Alpha, \beta, \Beta, \gamma, \Gamma,
  \pi, \Pi, \phi, \varphi, \Phi  \pi 
  X E[X]=0 \end{math}
\blindtext



