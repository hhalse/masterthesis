\chapter{Related work}
\label{chp:relatedwork} 



\section{Cyber-Insurance}
\subsection{Paper from Bohme - SKAL FJERNES ETTERHVERT}
While several authors have expressed doubts about the future of cyber-insurance, the authors of \cite{bohme2010modeling} still have faith in the prevalence of cyber-insurance. The paper describes the three main problems of cyber-insurance; information asymmetry, correlated risk and interdependent agents. They argue that a model for cyber-insurance has to encounter each of these obstacles. Instead of presenting a solution they propose a framework to classify models of cyber-insurance. 

The framework breaks the modeling down to five key components: 
\begin{itemize}[topsep=-1em,parsep=0em,itemsep=0em] 
 \item network environment(nodes controlled by agents, who extract utility. Risk arises here.)
 \item demand side(agents) 
 \item supply side(insurers) 
 \item information structure, distribution of knowledge among the players. 
 \item organizational environment. Public and private entities whose actions affect network security and agents security decisions.
 
\end{itemize}
The goal is that this unifying framework will help navigating the literature and stimulate research that results in a more formal basis for policy recommendations involving cyber-risk reallocation. They encourage to answer questions such as; under what conditions will a cyber-insurance market thrive? What is the effect of an insurance market, -will the Internet be more secure? Does it contribute to social welfare?
The paper studies other existing models, and reveals a discrepancy between informal arguments in favor of cyber-insurance and analytic results questioning the viability of a cyber-insurance market. 



\subsection{A novel cyber-insurance Model  - FJERNES ETTERHVERT}
The paper \cite{pal2011aegis} presents a cyber-insurance model which handles both risks due to security (e.g virus) and non-security related features such as power outage and hardware failure. Their model, Aegis, is a simple model in which the user accepts a fraction of loss recovery to himself and the rest is transferred to the insurance company. They show that when it is mandatory to purchase insurance, risk averse agents would prefer Aegis contracts over traditionally cyber-insurance products.
The model also give users incentive to take a greater responsibility in securing their own systems. Hence this answers one of the questions from \cite{bohme2010modeling}: The overall security of the Internet will increase if the Aegis is offered to the market. An interesting result from their analysis is the fact that a decrease/increase in the insurance premium may not always lead to increase/decrease in demand. From the insurers point of view, this features means that one can increase the margins without loosing possible customers. Hence it will be easier to create a market for cyber-insurance. This is good, because in our findings, we show that an increase in insurance premium is needed to make the insurance product attractive to the customers, because if the price is to low, it will encourage making risky decisions. 


\subsection{Cyber-insurance for cyber-security, A Topological Take on Modulating Insurance Premiums - FJERNES ETTERHVERT}
\cite{pal2012cyberinsurance} adopts a topological perspective in proposing a mechanism that accounts for the positive
 externalities (due to purchase of security mechanisms)and network location of users. In addition they provide an appropriate way to proportionally allocate fines/rebates on user premiums. This feature relates to our model, where a central node in the network receives a bulk insurance discount, in order to facilitate creation of insurable star topologies. 
  
  
 
 
\subsection{Differentiating Cyber-insurance Contracts, a topological Perspective - FJERNES ETTERHVERT}

\cite{paldifferentiating} present the importance of discriminating network users in insurance contracts. This is done to prevent adverse selection, partly internalizing the negative externalities of interdependent security, achieving maximum social welfare, helping a risk-averse insurer to distribute costs of holding safety capital among its clients, and insurers sustaining a fixed amount of profit per contract.
The paper proposes a mechanism to pertinently contract discriminate insured users when having complete network information. This is important since almost every node in the network is different from each other. Hence we need a way of distinguish good nodes from bad ones by the means of the premium price.




  
The paper \cite{majuca2006evolution} summarizes the evolution of cyber-insurance, from the early primitive hacker insurance policies, to the modern and comprehensive cyber-insurances. The insurances have evolved, since the insurers have better understanding of the risks and the needs of the businesses. They show how insurers are addressing the adverse selection and moral hazard problem, by classifying the risk level of the insured. They do so by requesting lots of background information about the customers. The paper presents a methodology for calculating the social welfare loss due to adverse selection. The paper is optimistic about the future of cyber-insurance, and concludes that cyber-insurance is making the Internet a safer environment, because insurers are giving users economic incentives to self-protect. 

High correlation in failures in information systems is a huge concern to the cyber-insurance market. The paper \cite{bohme2006models} introduces a new classification of correlation properties. The authors divide it into two levels, the first level is the correlation within firms, and the second level addresses the global correlations. Further, they create an economic model for risk arrival at these two levels, and use it in simulations to find where a market for cyber-insurance can exist or not. Böhme, et.al. results show that cyber-insurance is best suited for risk-classes where internal correlation is high, and the global correlation is low. This could be one of the reasons for the failure of cyber-insurance so far, the insurers have focused on the wrong markets, i.e. the markets with high global correlation and low internal correlation. One problem with \cite{bohme2006models} is that they do not cover interdependent security neither discuss the problem of information asymmetry.  

The papers \cite{bolot2008cyber, bolot2008new} present how risk management on the internet have only introduced methods to reduce the risks, such as firewalls, intrusion detection systems, anti virus etc. But none of these have managed to remove the risk completely. As mentioned in chapter 1, there are four possible ways of removing risk: avoid it, retain it, self-protect and mitigate it or transfer the risk. Most entities on the internet have chosen a mix of of retaining and mitigate by self-protecting. These solutions do not eliminate risk completely, and threats evolve over time. Thus, the only option for completely removing the risk, is to transfer it to a party who willingly accepts it, in exchange for a fee. 
The key result of these papers is that they show economic reasons for users to not invest in self-protection, and that cyber-insurance will act as an incentive for users to acquire self-protection, i.e. the level of security in the internet will increase with cyber-insurance. The reason for this positive spiral is that investment in insurance will result in overall higher payoff, and since the premiums discriminate users based on the investment in self protection, it will act as a strong incentive to acquire self-protection. 

In contrast to the papers \cite{bolot2008cyber, bolot2008new}, the paper \cite{shetty2010competitive} claims that in a competitive cyber-insurance market the users' utility will improve, but the network security will worsen relative to a market without cyber-insurance. By competitive insurers Shetty et.al, mean that there are several insurance contracts to choose from, and user choose the one that increases their utility the most.
\cite{shetty2010competitive} create and explains two models, in the first one the insurers suffer from information asymmetry regarding the users' investment in security. For most of the parameters used, there will not be offered any insurance in an equilibrium, due to the moral hazard problem. In the second model, the insurer is able to observe the users' security investment, and can thus contract discriminate bad and good users, i.e. no moral hazard is present. In this model the insurance increases the users' utility, but it does not result in overall better security.  

The paper \cite{radosavac2008using} also claims that so far in security management of the internet, there is no solution to the residual risk. To solve this the authors come up with an insurance policy, which they claim can survive in a competitive market. However, this solution does not cover the problem of correlated risk.
The actors in the model from \cite{radosavac2008using} are the ISP, who is also the insurer, and the users, the users are of type high- or low-risk users. The question asked is whether a policy that brings profit to the ISPs while protecting the users from risk, exists. Radosavac, et.al find an optimal insurance policy that can be offered to both low and high risk users. Additionally, they also conclude that even when there is an insurance for residual risk, it cannot be guaranteed that a profitable business model exists. 

The paper \cite{danezis2006network} describes an interesting network formation game. Although the paper tries to observe susceptibility to sybil attacks in peer-to-peer networks, the approach used for network formation can be related to our thesis. The game's characteristics is as follows: Nodes are either friends or strangers, and the goal of the nodes is to selfishly try to fulfill own communication needs. The nodes' needs is to communicate with as many as possible of their friends. This can be achieved either by direct or indirect connections. Every node has a link budget, i.e. a maximum number of links it can establish, and a set of friends it wants to connect to. 
\cite{danezis2006network} proposes two random games where nodes might have to take the risk of connecting to non-insured nodes.
\begin{enumerate}
\item Random model: Every node in the network initiates a set of friendships with other nodes, denoted $F$. All nodes have the same link budget $L<F$. 
\item Unbalanced Random Mode. The same friendship graph as in the random model is created. However, one of the nodes has a significantly larger link budget $(L_{0} > 2 F)$
\end{enumerate}
The first model does not result in any equilibrium, except the one where friends only connect to other friends.
The other model shows some new insights, when the link budget is comparable to their number of friends, most nodes still choose to only connect to friends. However, when the link budget is set to only one link, except for the rich node, then the resulting equilibrium is a star topology. 

\subsection{Summary}
There are many different papers that have described the problems of cyber-insurance, and proposed different models and solutions. However, as we revealed, the cyber-insurance market is yet far from established and still has lots of potential. Each of the presented models has a slightly different angle towards improving the cyber-insurance market. Although promising results the paper presents few improvements have appeared in the market, and it seems that another approach is needed. This is what we intend to do in this thesis. In brief, we will investigate whether there are any advantageous structures for cyber-insurance and see if it is possible for networks to evolve endogenously into these structures.  

