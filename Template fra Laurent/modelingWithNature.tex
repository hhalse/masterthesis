\section{Model 2b: Model with incomplete information}
\label{Model with incomplete information}
Although, we previously mentioned that we chose to assume complete information about other nodes type. We wanted to get an impression of the complexity when modeling a scenario where some nodes lack information about the other nodes type. The way we model this is by letting nature selecting whether a player is insured or not, a node is insured with probability $p$, and not insured with probability $1-p$. 
All nodes know their own type, but in the link establishment process only one node knows the type of the other. The other node only know the probability of the other node being insured or not. 
We want to see if it is possible for the nodes with incomplete information to distinguish an insured node from a non-insured one, and then be able to follow the same procedures as in the other nodes. 
For this model we will only present the results from the analysis, because the mathematics and analysis is to complex to include here. The actual analysis and mathematics of this game can be seen in appendix \ref{ch:analysis-of-model-2b}.

\subparagraph{Result and findings.}
When one player lack knowledge about the other player, we were only able to find two scenarios where the player with less information could separate the two types of the other node. This is only possible when player 2 is insured and $\beta<I_{l}$. Then separate the two types of the other player, because it is only the non-insured node who wishes to establish a link. However, since the benefit is less than the cost( $\beta<I_{l}$), his best response is to not establish any link.

The other scenario where the node with incomplete information are able to separate is when he is not insured, and this is true:$r<\beta<I_{l}+r$. In this scenario it is only the non-insured node who would want to establish a link. Thus in this scenario the game will end up with a link between two non-insured nodes.

We where also able to find some pooling equilibriums, if the node with incomplete information is insured, a link will be established if this is true: $\beta+rp-r>I_{l}$. However, if $I_{l}<\beta \text{ but } I_{l}>\beta+rp-r$, then the pooling equilibrium will be: node 1 wants to establish link, but node 2 rejects.
There is also a pooling equilibrium where both nodes want to establish a link, this occurs when node 2 is not insured and this is true: $\beta-I_{l}>r$. If the benefit from establishing a link, is less than the expected risk cost( $\beta<r$), there will be a pooling equilibrium where both players choose not to establish link. 

What this shows us is that when one player suffer from incomplete information, it is no longer possible for the insurer to force a network to evolve into a state where there exists a clique of only insured nodes. Also the incentive for establish links decreases , since the player with less information must act on beliefs.  
We chose not to include the simulation of this model, since the result would only be a clique of non-insured or a giant clique consisting of both insured and non-insured nodes.  



