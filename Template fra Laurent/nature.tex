\chapter{Evolutionary dynamics on graphs}
\label{chp:nature} 
\cite{lieberman2005evolutionary}
Tanker fra paper:

If advantegous mutant(f.eks virus) incerted in circulation graph, it will have a fixation probability 
\begin{equation}  p_{1}=\frac{(1-1/r)}{(1-1/r^{N})} \label{eq:fixation} \end{equation}
Isotherm graphs are a subgraph of circulation. 

If $W$ is symmetric, or isotherm then the fixation probability is allways \ref{eq:fixation}
isotherm means doubly stochastic, all rows and cols sum to 1. 
If a graph is one rooted, it has a fixation prob of $1/N$ regardless of $r$. If a graph has more then one root, its fixation probability is zero. 
Is it possible to find graphs with fixation probability that exceeds \ref{eq:fixation}? Is it possible to suppress drift and amplify selection?
In a star-topology \ref{fig:star} the fixation probability is\begin{equation}p_{2}=\frac{(1-1/r^{2})}{(1-1/r^{2N})} \label{eq:fixation2} \end{equation}.
or more generally: \begin{equation}
p_{k}=\frac{(1-1/r^{k})}{(1-1/r^{kN})} \label{eq:fixationk}
\end{equation}
And the selective difference is as we see amplified from $r$ to $r^{2}$. i.e. a star act as an evolutionary amplifier,
 favouring advantageous mutants and inhibiting disadvantageous mutants, tilts towards selection and against drift.
 in certain graphs, star, funnel, metafunnel, if N is large enough, fixation proability for advantageous mutant converges to 1. Fixprob for disadvantaeus converges to 0.
This could maybe be used to show that if we have a star, funnel, metafunnel or something, and we secure the nodes it could force the virus to die out?
Scale-free networks have most of their connectivity clustered in a few vertices, i.e. they are potent selection amplifiers.
More generalized, $W$ does not need to be stochastic, $w_{ij}>=0$. 
If the sum of all edges leaving a vertex is equal for all vertexes, then the graph will never suppress selection.
If the sum of all edges entering a vertex is equal for all vertexes, the graph never suppress drift.
If both then the graph is called a circulation.
 

\begin{figure}
\centering
\begin{tabular}{@{}c@{}}
\includegraphics[width=0.5\textwidth]{NetworkTopology-Star.png}
\end{tabular}
\caption{\label{fig:star} A star-topology}
\end{figure}



