\chapter{Network Games}
\label{chp:networkgames} 


notater, men brukt de i nature delen :)
When investigating cyber insurance and insurable topologies, one may end up only thinking about standard risk networks, like the internet. 
But we will try to investigate all kinds of networks, or more concrete networks where a players actions is influenced by neighbourhood structure, i.e. the network connections will affect each individual players payoff. 
In this case there are several types of networks to consider, all social and economic interactions where an agents well being is dependent on externalities as well as her own actions, will create a sort of network.

The internet is a good example, because on the internet we are "all" connected, the benefit we get from the internet is strongly dependent on this, and so is the risk we face when using the internet. 
There are other networks as well, when an actor are developing a software product, this development process is often done by several different firms, and thus creates a development network, where everyone is dependent on the result of the others. 
If one or more fail in some way, bankruptcy, failure to deliver at the expected time, higher cost etc. 
Then the whole network will be affected. 
Or in a cloud computing network, there are many different users and internet service providers, and the overall security is dependent on all of them. 
As we see there are many different types of networks, some face direct connections, other consist of social and economical connections. But they all share some main characteristics, they are all experiencing network effects, externalities, information asymmetry, correlated risk and interdependent security. \cite{networkgames}


This paper\cite{networkgames}  provide a framework for analyzing situations when a players actions is influenced by neghbourhood structure, modeled in terms of an underlying netwrok of connections that affect payoff.
The players are partially informed about the structure. 

There are many social and economic interactions where an agents well being depends on her own actions as well as on actions taken by others, i.e. externalities. 


