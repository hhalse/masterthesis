\pagestyle{empty}
\renewcommand{\abstractname}{Sammendrag}
\begin{abstract}
\noindent Cyberforsikring er et kraftfullt økonomisk konsept som kan støtte bedrifter i en verden full av nettkriminalitet. Forskere har fra starten av 80-tallet spådd en lys fremtid for cyberforsikring, og har ment at det kunne bli et viktig økonomisk verktøy for å håndtere cyberrisiko.

Markedsundersøkelsen i denne oppgaven avdekket at hverken det europeiske eller det amerikanske markedet for cyberforsikring har greid å bli en viktig faktor i IKT-industrien. Selv om det amerikanske markedet er noe bedre utviklet enn det europeiske, så har begge langt igjen. Det trengs nye og innovative framgangsmåter for å håndtere de unike problemene knyttet til cyberforsikring.

Denne oppgaven presenterer en ny tilnærming for å prøve å løse noen av problemene knyttet til cyberforsikring. Først ble nettverksstrukturer med egenskaper som gjorde dem spesielt godt egnet som cyberforsikringsnettverk funnet og definert. Deretter ble flere modeller for å skape disse nettverksstrukturene introdusert og analysert gjennom spillteori, og ved bruk av simuleringsverktøyet «Netlogo». I hver modell ble så nye egenskaper introdusert, der hver egenskap kunne relateres til den virkelige verden og til virkelige forsikringsprodukter, slik at modellene ble mer realistiske. Resultatene viste at forsikringsselskapene kan bruke forsikringspremien som et verktøy for å bestemme den resulterende nettverksformasjonen. Hvis forsikringspremien blir satt til riktig nivå, så vil de spesielt godt egnete nettverksstrukturene oppstå. 

Vi mener at våre modeller og resultater kan hjelpe cyberforsikringsmarkedet til å utvikle seg, og vil kunne gi forsikringsselskaper et egnet verktøy til å analysere og kontrollere hvordan cyberforsikringsnettverk oppstår.

Fremtidig arbeid bør søke å la modellene og simuleringene i denne oppgaven nærme seg den virkelige verdens nettverk ytterligere. Dette kan gjøres ved å finne og introdusere mer realistiske risikofunksjoner, og ved å la noder opprette linker etter preferanse heller enn tilfeldig.

\end{abstract}