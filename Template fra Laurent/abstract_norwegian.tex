\pagestyle{empty}
\renewcommand{\abstractname}{Sammendrag}
\begin{abstract}
\noindent Cyber-forsikring er et viktig konsept og kan hjelpe bedrifter i kampen mot cyberkriminalitet. Forskere har fra starten av 80-tallet spådd en lys fremtid for cyber-forsikring, og hevdet at det kunne bli et viktig verktøy for å håndtere cyberrisiko.

Markedsundersøkelsen i denne oppgaven avdekket at hverken det europeiske eller det amerikanske markedet for cyber-forsikring har greid å bli en betydningsfull faktor i IKT-industrien. Selv om det amerikanske markedet er noe bedre utviklet, så har begge langt igjen. Det trengs nye og innovative løsninger for å håndtere de unike problemene med cyber-forsikring.

Denne oppgaven prøver å løse problemene med cyber-forsikring på en ny måte. Først ble det funnet og definert nettverksstrukturer med egenskaper som gjorde de spesielt godt egnet som cyber-forsikrings nettverk. Dernest ble flere modeller for å skape disse nettverksstrukturene introdusert og analysert gjennom spillteori, og ved bruk av et simulerings verktøy kalt «Netlogo». Hver modell introduserte nye egenskaper, og hver egenskap kunne relateres til den virkelige verden og virkelige forsikrings produkter, noe som gjorde modellene mer realistiske.
Resultatene viste at forsikringsselskapene kan bruke forsikringspremien som et verktøy for å bestemme den resulterende nettverksformasjonen. Hvis forsikringspremien blir satt til riktig nivå, så vil de spesielt godt egnete nettverksstrukturene oppstå.

Vi mener at våre modeller og resultater kan hjelpe cyber-forsikring markedet til å utvikle seg, og gi forsikringsselskaper et egnet verktøy til å analysere og kontrollere formasjonen til cyber-forsikring nettverk.

Fremtidig arbeid bør forsøke å tilpasse modellene og simuleringene til mer realistiske scenarioer på en bedre måte. Dette kan bli gjort ved å finne og introdusere bedre funksjoner for risiko, og ved å la noder opprette linker etter preferanse.

\end{abstract}