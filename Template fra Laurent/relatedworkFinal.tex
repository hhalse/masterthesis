
While several authors have expressed doubts about the future of cyber-insurance, the authors of \cite{bohme2010modeling} still have faith in the prevalence of cyber-insurance. The paper describes the three main problems of cyber-insurance; information asymmetry, correlated risk and interdependent agents. They argue that a model for cyber-insurance has to encounter each of these obstacles. Instead of presenting a solution they propose a framework to classify models of cyber-insurance. 
The framework breaks the modeling down to five key components: 
\begin{itemize}[topsep=-1em,parsep=0em,itemsep=0em] 
 \item network environment(nodes controlled by agents, who extract utility. Risk arises here.)
 \item demand side(agents) 
 \item supply side(insurers) 
 \item information structure, distribution of knowledge among the players. 
 \item organizational environment. Public and private entities whose actions affect network security and agents security decisions.
 
\end{itemize}


The goal is that this unifying framework will help navigating the literature and stimulate research that results in a more formal basis for policy recommendations involving cyber-risk reallocation. They encourage to answer questions such as; under what conditions will a cyber-insurance market thrive? What is the effect of an insurance market, -will the Internet be more secure? Does it contribute to social welfare?
They also analyze several other papers on cyber-insurance, and show how all of them are touching into the problems and key components showed above, but none handles all of them.
The paper studies other existing models, and reveals a discrepancy between informal arguments in favor of cyber-insurance and analytic results questioning the viability of a cyber-insurance market. 

The paper \cite{pal2011aegis} presents a cyber-insurance model which handles both risks due to security (e.g virus) and non-security related features such as power outage and hardware failure. Their model, Aegis, is a simple model in which the user accepts a fraction of loss recovery to himself and the rest is transferred to the insurance company. They show that when it is mandatory to purchase insurance, risk averse agents would prefer Aegis contracts over traditionally cyber-insurance products.
The model also give users incentive to take a greater responsibility in securing their own systems. Hence this answers one of the questions from \cite{bohme2010modeling}: The overall security of the Internet will increase if the Aegis is offered to the market. An interesting result from their analysis is the fact that a decrease/increase in the insurance premium may not always lead to increase/decrease in demand. From the insurers point of view, this features means that one can increase the margins without loosing possible customers. Hence it will be easier to create a market for cyber-insurance.  

\cite{pal2012cyberinsurance} adopts a topological perspective in proposing a mechanism that accounts for the positive
 externalities (due to purchase of security mechanisms)and network location of users. In addition they provide an appropriate way to proportionally allocate fines/rebates on user premiums. This feature relates to our model, where a central node in the network receives a bulk insurance discount, in order to facilitate creation of star topologies. 
  
\cite{paldifferentiating} present the importance of discriminating network users in insurance contracts. This is done to prevent adverse selection, partly internalizing the negative externalities of interdependent security, achieving maximum social welfare, helping a risk-averse insurer to distribute costs of holding safety capital among its clients, and insurers sustaining a fixed amount of profit per contract.
The paper proposes a mechanism to pertinently contract discriminate insured users when having complete network information. This is important since almost every node in the network is different from each other. Hence we need a way of distinguish good nodes from bad ones by the means of the premium price.

  