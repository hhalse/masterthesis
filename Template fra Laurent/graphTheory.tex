\chapter{Graphs and Network Formation}
\label{chp:graphTheory} 

In nature and our society there are lots of scenarios that can be described using graphs. A range from infrastructure, such as railroads, water pipelines and electricity grid, to societal relationships, disease epidemics can all be visualized using graphs. Cyber-insurance is no exception, and can also be structured as a graph. This is of interest because, when one can describe a phenomenon with graphs, it is much easier to analyze and find its characteristics. Hence the graph therefore serves as an analytic tool \cite{audestad}. 

Several studies have been done on the characteristics of different graphs, such as E-R graphs and A-B graphs (scale-free graphs) which is thoroughly described in the methodology chapter. In addition, one have found special characteristics on star-shaped graphs and cliques. This chapter will highlight which characteristics are desirable to have in the cyber-insurance market, and point out the structures that possess these characteristics. The results found here will serve as the foundation of our modeling work, where we try to force these graph structures to emerge endogenously. 
 
\section{Real world graph structures}
As a starting point lets have a look at a couple of real world examples of how complex systems with huge amount of data could be structured as graphs. We will see how complex structures becomes rather intuitive when presented as a graph. By looking at the graph structure, one can determine what type of graph that appears, and hence certain characteristics will apply.  

\subparagraph{Stock markets.} The research paper \cite{greekStockMarket}, analyzes the correlation between different stocks in the Greek stock market in year 1997. They compared the daily closing price of stock $i$ at day $t$, and compared the similarity of a pair of stocks $i$ and $j$ by using the correlation coefficient. The idea is that the correlation coefficient between a pair of stocks can be expressed using different distances in a graph structure. A short distance means high correlation and long distance means low correlations between the stocks. Normally this network would be shown as a fully connected graph, which will consist of $\frac{n(n-1)}{2}$ edges, and would be difficult to analyze. However the approach taken in the paper will present a clear understandable graph consisting of $(n-1)$ edges showing the correlations between the stocks.

The resulting graph can be seen in Figure \ref{fig:greekStockMarket}, and show a network consisting of several clusters linked together. Instead of having to analyze a complex system with huge amount of data, the stock market can be analyzed by its topological properties, such as the high clustering coefficient, i.e a star-topology, which will among others point out which stocks have the most influence on others. 
\begin{figure}[h]
\centering
\begin{tabular}{@{}c@{}}
\includegraphics[width=1.0\textwidth]{../Figures/greekStockMarket.jpg}
\end{tabular}
\caption[Caption for LOF]{Network obtained by comparing two stocks correlation coefficient in the Greek stock market (Athens Stock Exchange, ASE) in year 1997. The different colors represent the different sectors of economic activity \cite{greekStockMarket}.
\label{fig:greekStockMarket}}
\end{figure}

\subparagraph{Airline routes.}
Another real world network which shows the same characteristics as scale-free graphs is the map of airline routes. Figure \ref{fig:airlineRouteMap} shows the US route map of the American airline company, SkyWest. The characteristic clustering emerges in the figure, where a majority of the flights departs from either Denver, Chicago or San Francisco. Not surprisingly, these airports are all in the top 7 busiest airports in the US \cite{busiestAirports}, and serves as hubs for many of SkyWest flights. In the airline industry some airports are called hubs, because that's what they are, - a connection point for major parts of the network of flights. The network of flights, as depicted in Figure \ref{fig:airlineRouteMap} follows the characteristics for A-B graphs. Hence, as we also can confirm from looking at the graph, the network are vulnerable against direct attacks. This means that if an low edge degree airport is shut down, there will be little consequence for the rest of the network. However, if one of the hubs is forced to close, it will provoke huge delays through out the whole network of flights, because many of the destinations are interconnected via the hubs. 


\begin{figure}[h]
\centering
\begin{tabular}{@{}c@{}}
\includegraphics[width=1.0\textwidth]{../Figures/airlineRoutesUSA.png}
\end{tabular}
\caption[Caption for LOF]{SkyWest Airline combined route map \cite{airlineRoutes}.
\label{fig:airlineRouteMap}}
\end{figure}

Both examples can be characterized as a scale-free network, and often natural networks are scale-free. Since we are able to ascertain this we know for example that one will experience large consequences if a hub in the network stop functioning. 
This shows the strength of being able to structure systems as graphs. When a certain structure appears one can assume that the network will behave according to a set of rules. This is why we wish determine whether there are any structures that possesses preferred characteristics to be associated with cyber-insurance. Because then the work is reduced to find a proper way to force the formation of the given network structure.


\chapter{Evolutionary dynamics on graphs}
\label{chp:nature} 
\cite{lieberman2005evolutionary}
Tanker fra paper:

If advantegous mutant(f.eks virus) incerted in circulation graph, it will have a fixation probability 
\begin{equation}  p_{1}=\frac{(1-1/r)}{(1-1/r^{N})} \label{eq:fixation} \end{equation}
Isotherm graphs are a subgraph of circulation. 

If $W$ is symmetric, or isotherm then the fixation probability is allways \ref{eq:fixation}
isotherm means doubly stochastic, all rows and cols sum to 1. 
If a graph is one rooted, it has a fixation prob of $1/N$ regardless of $r$. If a graph has more then one root, its fixation probability is zero. 
Is it possible to find graphs with fixation probability that exceeds \ref{eq:fixation}? Is it possible to suppress drift and amplify selection?
In a star-topology \ref{fig:star} the fixation probability is\begin{equation}p_{2}=\frac{(1-1/r^{2})}{(1-1/r^{2N})} \label{eq:fixation2} \end{equation}.
or more generally: \begin{equation}
p_{k}=\frac{(1-1/r^{k})}{(1-1/r^{kN})} \label{eq:fixationk}
\end{equation}
And the selective difference is as we see amplified from $r$ to $r^{2}$. i.e. a star act as an evolutionary amplifier,
 favouring advantageous mutants and inhibiting disadvantageous mutants, tilts towards selection and against drift.
 in certain graphs, star, funnel, metafunnel, if N is large enough, fixation proability for advantageous mutant converges to 1. Fixprob for disadvantaeus converges to 0.
This could maybe be used to show that if we have a star, funnel, metafunnel or something, and we secure the nodes it could force the virus to die out?
Scale-free networks have most of their connectivity clustered in a few vertices, i.e. they are potent selection amplifiers.
More generalized, $W$ does not need to be stochastic, $w_{ij}>=0$. 
If the sum of all edges leaving a vertex is equal for all vertexes, then the graph will never suppress selection.
If the sum of all edges entering a vertex is equal for all vertexes, the graph never suppress drift.
If both then the graph is called a circulation.
 

\begin{figure}
\centering
\begin{tabular}{@{}c@{}}
\includegraphics[width=0.5\textwidth]{NetworkTopology-Star.png}
\end{tabular}
\caption{\label{fig:star} A star-topology}
\end{figure}











