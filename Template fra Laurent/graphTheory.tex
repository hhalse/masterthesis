\chapter{Graphs and Network Formation}
\label{chp:graphTheory} 

In nature and society, many scenarios can be described using graphs. Infrastructure, such as railroads, water pipelines and electricity grid, societal relationships and disease epidemics, can all be visualized using graphs. Cyber-insurance is no exception, and can also be structured as a graph. This is of interest because, when one can describe a phenomenon with graphs, it is easier to analyze and possibly find some characteristics, hence the graph can be used as an analytic tool \cite{audestad}. 

Several studies have been done on the characteristics of different graphs, such as E-R graphs and A-B graphs (scale-free graphs), these are thoroughly described in the methodology chapter of this thesis. In addition, one has found special characteristics on star-shaped graphs and cliques. This chapter will highlight which characteristics that are desirable in the cyber-insurance market, and which structures that possess these characteristics. These findings will serve as the foundation of our models, where we try to force these graph structures to emerge.
 
\section{Real-world graph structures}
As a starting point, let's have a look at a couple of real-world examples of how complex systems with huge amount of data could be structured as graphs. We will see how complex structures become rather intuitive when presented as graphs. By looking at the graph structure, one can determine what type of graph that appears, and hence certain characteristics will apply.  

\subparagraph{Stock markets.} The research paper \cite{greekStockMarket} analyzes the correlation between different stocks in the Greek stock market in year 1997. The authors compared the daily closing price of stock $i$ at day $t$, and compared the similarity of a pair of stocks $i$ and $j$ by using the correlation coefficient. The idea is that the correlation coefficient between a pair of stocks can be expressed using different distances in a graph structure. A short distance means high correlation and a long distance means low correlation between the stocks. Normally, this network would be shown as a fully connected graph, which will consist of $\frac{n(n-1)}{2}$ edges, and would be difficult to analyze. However, the new approach presents a clear and understandable graph, consisting of $(n-1)$ edges showing the correlations between the stocks.

The resulting graph can be seen in Figure \ref{fig:greekStockMarket}, and shows a network consisting of several clusters linked together. Instead of having to analyze a complex system with huge amounts of data, the stock market can be analyzed by its topological properties, such as the high clustering coefficient, i.e a star topology, which will among other things point out which stocks have the most influence on others. 
\begin{figure}[h]
\centering
\begin{tabular}{@{}c@{}}
\includegraphics[width=1.0\textwidth]{../Figures/greekStockMarket.jpg}
\end{tabular}
\caption[Caption for LOF]{Network obtained by comparing two stocks' correlation coefficient in the Greek stock market (Athens Stock Exchange, ASE) in year 1997. The different colors represent the different sectors of economic activity \cite{greekStockMarket}.
\label{fig:greekStockMarket}}
\end{figure}

\subparagraph{Airline routes.}
Another real-world network which shows the same characteristics as scale-free graphs is the map of airline routes. Figure \ref{fig:airlineRouteMap} shows the US route map of the American airline company SkyWest. The characteristic clustering emerges in the figure, where a majority of the flights departing from either Denver, Chicago or San Francisco. Not surprisingly, these airports are all in the top 7 busiest airports in the US \cite{busiestAirports}, and serve as hubs for many of SkyWest's flights. In the airline industry some airports are called hubs, because that's what they are, - a connection point for major parts of the network of flights. The network of flights, as depicted in Figure \ref{fig:airlineRouteMap}, follows the characteristics of A-B graphs. Hence, as we can confirm from looking at the graph, the network are vulnerable against direct attacks, meaning that shutting down a low degree airport wont create much trouble. However, if one of the hubs is forced to close, it will provoke huge delays throughout the whole network, because a majority of the destinations is interconnected via the hubs. 


\begin{figure}[h]
\centering
\begin{tabular}{@{}c@{}}
\includegraphics[width=1.0\textwidth]{../Figures/airlineRoutesUSA.png}
\end{tabular}
\caption[Caption for LOF]{SkyWest Airline combined route map \cite{airlineRoutes}.
\label{fig:airlineRouteMap}}
\end{figure}

Here, both examples can be characterized as scale-free networks, and the work done by Albert and Barabási shows that a large part of natural systems is in fact scale-free graphs \cite{audestad}. Since we are able to determine the graph's type, which in this case is a scale-free graph, we now know that the graph is vulnerable to attacks directed towards the hubs, i.e. the hubs need to be secured. For example, if a delay occurs at an airline hub, these delays will probably cascade throughout the network.
This shows the strength of being able to structure systems as graphs. When certain structures appear, one can assume that the network will behave according to a set of rules. This is why we wish to determine whether there are any structures that possess preferred characteristics for cyber-insurance, and then find a proper way to force these formations to evolve.


\chapter{Evolutionary dynamics on graphs}
\label{chp:nature} 
\cite{lieberman2005evolutionary}
Tanker fra paper:

If advantegous mutant(f.eks virus) incerted in circulation graph, it will have a fixation probability 
\begin{equation}  p_{1}=\frac{(1-1/r)}{(1-1/r^{N})} \label{eq:fixation} \end{equation}
Isotherm graphs are a subgraph of circulation. 

If $W$ is symmetric, or isotherm then the fixation probability is allways \ref{eq:fixation}
isotherm means doubly stochastic, all rows and cols sum to 1. 
If a graph is one rooted, it has a fixation prob of $1/N$ regardless of $r$. If a graph has more then one root, its fixation probability is zero. 
Is it possible to find graphs with fixation probability that exceeds \ref{eq:fixation}? Is it possible to suppress drift and amplify selection?
In a star-topology \ref{fig:star} the fixation probability is\begin{equation}p_{2}=\frac{(1-1/r^{2})}{(1-1/r^{2N})} \label{eq:fixation2} \end{equation}.
or more generally: \begin{equation}
p_{k}=\frac{(1-1/r^{k})}{(1-1/r^{kN})} \label{eq:fixationk}
\end{equation}
And the selective difference is as we see amplified from $r$ to $r^{2}$. i.e. a star act as an evolutionary amplifier,
 favouring advantageous mutants and inhibiting disadvantageous mutants, tilts towards selection and against drift.
 in certain graphs, star, funnel, metafunnel, if N is large enough, fixation proability for advantageous mutant converges to 1. Fixprob for disadvantaeus converges to 0.
This could maybe be used to show that if we have a star, funnel, metafunnel or something, and we secure the nodes it could force the virus to die out?
Scale-free networks have most of their connectivity clustered in a few vertices, i.e. they are potent selection amplifiers.
More generalized, $W$ does not need to be stochastic, $w_{ij}>=0$. 
If the sum of all edges leaving a vertex is equal for all vertexes, then the graph will never suppress selection.
If the sum of all edges entering a vertex is equal for all vertexes, the graph never suppress drift.
If both then the graph is called a circulation.
 

\begin{figure}
\centering
\begin{tabular}{@{}c@{}}
\includegraphics[width=0.5\textwidth]{NetworkTopology-Star.png}
\end{tabular}
\caption{\label{fig:star} A star-topology}
\end{figure}











