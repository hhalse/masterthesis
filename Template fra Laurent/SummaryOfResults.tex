\chapter{Summary}
\section{Discussion}
From our background study, it was relieved that the current market for cyber-insurance is far from healthy, and many have failed in attempts to establish a cyber-insurance market, also here in Norway. As described in the introduction, there are certain obstacles unique for cyber-insurance, and arguably these are the reasons why cyber-insurance have not emerged as expected, and many are skeptical about the future of cyber-insurance.  However, we believe that there is a need for cyber-insurance, and that our new approach of analyzing the cyber-insurance market through graphs and network formation games could help improving and establishing the market.   

We studied a variety of different network formation games, in order to find out if there where any superior network topologies that would fit as a cyber-insurance network. -Where ideally both the insurer and customers gets a higher payoff from purchasing cyber-insurance. We found that star and clique networks had appropriate characteristics, not only do they have calculable fixation probability, but they could also generate better security and overall higher payoff for the nodes. With these networks in mind, we wanted to find a way of forcing networks to evolve into these structures.  We found that insurers could adjust the insurance premium in order to control the formation of networks. If the price is set to the right level, networks with calculable risk will evolve, and if the insurer are able to separate the nodes into two different network, one consisting of trusted, insured nodes, the other of non-insured nodes, the trusted nodes can even further increase their payoff, compared to a non-trusted network. The insurer now posses a tool for setting the insurance premium properly, resulting in possible better products for both the customer and the insurer.

We created several different models, where the first model, showed a very simple and naive way for the insurer to separate insured and non-insured nodes, into two cliques. To make the model more applicable to real world scenarios, we created several models and for each model we added some new features. To get an overview of the models we created, we refer to the Figure \ref{fig:Overview-of-models}. 

In model-2 we made model-1 realizable, by including the parameters: expected cost of risk, insurance cost and the benefit per link. Then we analyzed the parameters and found out when and how different network structures would evolve. By adjusting the insurance cost to the right level, the insurer can make the network formation game end up in a giant clique of both insured and non-insured nodes, or a clique of only insured and another of only non-insured. The condition for separating insured from non-insured are: $\beta-r<I_{l}<\beta$, additionally if $\beta>r$, the non-insured nodes will also form a clique, and the resulting network will be two cliques. The solution is also stable, since the resulting network consist of one or two cliques, there is not possible to add any more links. Because the change in payoff is linear and non-dependent on the rest of the network, when a link is added, there is no reason to remove it later. This holds for model 1,2,3 and 4.
We also showed that when the insurer sets the cost such that the network ends up in two cliques, it is not the socially optimal. Because the network will suffer from the lost benefits of connections between insured and non-insured nodes, i.e. it has a price of anarchy less than 1.

In model 2b, we showed that to be able to separate the networks into two cliques, the nodes must know the other nodes types. Else, the nodes will have incentive to pretend to be an insured node, which will result in an untrusted network. We think it is reasonable to assume that nodes, in a real world scenario, know whether their transactional partner have insurance or not,therefore we chose not to include this uncertainty in the other models.

In model-3 we applied the model to certain real world scenarios, such as software development firms/chains, or other networks where the final product is dependent on the collaboration of multiple participants.
This was done by including a bonus, which is first received when a node reach the desired number of links(called max-degree). This made the separation process of insured and non-insured nodes more difficult for the insurer. Due to the possibility of reaching the bonus, a node will have more incentive to establish links, and are thus more acceptable towards establishing links with risky nodes. The condition for separating insured and non-insured nodes in this scenario are: $\beta+\gamma-r<I_{l}<\beta+\frac{\gamma}{m}$. For the separation of insured and non-insured nodes to be possible, the following has to hold: $1-\frac{1}{m}<\frac{r}{m}$. As we see, as $\gamma$ and/or $m$ increases, this gets more and more difficult to achieve. 

In Model-4 we tried to implement a common feature used by insurance companies, bulk-discount, in order to see how this affected the network formation. The cost of insuring a link are now dependent on the nodes degree. We implemented this feature on both model 2 and 3, which resulted in even higher incentive for insured nodes to establish links with non-insured nodes. The reason is intuitive since the cost of doing so decreases as the node degree increases. 
When we applied the discount on model 2, the condition for ensuring separation of insured and non-insured nodes were: $N_{I}(\beta-r)<I_{l}<\beta$, where $N_{I}$ represents the number of insured nodes in the network. This condition is very strong, because for the separation to be possible the following has to hold: $N_{I}(\beta-r)<\beta$. As we see, it is now more difficult for the insurer to separate insured and non-insured, compared to model 2. Because now the lower boundary on the insurance cost is multiplied with the number of insured nodes in the network($N_{I}\times(\beta-r)$).

When applying the discount to model 3, the condition to ensure separation becomes: $m(\beta+\gamma-r)<I_{l}<\beta+\frac{\gamma}{m}$, and as in the other models, this further complicates the separation process for the insurer. 

We also showed that the price of anarchy is even higher when applying discount to model 2. This is because the costs are decreasing, and thus when we have two separate cliques the potential lost payoff between them will increase.
We were not able to calculate the price of anarchy in model 3, because the calculation of the optimal solution is to complex when the bonus and max degree is introduced. However, we can see that since the incentive for establishing links have increased, and thus the insurer has to set a higher price to compensate for this, the price of anarchy will be less than 1, i.e. the more incentive for link establishment, the harder it gets to ensure separation of the nodes.  

In our last model we applied our model-4(discount) to an already existing model, "the symmetric connection game". In this old game it has been shown that there exists three different efficient and stable networks, clique, star and an empty network, that arise under certain cost conditions. If $I_{l}<\beta-\beta^{2}$, the efficient and stable network is a clique. If $\beta-\beta^{2}<I_{l}<\beta$ a star is both stable and efficient. If $I_{l}>\beta+\frac{N-2}{2}\beta^{2}$ an empty network is both stable and efficient. In general a clique is the most efficient if the cost of establishing links, is less than the benefit gained from indirect connections. A star is the most efficient if the cost is higher than the benefit from indirect connections, but less than the benefit of direct connections. 
Unfortunately, it is proved that as the number of nodes in the networks increase, the probability of the network ending up in star goes to zero. However when we applied our insurance discount to this model, we found conjectures saying that, setting the cost to the right level, one can with high probability ensure that either a clique, a star or a scale-free structure will evolve. This changes the connection game drastically, because now the insurer are able to force the network into three possible network formations. Where the star has a fixation probability that exceeds the cliques. The insurer can use these findings to ensure that one of the beneficial structures, star or clique evolves. If the insurer are able to force a star to evolve, this can be used to drastically increase the overall security, and at the same time minimizing the overall link-cost. 

\subparagraph{Limitations and future work}
One limitation to our work, and a suggestion for future work, is mapping our models and simulations to real world networks in a more convincing way. Real world network are not random, nodes may prefer to talk to nodes with high degree or low degree, i.e. the payoff function has to be changed. 

Another limitation and suggestion for future work, is that we have assumed additive risk. It is reasonable to assume that the probability of failure increases if a node accepts more and more links to non-trusted nodes. However, we where not able to determine whether the risk parameter increases according to an additive distribution, exponential, logarithmic or something completely different. By introducing a complex risk function, we would only have distorted the goal of the models. The decision of using additive risk was taken due to the simplicity of the function and the fact that we do not know for sure how the distribution actually looks like.

Another interesting thing to research, is the game of choosing insurance or not, in future work this could be applied to our models, however this could possibly be too complex, and only disrupt the models.

\section{Conclusion}

The current market for cyber-insurance is far from healthy, and many have failed in attempts to establish a cyber-insurance market. However, we believe that there is a need for cyber-insurance, and that our new approach of analyzing the cyber-insurance market through graphs and network formation games could help improving and establishing a better market.  

We surveyed different literature on networks and risk, and found recent literature who showed how graphs like cliques, star, super-star, funnel and meta-funnel, all have a calculable fixation probability, and that stars and funnels fixation probability exceeds the one of a clique. 
With these structures in mind, we created and analyzed different network formation games, and tried to find link-cost constraints, which enabled these structures to evolve.

In model one to four, we found cost constraints to separate insured and non-insured nodes into two cliques. 
For each model, we added some new features that made the model more applicable to real world scenarios, and for every feature added, it became more difficult for the insurer to separate the two types of nodes. This is due to the increased incentive for establishing links, and thus the nodes became more and more acceptable towards risk. 


In the last model, we introduced the concept of bulk-insurance into an already existing network formation game, "the symmetric connection game", and showed that this enabled the insurer to determine, with high probability, when and how, cliques, stars or scale-free network would evolve. We showed that at a point, called critical degree, a nodes optimal strategy would change from relaying on indirect connections, to suddenly wanting to connect to everyone. If the critical degree is set to the right level, one can ensure that the different structures evolve. If the critical degree is set to a low degree, a clique will most certainly evolve, at a medium level, a star will evolve, and at a high level, a scale-free network will evolve. We proved this by performing multiple simulations, 50 simulations for every critical degree.  
What makes this a very interesting finding, is that in the connection game, earlier research has proven that as the number of nodes increases, the probability of the network reaching a star goes towards zero. However, by introducing a discount, that will subsidize the center node, one can drastically increase the probability of the network ending up in a star.

In summary, we have shown how insurers can determine the resulting networks, by adjusting the insurance cost, for several network formation games. At the same time helping the insurer calculating the overall probability of fixation.
We found these conditions for several models, with different properties that relate them to the real world and other insurance products. We believe our findings can help the cyber-insurance market evolve into the market everyone thought it would reach.


\subparagraph{NOTATER.....AVOID this in conclsuion}
Avoid claiming findings that you have not proventhroughout your thesis•Avoid introducing new data•Avoid hiding weaknesses or limitations in your thesis(make a virtue of showing strong analytical skills and self-critique by discussing the limitations--but don’t gooverboard on this!)• Avoid making practical recommendations (e.g. for policy).If you must include them put them in an appendix.•Avoid being too long (repetitive) or too short (sayingnothing of importance)
