\chapter{Summary of results/conclusion}

One of the problems with cyber-insurance is to define the risk and calculate premiums, because the network structure is undefined. If an insurer where able to predict the network structure, the calculations of overall risk would be realizable, and even better if the insurer where able to force some network structures to evolve. Especially if these structures had properties that made them better, with an insurers point of view, compared to other structures. Examples of such structures are scale-free network, which have been proven to be very robust against random attacks. Star-topologies, are very good, because here the average path length is minimized, i.e. there is one center node that connect all the other nodes. And they have fixation probability that exceeds the fixation probability of a circulation graph. The nice thing about star structures is that if one can ensure that this center node is secure, one can consider the whole network as very secure. 
So our goal with the models where to find out how and when different networks evolve, and how the insurer can ensure that this will happen, by adjusting the parameter he can control, the insurance cost. 

The first model we made, showed a very simple and naive way for the insurer to separate insured and non-insured nodes, into two cliques. 

 In model-2 we made model-1 realizable, by including parameters and making it apply for multiple nodes. We then analyzed the parameters and found out when and how different network structures would evolve. By adjusting the insurance cost to the right level, the insurer can make the network formation game end up in, one clique of both insured and non-insured nodes, or a clique of only insured and another of only non-insured. 
We also showed that when the insurer sets the cost such that the network ends up in two cliques, then this has a cost compared to the most efficient network. The cost of stability, this is because in overall the network will suffer from the lost benefits of connections between insured and non-insured nodes.

In model-3 we tried to apply the model to certain real world scenarios, such as software development firms/chains, or other networks where the final product is dependent on the collaboration of several parts.
This was done by including a bonus, which where received when you reached the desired number of links(called max-degree). This made the separation process of insured and non-insured nodes, more difficult for the insurer. Because the nodes now have more incentive to establish links, and are thus more acceptable towards risk. We found the conditions for the different network structures to evolve, and showed that these where very strongly dependent on the max-degree. And when the max-degree increases, it gets harder and harder to guarantee two separate cliques. 


In Model-4 we tried to make the model more comparable to other insurance products, by including a bulk-discount. We did this on both model 2 and 3. This resulted in even more incentive, or less disincentive, for insured node to establish links with non-insured nodes, since the cost of doing so decreases linearly with the nodes degree. 
We found the different conditions for when the different networks would evolved, and showed that when applying the discount to model 3, it is very hard for the insurer to ensure two seperate cliques. 
We also showed that the price of stability is even higher when applying discount to model 2. This is because the costs are decreasing, and thus the potential payoff that are missing, when we have two separate cliques, are increasing. 


In model-5 we used the findings from a well know game called the symmetric connection game, and applied these to our scenario with insurance. Since the game is well known in network formation research, this game have allready been analyzed thoroughly, some of these findings are very interesting. They show how a star, under certain cost-conditions, is the most efficient, and can also be stable. But they also show that as the number of nodes increases, the probability of reaching a star approaches zero.
By applying our insurance discount to this model, we found a conjecture that says, by setting the cost to the right level, one can with high probability ensure that a star will evolve.

We have shown how an insurer can separate insured and non-insured nodes into two cliques, by adjusting the insurance cost parameter. This enables him to calculate the probability of fixation, i.e. the topologies are insurable. And also if he can give the cliques incentive to secure them self, so that they will have a higher relative fitness, a virus will be considered disadvantegous. And thus the fixation probability will decrease. 
We found these conditions for several models, with different aspects that relate them to real world and other insurance products.

In our last model we applied our model-4(discount) to an already existing model, "the symmetric connection game". In this old game it had been shown that there exists three different efficient networks that arise under certain cost conditions, and that these also some times where stable. A clique is the most efficient if the cost of establishing links, is less than the benefit gained from indirect connections. A star is the most efficient if the cost is higher than the benefit from indirect connections, but less than the benefit of direct connections. 
It is also shown that as the number of nodes in the networks increase, the probability of the network ending up in star goes to zero. However, when we introduced the cost-discount, we showed that the probability of the network evolving into a star structure rose significantly for a large number of critical degrees. 


