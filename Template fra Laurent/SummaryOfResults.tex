\chapter{Summary}

\section{Discussion}
\subparagraph{notater om diskusjon}
 a) describe the results of running your system on different problem scenarios, and b) analyze those results. Analysis consists of both pointing out interesting patterns in the data and trying to explain them in terms of the structure of your system.
It is fine to divide this chapter into two: one for results and one for a detailed discussion of them.

In many cases, a evaluator is LESS concerned with the actual results but MORE focused on your ability to explain those results. Good analysis can easily be a 1-2 letter-grade difference in a thesis. This is where you can show your skills as a scientist by setting up proper experiments and formally assessing their outcome. For example, if using an evolutionary algorithm to solve a problem, the results of a single run have absolutely no statistical significance. 20-50 runs are preferable.
\textbf{Slutt på notater}

Interesting: insurer are able to separate into two cliques in all scenarios. however, the insurance cost for doing so is increasing. Due to the increased risk acceptance. 
By separating the ensurer can calculate, and thus provide a better product. Also a trusted clique can get super critical payoff in some environments. 
change the old connection game, into evolving into a star. prooved it with simulations. future work, prove it analytically.


From our background study it was relieved that the current market for cyber-insurance is far from healthy, and many people are skeptical about the future of cyber-insurance. We believe that there is a need for cyber-insurance, in order to ensure that companies does not experience a sudden death when facing a technological disaster. As described in the introduction, there are certain obstacles unique for cyber-insurance, and arguably these are the reasons why cyber-insurance have not emerged as expected. In light of earlier failed attempts to establish a cyber-insurance market, such as those we found here in Norway, we wanted to take a new approach towards analyzing the cyber-insurance market. 

First we studied a variety of different network formation games, in order to find out if there where any superior network topologies that would fit as a cyber-insurance network. -Where ideally both the insurer and customers gets a higher payoff from purchasing cyber-insurance. As described in the background study, we found that star structures and cliques consisting of only insured nodes had appropriate characteristics.  From other studies, these networks seemed to generate the overall highest payoffs. Our next task where to find a way of forcing the networks to end up in these topologies. Hence our work contributes mainly to the insurer-side of the market. We found that insurers could actually price their insurance in order to control the formation of networks. These networks are insurable topologies as described earlier. In summary, this means that the insurer can create networks with calculable risk. Meaning that they now posses the tool to price the insurance premium properly, resulting in possible better products for the customer, and their own risk of going bankrupt is highly reduced.

The first model we made, showed a very simple and naive way for the insurer to separate insured and non-insured nodes, into two cliques. By saying that insured only connects to other insured, and non-insured to non-insured. 

 In model-2 we made model-1 realizable, by including the parameters: expected risk cost,insurance cost and the benefit per link. We then analyzed the parameters and found out when and how different network structures would evolve. By adjusting the insurance cost to the right level, the insurer can make the network formation game end up in one clique of both insured and non-insured nodes, or a clique of only insured and another of only non-insured. The condition for separating insured from non-insured are: $\beta-r<I_{l}<\beta$, and if also $\beta>r$, then the resulting network will be two cliques.
We also showed that when the insurer sets the cost such that the network ends up in two cliques, it is not the socially optimal. Because the network will suffer from the lost benefits of connections between insured and non-insured nodes. 

In model-3 we applied the model to certain real world scenarios, such as software development firms/chains, or other networks where the final product is dependent on the collaboration of several partners.
This was done by including a bonus, which is received when a node reach the desired number of links(called max-degree). This made the separation process of insured and non-insured nodes more difficult for the insurer. Due to the possibility of reaching the bonus, a node will have more incentive to establish links, and are thus more acceptable towards establishing connections to risky nodes. The condition for separating insured and non-insured nodes are: $\beta+\gamma-r<I_{l}<\beta+\frac{\gamma}{m}$. For the separation to be possible, $1-\frac{1}{m}<\frac{r}{m}$, and as we see as $\gamma$ and/or $m$ increases, this gets more and more difficult to achieve. 
I.e the conditions are strongly dependent on the max-degree. When the max-degree and bonus increases , it gets harder and harder to guarantee separation of insured and non-insured nodes.

In Model-4 we tried to implement a common feature used by insurance companies, bulk-discount, in order to see how this affected the network formation. The cost of insuring a link are now dependent on the nodes degree. We implemented this feature on both model 2 and 3, which resulted in even more incentive, or less disincentive, for an insured node to establish links with non-insured nodes. The reason is intuitive since the cost of doing so decreases as the nodes degree increases. 
When we applied the discount on model 2, the condition for ensuring separation of insured and non-insured nodes is: $N_{I}(\beta-r)<I_{l}<\beta$, where $N_{I}$ represents the number of insured nodes in the network. Since there will be a link between every insured node as long as $I_{l}<\beta$, we know that every insured node will have a degree of $N_{I}-1$, and thus to ensure that it does not connect to non-insured nodes $N_{I}(\beta-r)<I_{l}$. This is a very strong condition, because we also know that $N_{I}(\beta-r)<\beta$ for the separation to be possible. For this to be true for a large number of insured nodes, we see that $\beta$ need to be just slightly bigger than $r$. And as we see, it has been more difficult for the insurer to separate insured and non-insured, compared to model 2.

When applying the discount to model 3, this is the condition to ensure separation: $m(\beta+\gamma-r)<I_{l}<\beta+\frac{\gamma}{m}$, and as in the other models, this further complicates the separation process for the insurer. 

We also showed that the price of stability is even higher when applying discount to model 2. This is because the costs are decreasing, and thus when we have two separate cliques the potential lost payoff between them will increase.
We where not able to calculate the price of stability in model 3, because the calculation of the optimal solution is to complex when the bonus and max degree is introduced. However, we can see that since the incentive for establishing links have increased, and thus the insurer has to set a higher price to compensate for this, the price of stability will increase. The same holds when we apply the discount to model 3, because this only further complicates the problem of separation.  

In our last model we applied our model-4(discount) to an already existing model, "the symmetric connection game". In this old game it had been shown that there exists three different efficient networks that arise under certain cost conditions. In addition these networks where some times became stable. In general a clique is the most efficient if the cost of establishing links, is less than the benefit gained from indirect connections. A star is the most efficient if the cost is higher than the benefit from indirect connections, but less than the benefit of direct connections. 
It is also shown that as the number of nodes in the networks increase, the probability of the network ending up in star goes to zero. However, By applying our insurance discount to this model, we found a conjecture that says, by setting the cost to the right level, one can with high probability ensure that a star will evolve. 



\section{Conclusion}
\subparagraph{One paragraph stating what you researched and what your original contribution to the field is, after that brake into sections}
One of the many problems with cyber-insurance is to define the risk and calculate premiums, because the network structure is undefined. We have shown how insurers can, by adjusting the insurance cost, determine the result of different network formation games, and even force the network to end up in insurable topologies.  
\subsection{One section on what you researched and how you did it}
We surveyed different literature on networks and risk, and found recent literature who showed how graphs like cliques, star, meta-star and funnel, has a calculable fixation probability. In particular the results that star-structures act as an evolutionary amplifier where interesting. 
With these structures in mind, we created and analyzed different network formation games, and tried to find link-cost constraints, that enabled these structures to evolve. 
The analyzis where performed mathematically by using game-theory and confirming the results with a simulator created in NetLogo.
We created and analyzed five different models, starting with a very simple model, and step wise made it more complex and realistic. 
\subsection{One section on what are the main findings were… showinglinks across chapters (this explains why you chose thestructure you did)}
In model one to four, we found cost constraints to ensure the formation of different network structures. In each scenario, the insurer have the opportunity to set the cost, such that the game will end up in two cliques, one consisting of insured and the other of non-insured. 
In every model we added some new features that made the model more applicable to real world scenarios, and for every feature added it became more difficult for the insurer to separate the two types of nodes. This is due to the increase in incentive of establishing links, and thus the nodes became more and more acceptable towards risk. From this findings we could see that the price of stability also increased. 
We also showed that to be able to separate the networks into two cliques, the nodes must know the other nodes types. Or else, the nodes will have incentive to pretend to be an insured node. This will result in risky links between insured and non-insured nodes. 

In the last model we introduced the concept of bulk-insurance into an already existing network formation game, the connection game, and showed that this would drastically increase the probability of the network ending up in star. This is because at a point, called critical degree, the nodes optimal strategy will change from relaying on indirect connections to increase its payoff, to suddenly wanting to connect to everyone. 
We proved this by performing multiple simulations. The result showed that in a network with twenty nodes, the network would with high probability result in a star if the critical degree where six or higher. 
Whats makes this a very interesting finding, is that it is earlier proven that in the connection game, the probability of the network reaching a star goes towards zero as the number of nodes increases. However by introducing a discount, that will subsidize the center node, the game is more likely to end up in a star the more nodes are joining the network, given the critical degree is reached at the right level. \textbf{MUST DEFINE WHERE THIS LEVEL IS}

\subsection{One section on possible areas for future research}
On suggestion for future work, is mapping our models and simulations to real world scenarios in a more convincing way. Because real world network are not random, nodes may prefer to talk to nodes with high degree or low degree, i.e. the payoff function has to be changed. We have assumed additive benefits and risk, so suggestions for future work could be to introduce different risk and benefit functions, that are more applicable to the real world. 
Another interesting thing to research, is the game of choosing insurance or not, in future work this could be applied to our models.
 
\subsection{Final section reminding readers of the original contributionand significance of your research to your field}
We have shown how an insurer can determine the resulting network structure, like separate insured and non-insured nodes into two cliques, by adjusting the insurance cost parameter. In this way the network can be considered as an insurable topology, since it enables the insurer to calculate the probability of fixation. 
We found these conditions for several models, with different aspects that relate them to real world and other insurance products.

\subparagraph{TIPS og TRIKS til konklusjon}
Her oppsummerer en metoder, resultater og viktigste konklusjoner. Dette likner en del på 
sammendraget i begynnelsen, men konklusjonen er normalt fyldigere. Normalt skal ikke nytt stoff 
være med i konklusjonen, men ha vært beskrevet tidligere i rapporten. 

To summarize
 –What you researched –Nature of your main arguments –How you researched it –What you discovered –What pre-existing views were challenged
2.To provide an overview of 
The new knowledge or information discovered•The significance of your research (where is it new?)-The limitations of your thesis (concepts, data)-Speculation on the implications of these limitations-Areas for further development and research(alternative data sets; links with other fields; differentmethod applied to same data
\subparagraph{AVOID this in conclsuion}
Avoid claiming findings that you have not proventhroughout your thesis•Avoid introducing new data•Avoid hiding weaknesses or limitations in your thesis(make a virtue of showing strong analytical skills and self-critique by discussing the limitations--but don’t gooverboard on this!)• Avoid making practical recommendations (e.g. for policy).If you must include them put them in an appendix.•Avoid being too long (repetitive) or too short (sayingnothing of importance)
