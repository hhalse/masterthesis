\chapter{Summary}
\section{Discussion}
From our background study, it was revealed that the current market for cyber-insurance is far from healthy, and many have failed in attempts to establish a cyber-insurance market, also here in Norway. 
As described in the introduction, there are certain obstacles that are unique for cyber-insurance, and arguably these are the reasons why cyber-insurance has not emerged as expected. 
 However, we believe that there is a need for cyber-insurance, and that our new approach of analyzing the cyber-insurance market through graphs and network formation games could help establishing and improving the market.

We studied a variety of different network formation games, in order to find out if there were any superior network topologies that would fit as a cyber-insurance network, were ideally both the insurer and customers get a higher payoff from purchasing cyber-insurance. 
We found that star and clique networks had appropriate characteristics, not only do they have calculable fixation probability, but they could also generate better security and overall higher payoff for the nodes. With these networks in mind, we wanted to find a way of forcing networks to evolve into these structures.  We found that insurers could adjust the insurance premium in order to control the formation of networks. If the price is set to the right level, networks with calculable risk will evolve, and if the insurer is able to separate the nodes into two different networks, one consisting of trusted, insured nodes, the other of non-insured nodes, the trusted nodes can even further increase their payoff, compared to a non-trusted network. The insurer now possesses a tool for setting the insurance premium properly, possible resulting in better products for both the customer and the insurer.


We created several different models, where the first model showed a very simple and na\"{\i}ve way for the insurer to separate insured and non-insured nodes into two cliques. To make the model more applicable to real-world scenarios, we created several models, and for each model we added some new features. To get an overview of the models we created, we refer to Figure \ref{fig:Overview-of-models}.

In model 2a we made model 1 realizable, by including the parameters: expected cost of risk, insurance cost and the benefit per link. Then we analyzed the parameters and found out when and how different network structures would evolve. By adjusting the insurance cost to the right level, the insurer can make the network formation game end up in a giant clique of both insured and non-insured nodes, or a clique of only insured nodes and another of only non-insured nodes. The condition for separating insured from non-insured nodes are: $\beta-r<I_{l}<\beta$, additionally if $\beta>r$, the non-insured nodes will also form a clique, and the resulting network will be two cliques. The solution is pairwise stable, since the change in payoff is linear and non-dependent on the rest of the network, when a link is added, there is no reason to remove it later. And since the resulting network consists of one or two cliques, it is not possible to add any more links. This holds for models 1, 2, 3 and 4.
We also showed that when the insurer sets the cost such that the network ends up in two cliques, it is not the socially optimal, because the network will suffer from the lost benefits of connections between insured and non-insured nodes, i.e. it has a price of anarchy less than 1.


In model 2b, we showed that to be able to separate the networks into two cliques, the nodes must know the other nodes' types. Otherwise, the nodes will have an incentive to pretend to be an insured node, which will result in an untrusted network. We think it is reasonable to assume that nodes in a real world-scenario know whether their transactional partner has insurance or not, therefore we chose not to include this uncertainty in the other models.

In model 3 we applied the model to certain real-world scenarios, such as software development firms/chains, or other networks where the final product is dependent on the collaboration of multiple participants.
This was done by including a bonus, which is first received when a node reaches the desired number of links (called max-degree). This made the separation process of insured and non-insured nodes more difficult for the insurer. Due to the possibility of achieving the bonus, a node will have more incentive to establish links, and is thus more accepting towards establishing links with risky nodes. The conditions for separating insured and non-insured nodes in this scenario are: $\beta+\gamma-r<I_{l}<\beta+\frac{\gamma}{m}$. For the separation of insured and non-insured nodes to be possible, the following has to hold: $1-\frac{1}{m}<\frac{r}{m}$. As we see, as $\gamma$ and/or $m$ increases, this gets more and more difficult to achieve. 

In Model 4 we tried to implement a common feature used by insurance companies, bulk discount, in order to see how this affected the network formation. The cost of insuring a link is now dependent on the node's degree. 
We implemented this feature on both model 2 and 3, which resulted in even higher incentive for insured nodes to establish links with non-insured nodes. The reason is intuitive, since the cost of doing so decreases as the node degree increases. 
When we applied the discount on model 2, the conditions for ensuring separation of insured and non-insured nodes were: $N_{I}(\beta-r)<I_{l}<\beta$, where $N_{I}$ represents the number of insured nodes in the network. This condition is very strong, because for the separation to be possible the following has to hold: $N_{I}(\beta-r)<\beta$. As we see, it is now more difficult for the insurer to separate insured and non-insured nodes, compared to model 2, because now the lower boundary on the insurance cost is multiplied with the number of insured nodes in the network ($N_{I}\times(\beta-r)$).

When applying the discount to model 3, the condition to ensure separation becomes: $m(\beta+\gamma-r)<I_{l}<\beta+\frac{\gamma}{m}$, and as in the other models, this further complicates the separation process for the insurer. 

We also showed that the price of anarchy is even higher when applying discount to model 2. This is because the costs are decreasing, and thus when we have two separate cliques, the potential lost payoff between them will increase.
When we included both bonus and discount, the calculation of price of anarchy became too complex. However, we see that the incentive for establishing links has increased, and thus the insurer has to set a higher price to compensate for this, and therefore the potential price of anarchy is even higher i.e. the more incentive for link establishment you have, the harder it gets to ensure separation of the nodes.  

In our last model we applied our model 4 (discount) to an already existing model, "the symmetric connection game". In this old game it has been shown that there are three different efficient and stable networks, clique, star and an empty network, that arise under certain cost conditions. If $I_{l}<\beta-\beta^{2}$, the efficient and stable network is a clique. If $\beta-\beta^{2}<I_{l}<\beta$ a star is both stable and efficient. If $I_{l}>\beta+\frac{N-2}{2}\beta^{2}$ an empty network is both stable and efficient. In general, a clique is the most efficient if the cost of establishing links is less than the benefit gained from indirect connections. A star is the most efficient if the cost is higher than the benefit from indirect connections, but less than the benefit of direct connections. 
Unfortunately, it is proved that as the number of nodes in the networks increases, the probability of the network ending up in star goes to zero. However, when we applied our insurance discount to this model, we found conjectures saying that, by setting the cost to the right level, one can with high probability ensure that either a clique, a star or a scale-free structure will evolve. This changes the connection game drastically, because now the insurer is able to force the network into three possible network formations, where the star has a fixation probability that exceeds the cliques. The insurer can use these findings to ensure that one of the beneficial structures, star or clique evolves. If the insurer is able to force a star to evolve, this can be used to drastically increase the overall security, and at the same time minimize the overall link cost. 

\subparagraph{Limitations and future work}
One limitation to our work, and a suggestion for future work, is to map our models and simulations to real-world networks in a more convincing way. Real-world networks are not random. Nodes may prefer to talk to nodes with high degree or low degree. In addition, the decision to use additive risk were taken due to the simplicity of the function and the fact that we do not know how a real-world risk distribution actually looks like.  By introducing a complex risk function, we would only have distorted the goal of our models. i.e. suggestions for improving our models is to introduce more realistic payoff functions.

Another interesting thing to research, is the game of choosing insurance or not. In future work this should be applied to our models, but this could also possibly be too complex, and only disrupt the models.

\section{Conclusion}

So far, cyber-insurance has failed to reach its promising potential, and many have failed to establish a sustainable cyber-insurance market. We believe that cyber-insurance is an essential part of the internet economy, and that our new approach of analyzing the cyber-insurance market through graphs and network formation games could help improving and establishing a better market.  

We surveyed literature on networks and risk, and found recent literature that showed how graphs like cliques, star, super-star, funnel and meta-funnel all have a calculable fixation probability, and that stars and funnels fixation probability exceeds the one of a clique. 
With these structures in mind, we created and analyzed different network formation games, and tried to find link-cost constraints, which enabled these structures to evolve.

In models one to four, we found cost constraints to separate insured and non-insured nodes into two cliques. 
For each model, we added some new features that made the model more applicable to real world scenarios, and for every feature added, it became more difficult for the insurer to separate the two types of nodes. This is due to the increased incentive for establishing links, and thus the nodes became more and more accepting towards risk. 


In the last model, we introduced the concept of bulk insurance into an already existing network formation game, "the symmetric connection game", and showed that this enabled the insurer to determine, with high probability, when and how, cliques, stars or scale-free network would evolve. We showed that at a point, called critical degree, a node's optimal strategy would change from relying on indirect connections, to suddenly wanting to connect to everyone. If the critical degree is set to the right level, one can ensure that the different structures evolve. If the critical degree is set to a low degree, a clique will most certainly evolve, at a medium level, a star will evolve, and at a high level, a scale-free network will evolve. We proved this by performing multiple simulations, 50 simulations for every critical degree.  
What makes this a very interesting finding, is that in the connection game, earlier research has proven that as the number of nodes increases, the probability of the network reaching a star goes towards zero. However, by introducing a discount, that will subsidize the center node, one can drastically increase the probability of the network ending up in a star.

In summary, we have shown how insurers can determine the resulting structure of insurance networks, by adjusting the insurance cost, for several network formation games. We have also showed how insurers can be assisted in calculating the overall probability of fixation.
We found these conditions for several models, with different properties that relate them to the real world and other insurance products. We believe our findings can help the cyber-insurance market evolve into a viable and better market.

