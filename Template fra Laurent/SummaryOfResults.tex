\chapter{Summary of results/conclusion}
Model-1: Showed a very simple and naive way for the insurer to separate between insured and non-insured nodes.

Model-2: Made model-1 realizable, by including parameters and making it apply for multiple nodes. We then analyzed the parameters and found out when and how different network structures would evolve. By adjusting the insurance cost to the right level, the insurer can make the network formation game end up in, one clique of both insured and non-insured nodes, or a clique of only insured and another of only non-insured. 
We also showed that when the insurer sets the cost such that the network ends up in two cliques, then this has a cost compared to the most efficient network. The cost of stability, this is because in overall the network will suffer from the lost benefits of connections between insured and non-insured nodes.

Model-3: In this model we tried to apply the model to certain real world scenarios, such as software development firms/chains, or other networks where the final product is dependent on the collaboration of several parts.
This was done by including a bonus, which where received when you reached the desired number of links(called max-degree). This made the separation process of insured and non-insured nodes, more difficult for the insurer. Because the nodes now have more incentive to establish links, and are thus more acceptable towards risk. We found the conditions for the different network structures to evolve, and showed that these where very strongly dependent on the max-degree. And when the max-degree increases, it gets harder and harder to guarantee two separate cliques. 

Model-4:
In this model we tried to make the model more comparable to other insurance products, by including a bulk-discount. We did this on both model 2 and 3. This resulted in even more incentive, or less disincentive, for insured node to establish links with non-insured nodes, since the cost of doing so decreases linearly with the nodes degree. 
We found the different conditions for when the different networks would evolved, and showed that when applying the discount to model 3, it is very hard for the insurer to ensure two seperate cliques. 
We also showed that the price of stability is even higher when applying discount to model 2. This is because the costs are decreasing, and thus the potential payoff that are missing, when we have two separate cliques, are increasing. 

Model-5:
In this model we used the findings from a well know game called the symmetric connection game, and applied these to our scenario with insurance. Since the game is well known in network formation research, this game have allready been analyzed thoroughly, some of these findings are very interesting. They show how a star, under cost-conditions, is the most efficient, and can also be stable. But they also show that as the number of nodes increases, the probability of reaching a star approaches zero.
By applying our insurance discount to this model, we found a conjecture that says, by setting the cost to the right level, one can most certainly ensure that a star will evolve. 
