\begin{titlingpage}

\noindent
\begin{tabular}{@{}p{4cm}l}
\textbf{Title:} 	& \thetitle \\
\textbf{Students:}	& Håvard Råmundal Halse \& Jonas Hoemsnes \\
\end{tabular}

\vspace{4ex}
\noindent\textbf{Problem description:}

Security breaches are increasingly prevalent in the Internet age causing huge financial losses
for companies and their users. Cyber-insurance is a powerful economic concept that can help
companies in the fight against such malicious behavior. Earlier research suggests that cyber-
insurance has failed to reach its promising potential, although the concept of cyber-insurance has
been around since the 1980s. The researchers claims that a functional model for cyber-insurance has to handle its unique problems regarding interdependent security, correlated-risk and asymmetrical-information. These challenges can be described and analyzed by network graphs, and positively some graphs will yield overall higher security (insurable topologies) than other graphs. In order to cater for cyber-insurance, it is essential to understand how to create new or transform existing networks to insurable topologies.

The students will:
\begin{itemize}

\item conduct a background study and a market survey to validate the current state of cyber-
insurance
\item study and characterize graphs describing insurable topologies
\item build a model of network formation which gives rise to such insurable topologies
\item apply the model to investigate a realistic ecosystem, e.g., cloud computing

\end{itemize}
\vspace{2ex}

\noindent
\begin{tabular}{@{}p{4cm}l}
\textbf{Supervisor:}			& \thesupervisor \\
\textbf{Responsible professor:} 	& \theprofessor \\
\end{tabular}

\end{titlingpage}