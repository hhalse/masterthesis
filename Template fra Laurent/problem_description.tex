\begin{titlingpage}

\noindent
\begin{tabular}{@{}p{4cm}l}
\textbf{Title:} 	& \thetitle \\
\textbf{Students:}	& Håvard Råmundal Halse \& Jonas Hoemsnes \\
\end{tabular}

\vspace{4ex}
\noindent\textbf{Problem description:}

Security breaches are increasingly prevalent in the Internet age causing huge financial losses for companies and their users. Cyber-insurance is a powerful economic concept that could potentially help companies in the fight against such malicious behavior. Earlier research suggests that cyber-
insurance has failed to reach its promising potential, although the concept has been around since the 1980s. Researchers have proposed and analyzed several functional models of cyber-insurance in order to cope with or mitigate its unique characteristics: Interdependent security, correlated risk and asymmetrical information. However, the cyber-insurance market is still in a dormant phase.

A specific problem with cyber-insurance is determining the overall risk of the network structure to be insured. If cyber-insurance networks were describable and analyzable through graphs, the calculation of overall risk would be much easier. Are there specific graphs with properties that will help improve the market? Is it possible for insurers to incentivize companies forming networks with such beneficial graph structure or transforming existing networks into these superior structures?

The students will:

\begin{itemize}
 
\item conduct a background study and a market survey to validate the current state of cyber-
insurance
\item study and characterize graphs with desirable properties regarding cyber-insurance
\item build and analyze a model of network formation which gives rise to such network structures
\end{itemize}


\vspace{2ex}

\noindent
\begin{tabular}{@{}p{4cm}l}
\textbf{Supervisor:}			& \thesupervisor \\
\textbf{Responsible professor:} 	& \theprofessor \\
\end{tabular}

\end{titlingpage}