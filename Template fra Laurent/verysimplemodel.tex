

\section{Very Simple Model}


As a starting point the model is highly simplified in order to show the concept of how cyber-insurance can be used to create an insurable topology. Through out this chapter new features will be added to the model to make it more realistic and applicable. To begin, the model is formulated as follows.
A set of $n$ agents are randomly chosen to be insured or not. They all get their own fixed income, and by connecting to other agents they will receive a benefit resulting in higher payoff. Non-insured agents will have a risk of failure i.e. an expected cost of failure. Therefore if an insured agents chooses to connect to a non-insured agent they will also suffer from this expected cost of failure. In other words, the model follows a rule that insured agents are only willing to connect to other insured agents and non-insured agents can only connect with each other. In addition we apply the assumption that each node goes through the whole graph to decide whether to establish a connection or not. Since the decision is bidirectional, i.e. each agent must agree to establish the connection, the resulting graph will always be two fully connected cliques, one consisting of a insured agents and the other of non-insured agents. 


This dichotomy represents a trusted environment for the insured nodes, because they are able to trust each other since everyone is protected from risks such as financial catastrophe. These agents will benefit from each connection without having to worry about contagious risks from the connected agents. 
An agent in the non-insured clique will also receive the aggregated benefits from the connections, however each of the connection has a probability of failure. Hence this environment is not trusted, and a decision on whether to connect always involves some risks. 

In many situations agents, such as companies are in a situation where they have to establish connections. One example can be a non-insured company needing to outsource certain tasks to remain competitive, if all the potential companies for outsourcing are insured, the company will have a strong incentive to also buy insurance in order to be able to establish a connection. Hence this model, although very simple, shows an insurable topology where insured agents benefit from being insured. 

There are some limitations of the model, among others it fails to reflect the dynamics of a real world scenario, where each node will have different variables with different values. In addition, each node have a complete overview of the other nodes status. i.e. the problem with information asymmetry is not taken into account. 