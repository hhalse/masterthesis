\pagestyle{empty}
\begin{abstract}
Cyber-insurance is a powerful economic concept that can help companies in the fight against cyber-crime. From the early 80s, several researchers claimed that cyber-insurance had a bright future, were it would become a huge economical tool for handling residual cyber-risks.

The market study for this thesis revealed that both the European and US cyber-insurance market have failed to grasp its promising potential. The US market has matured more than the European market, but both still have failed compared to the potential: to fully grasp this potential they need some innovative approaches to handle the unique problems of cyber-insurance.

The thesis proposes several cyber-insurance network formation models, and uses game theory and a simulation tool, Netlogo, to analyze these models. In every model, new properties that relate the model to the real-world and real-world insurance products are introduced. The results show that insurers can use the insurance premium as a tool for determining the resulting formation of the network. If the premium is set at the right level, certain structures will evolve. In recent literature these structures have shown to possess properties that make them particularly good for cyber-insurance networks. Cases in point are minimizing the average cost, enabling the insurer to calculate the overall risk and possibly increased overall security and utility.
We believe our findings could help the cyber-insurance market evolve, by giving the insurers a proper tool to better analyze and control formation of cyber-insurance networks. 

Further work should try mapping our models and simulations to real world networks more extensively. This could be achieved by finding and introducing more realistic risk functions, and by letting nodes choose their neighbors by preference, not randomly. 



\end{abstract}