\pagestyle{empty}
\begin{abstract}
Cyber-insurance is a powerful economic concept that can help companies in the fight against cyber-attacks. From the early 90s, most researchers claimed that cyber-insurance had a positive future; it would become a huge economical tool for handling residual cyber-risks. 

The market study of the thesis revealed that both the European and US cyber-insurance market is have failed to grasp its promising potential. The US-market has matured more compared to the European-market, but both still have failed compared to the potential market, too fully grasp this potential they need some innovative approaches to handle the unique problems of cyber-insurance.

The thesis proposes several cyber-insurance network formation models, and uses game theory and a simulation tool, Netlogo, to analyze these models. In every model, there are introduced new properties that relate the model to the real world and real insurance products. The results show that insurers can use the insurance-premium as a tool for determining the resulting formation of the network. If the premium is set to the right level, certain structures will evolve, in recent literature these formations have shown to possess properties who make them particularly good for cyber-insurance network. Such as minimizing the average cost, enabling the insurer to calculate the overall risk and possibly increased overall security and utility. 

We believe our findings will help the cyber-insurance market evolve, by giving the insurers a proper tool to better analyze and control their cyber-insurance network, in this way helping the cyber-insurance market reach its promised potential.  

Further work should try mapping our models and simulations to real world networks in a more convincing way. This could be achieved by finding and introducing better suited risk functions, and by letting nodes choose their neighbors by preference, not randomly. 


\end{abstract}